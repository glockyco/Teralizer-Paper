\subsection{RQ6: What are the causes of unsuccessful generalization attempts under real-world conditions?}
\label{sec:limitations-eval-extended}

RQ5 established exclusion causes under controlled conditions.
RQ6 now examines how \ToolTeralizer{} performs
on 632 Java projects from the RepoReapers dataset
(Section~\ref{sec:experimental-framework})
to identify generalization barriers in real-world projects.
Section~\ref{sec:project-level-exclusions-extended} examines project-level exclusions.
Section~\ref{sec:test-assertion-generalization-exclusions-extended} then examines
test, assertion, and generalization exclusions,
comparing exclusion rates to controlled settings.
Full project-level exclusions and
partial test, assertion, and generalization exclusions are interconnected:
when filtering or failures exclude all tests, assertions, or generalizations in a project,
the project is excluded.
Understanding exclusion patterns guides future work toward addressing the most impactful limitations.

\subsubsection{Project-Level Exclusions}
\label{sec:project-level-exclusions-extended}

Only 11 of 632 projects (1.7\%) successfully complete
all five processing stages (Table~\ref{tab:processing-failures}),
revealing substantial barriers to real-world applicability.
To understand where and why processing fails,
we examine exclusions stage by stage,
distinguishing between internal causes
(caused by configured resource limits or limitations of \ToolTeralizer{}),
external causes
(caused by \ToolTeralizer{}'s dependencies: JUnit, Spoon, \ToolJPF{}/\ToolSPF{}, \ToolJacoco{}, and \ToolPit{}),
and mixed causes
(influenced by both internal and external factors).
This reveals which barriers are addressable through future improvements of \ToolTeralizer{}
and which ones reflect less actionable limitations in \ToolTeralizer{}'s dependencies.

\begin{table}[b]
\caption{%
    Project-level exclusions by stage and cause for the \VariantImprovedC{} generalization strategy in RepoReapers projects.
    Internal causes are due to configured resource limits
    or current limitations of \ToolTeralizer{}.
    External causes are due to \ToolTeralizer{}'s dependencies
    (i.e., JUnit, Spoon, \ToolJPF{} / \ToolSPF{}, \ToolJacoco{}, and \ToolPit{}).
    Mixed causes are influenced by both internal as well as external factors.
}
\label{tab:processing-failures}
\centering
\begin{tabular}{rllr}
\toprule
\# & Type & Cause of Project-level Exclusion & Count \\
\midrule
\multicolumn{4}{l}{\textit{\makebox[15em][l]{Stage 1 + 2 - Project Analysis:} 632 projects\quad{}130 inclusions\quad{}502 exclusions\quad{}20.6\% inclusion rate}} \\
\midrule
1 & Mixed & all assertions excluded due to filter rejections & 255 \\
2 & Mixed & all tests excluded due to filter rejections and failures & 129 \\
3 & Internal & timeout exceeded (60 seconds per \VariantOriginal{} test suite) & 48 \\
4 & Internal & JUnit reports not found & 31 \\
5 & Internal & compilation outputs not found & 18 \\
6 & External & JUnit execution error during test execution & 13 \\
7 & External & Spoon execution error during test analysis & 8 \\
\midrule
\multicolumn{4}{l}{\textit{\makebox[15em][l]{Stage 3 - Specification Extraction:} 130 projects\quad{}117 inclusions\quad{}\phantom{0}13 exclusions\quad{}90.0\% inclusion rate}} \\
\midrule
8 & Mixed & all assertions excluded due to earlier filter rejections and new failures & 11 \\
9 & External & Spoon execution error during test instrumentation & 1 \\
10 & Internal & timeout exceeded (60 seconds per \VariantInitial{} test suite) & 1 \\
\midrule
\multicolumn{4}{l}{\textit{\makebox[15em][l]{Stage 4 - Generalized Test Creation:} 117 projects\quad{}114 inclusions\quad{}\phantom{00}3 exclusions\quad{}97.4\% inclusion rate}} \\
\midrule
11 & Internal & all generalizations excluded due to filter rejections and failures & 3 \\
\midrule
\multicolumn{4}{l}{\textit{\makebox[15em][l]{Stage 5 - Test Suite Reduction:} 114 projects\quad{}\phantom{0}11 inclusions\quad{}103 exclusions\quad{}\phantom{0}9.6\% inclusion rate}} \\
\midrule
12 & Internal & JaCoCo outputs not found & 40 \\
13 & Internal & timeout exceeded (300 seconds per test suite variant) & 40 \\
14 & External & \ToolPit{} execution error during mutation testing & 16 \\
15 & Internal & \ToolPit{} reports not found & 4 \\
16 & Internal & failed to process \ToolPit{} reports & 2 \\
17 & External & JaCoCo execution error during coverage collection & 1 \\
\midrule
\multicolumn{4}{l}{\textit{\makebox[15em][l]{Overall:} 632 projects\quad{}\phantom{0}11 inclusions\quad{}621 exclusions\quad{}\phantom{0}1.7\% inclusion rate}} \\
\bottomrule
\end{tabular}
\end{table}

The pipeline shows a distinct funnel pattern with two major barriers
at the early and late processing stages,
separated by high-success middle stages.
Stage~1+2 (project analysis) excludes 79.4\% of projects (502 of 632),
forming the first major barrier.
Projects that pass this initial filter progress largely successfully through
Stage~3 (specification extraction, 90.0\% pass rate: 117 of 130 projects)
and Stage~4 (generalized test creation, 97.4\% pass rate: 114 of 117 projects).
The second major barrier emerges at Stage~5 (test suite reduction via mutation testing),
where 90.4\% of remaining projects are excluded (103 of 114 projects),
leaving only 11 projects (1.7\% of the original 632) to complete all stages.
This pattern suggests that the core generalization mechanisms (Stages 3--4) operate reliably
when projects match current tool capabilities,
but both early filtering and final mutation testing highlight substantial practical challenges.

\paragraph{Stage 1 + 2 Exclusions:}

Project analysis excludes 502 of 632 projects (79.4\%),
with the primary failure mode being complete absence of suitable generalization candidates.
Cases where all assertions are excluded affect 255 projects (40.3\% of stage input),
while all tests being excluded affects 129 projects (20.4\%).
For assertion exclusions, all 255 result from filter rejections,
indicating that these projects contain only assertion patterns
that are currently unsupported by \ToolTeralizer{}
or beyond the current capabilities of the underlying symbolic analysis performed by \ToolSPF{}
(detailed in Section~\ref{sec:test-assertion-generalization-exclusions-extended}).
For test exclusions, 116 projects (89.9\% of the 129) stem from filter rejections,
6 projects (4.7\%) from failures during test analysis (e.g., missing test report files),
and 7 projects (5.4\%) from a combination of both.

Further internal exclusions
caused by configured timeouts (60 seconds per project) and output detection failures
affect 97~projects (15.3\%).
Timeout-based exclusions can be observed in 48~projects (7.6\%)
that have particularly long-running \VariantOriginal{} test suites.
Output detection failures affect 49~projects (7.8\%),
split between failed JUnit report detection (31~projects, 4.9\%)
and failed compilation output detection (18~projects, 2.8\%).
Both types of output detection failures
are caused by projects that store these outputs in non-standard output directories.
Thus, these exclusions indicate that current search heuristics used by \ToolTeralizer{}
cannot accommodate the full diversity of real-world project structures.

External execution errors affect 21 projects (3.3\%):
13 projects (2.1\%) encounter JUnit execution errors during test execution,
and 8 projects (1.3\%) encounter Spoon execution errors during test analysis.
These failures stem from \ToolTeralizer{}'s dependencies
and, therefore, lie outside the direct control of \ToolTeralizer{}.

\paragraph{Stage 3 + 4 Exclusions:}

The 130 projects that complete project analysis
face substantially lower exclusion rates in the two subsequent stages.
This confirms that early filtering successfully identifies viable generalization candidates.
Stage~3 (specification extraction) excludes 13 projects (10.0\%)
due to \ToolSPF{} execution errors, \ToolTeralizer{} errors, or exceeded resource limits
(detailed in Section~\ref{sec:test-assertion-generalization-exclusions-extended}).
One external failure represents a Spoon execution error during test instrumentation,
and one internal failure occurs due to a test suite execution timeout
(60 seconds per \VariantInitial{} test suite).
%
Stage~4 (generalized test creation) shows the lowest failure rate in the pipeline:
only 3 of 117 projects (2.6\%) are excluded,
all due to filter rejections that exclude all generalizations.
The high inclusion rates at both stages
(90.0\% and 97.4\% respectively)
demonstrate that projects containing suitable assertions
generally proceed successfully through specification extraction and test generation,
supporting the pipeline's core generalization mechanisms.

\paragraph{Stage 5 Exclusions:}

Test suite reduction via mutation testing
represents the second major exclusion barrier,
excluding 103 of 114 projects that reach this stage (90.4\%).
Unlike Stage~1 + 2 failures that are primarily caused by proactive filtering
of unsuitable tests and assertions,
Stage~5 failures stem mainly from failures to detect required coverage and mutation testing reports,
exceeded timeouts (300 seconds per test suite variant), and external tool failures.

In total, output detection failures affect 44 projects (38.6\% of stage input).
\ToolJacoco{} reports in non-standard locations cause 40 of these exclusions,
while the remaining 4 projects are due to \ToolPit{} reports in non-standard locations.
These failures mirror the output detection issues seen in Stage~1 + 2,
further supporting the observation that real-world projects organize build artifacts more diversely
than \ToolTeralizer{}'s current output detection heuristics accommodate.

All 40 projects that are excluded due to exceeded runtime limits
already reach the configured timeout (300 seconds per test suite variant)
when performing mutation testing for the \VariantOriginal{} test suite.
No additional exclusions occur during mutation testing of the test suite
created by the \VariantImprovedC{} generalization strategy.

External execution errors affect 17 projects (14.9\%):
16~projects encounter \ToolPit{} errors during mutation testing
and 1~project encounters a \ToolJacoco{} error during coverage collection.
Two additional projects (1.8\%) fail during \ToolPit{} report parsing,
where \ToolTeralizer{} fails to successfully process the generated mutation reports.

% \paragraph{Actionability of Project-Level Exclusions:}

% Grouping these 621 project-level exclusions by underlying cause
% reveals varying degrees of actionability.
% Filter-based exclusions affect 384 projects (61.8\%)
% where all tests or assertions are excluded
% because projects contain no patterns currently supported by \ToolTeralizer{}.
% Addressing these requires expanding tool capabilities,
% as detailed in Section~\ref{sec:test-assertion-generalization-exclusions-extended}.
% Project structure detection failures affect 93 projects (15.0\%),
% split between missing outputs during project analysis (49 projects)
% and undetectable mutation testing artifacts (44 projects).
% These failures could be reduced through more robust search heuristics
% that accommodate diverse project layouts.
% Timeout failures affect 88 projects (14.2\%),
% representing engineering trade-offs between analysis depth and execution time.
% External tool failures affect 38 projects (6.1\%),
% stemming from dependencies outside direct control.
% The remaining 18 projects (2.9\%) encounter various internal implementation issues,
% offering limited improvement potential given their small contribution to overall failures.

% This distribution indicates that expanding filtering capabilities
% and improving project structure detection
% represent the highest-impact opportunities for increasing applicability,
% potentially recovering 477 of 621 excluded projects (76.8\%).

\subsubsection{Test, Assertion, and Generalization Exclusions}
\label{sec:test-assertion-generalization-exclusions-extended}

% % | Category                     | Count  | Where It Appears |
% % |------------------------------|--------|------------------|
% % | Included                     | 33,390 | Included column  |
% % | Filtering (REJECT decisions) | 40,602 | Filtering column |
% % | Before filtering             | 7,502  | Failures column  |
% % | TestAnalysisTask failures    | 321    | Failures column  |
% % | Filtering exceptions         | 19     | Failures column  |
% % | Total                        | 81,834 | Total column     |

Beyond project-level exclusions,
individual tests, assertions, and generalizations are excluded
throughout pipeline processing via filtering and due to processing failures.
Table~\ref{tab:exclusions-breakdown-extended} quantifies
exclusion rates across all 632 projects,
distinguishing between filtering-based and failure-based exclusions.
Table~\ref{tab:exclusions-filtering-extended} provides
further details about the causes of filter rejections across the three levels.
For brevity, we only include the results for the \VariantImprovedC{} generalization strategy.
Since most exclusions occur in the \VariantShared{} processing stages,
differences across generalization strategies are minor.
Full results are available in our replication package~\cite{replicationpackage}.

\paragraph{Test-level Exclusions}
\label{par:real-world-test-exclusions}

Only 40.8\% of real-world tests are included (33,385 of 81,810),
compared to 83.1\% under controlled conditions
(Table~\ref{tab:exclusions-breakdown-extended} vs.\ Table~\ref{tab:exclusions-breakdown}).
Of the 48,425 excluded tests, 49.6\% are rejected through filtering
while 9.6\% are excluded due to processing failures.
The high prevalence of filtering-based exclusions indicates that
generalization of real-world tests commonly requires capabilities
that are beyond what \ToolTeralizer{} currently supports.

The \texttt{NoAssertions} filter shows the highest exclusion rate,
rejecting 41.3\% of real-world tests versus 10.3\% under controlled conditions.
However, RQ5 identified 86.3\% of \texttt{NoAssertions} rejections
in developer-written tests to be false positives:
tests that are rejected by the \texttt{NoAssertions} filter but actually contain assertions in helper methods
which \ToolTeralizer{}'s interprocedural static analysis does not detect.
Given that RepoReapers exclusively contains developer-written tests,
these rejections are likely to contain a high rate of false positives as well.

The \texttt{NonPassingTest} filter rejects 11.8\% of real-world tests
compared to the 6.6\% rejection rate under controlled conditions.
As explained in Section~\ref{sec:limitations-eval},
this filter operates at the test class level
because \ToolPit{} requires a green test suite
but only supports class-level exclusions.
As a result, the filter rejects all 8,741 test methods from 974 classes containing at least one failing test.
Of these, 4,709 (54\%) actually failed during execution,
while 4,032 (46\%) passed but were rejected due to class-level filtering.
Analysis of the 4,709 failing tests reveals that failures stem primarily from
infrastructure and environment issues rather than broken tests:
18.8\% encounter missing dependencies (NoClassDefFoundError),
14.1\% fail due to unavailable external services (Redis, MongoDB, MySQL),
10.0\% encounter null pointer exceptions, etc.
Only 18.3\% of failures are genuine assertion failures
where tests execute to completion but produce incorrect results.

The \texttt{TestType} filter rejections increase
from 0.8\% under controlled conditions to 12.5\% in real-world projects.
Under controlled conditions,
all rejections are @ParameterizedTest annotations in \DatasetCommonsDev{}.
In contrast, all 9,277 RepoReapers rejections are legacy JUnit 3 test methods
that use the JUnit 3 naming convention (method names starting with ``test'')
instead of @Test annotations,
occurring across 70 different RepoReapers projects.

\paragraph{Assertion-level Exclusions}
\label{par:real-world-assertion-exclusions}

Assertion exclusions differ substantially between
controlled and real-world conditions.
Only 0.6\% of assertions in RepoReapers projects are included (711 of 122,153)
versus 47.8\% under controlled conditions.
Filtering accounts for 99.1\% of exclusions while failures represent 0.3\%,
confirming that the primary barriers are known limitations of \ToolTeralizer{}
rather than unexpected failures during processing.

The \texttt{MissingValue} filter shows the largest assertion-level rejection rate of 57.9\%,
compared to 24.7\% under controlled conditions.
Rejections have three underlying causes.
First, 41\% involve unsupported assertion types
(assertThat, assertNull, fail, etc.)
where method identification is not attempted.
Second, 26\% involve assertions where the actual value is not a method invocation
or cannot be traced back to a method declaration,
including field accesses, comparison expressions, and literal values.
Third, 33\% identify a method call but Spoon cannot resolve its declaration.

The \texttt{ParameterType} filter rejects 49.4\% of assertions
versus 15.4\% under controlled conditions.
This difference reflects dataset characteristics:
the controlled dataset used methods with primarily numeric and boolean parameters.
In contrast, real-world projects show
52.6\% no-parameter methods,
26.9\% object and array parameters,
and only 19.8\% numeric and boolean parameters.
No-parameter methods cannot benefit from input generalization,
while methods with object and array parameters require capabilities
beyond \ToolTeralizer{}'s current support for numeric and boolean types.

The \texttt{ReturnType} filter defers on 57.9\% of assertions
where the method is unknown (matching the \texttt{MissingValue} rejection rate)
and rejects 32.6\% of all assertions.
Among methods with known return types (42.1\% of all assertions),
52.7\% return objects, 46.6\% return numeric or boolean types,
and 0.7\% return other types (char, arrays, void).
This contrasts with controlled conditions where 98.3\% of methods returned numeric and boolean types.

\begin{table}
  \caption{Exclusion results for \VariantImprovedC{} in the RepoReapers projects.}
  \label{tab:exclusions-breakdown-extended}
  \begin{tabular}{lrrrr}
    \toprule
    & & & \multicolumn{2}{c}{Excluded} \\
    \cmidrule(lr){4-5}
    Level & Total & Included & Filtering & Failures \\
    \midrule
    Test & 81,834 & \phantom{}33,390\; (40.8\%) & \phantom{0,0}40,621\; (49.6\%) & \phantom{}7,823\; (\phantom{0}9.6\%) \\
    Assertion & 122,166 & \phantom{0,}711\; (\phantom{0}0.6\%) & \phantom{}121,073\; (99.1\%) & \phantom{0,0}382\; (\phantom{0}0.3\%) \\
    Generalization & 239 & \phantom{0,}206\; (86.2\%) & \phantom{0,0,}23\; (\phantom{0}9.6\%) & \phantom{0,}10\; (\phantom{0}4.2\%) \\
    \bottomrule
  \end{tabular}
\end{table}
\begin{table}[H]
  \caption{Filtering results of the extended dataset for tests, assertions, and generalizations.}
  \label{tab:exclusions-filtering-extended}
  \begin{tabular}{lllrrrr}
    \toprule
    Variant & Type & Filter Name & Total & \multicolumn{1}{c}{Accept} & \multicolumn{1}{c}{Defer} & \multicolumn{1}{c}{Reject} \\
    \midrule
    \VariantShared{} & Test & NonPassingTest & 74,332 & 65,579\; (88.2\%) & - & 8,753\; (11.8\%) \\
    \VariantShared{} & Test & TestType & 74,332 & 65,052\; (87.5\%) & - & 9,280\; (12.5\%) \\
    \VariantShared{} & Test & NoAssertions & 56,853 & 33,390\; (58.7\%) & - & 23,463\; (41.3\%) \\
    \midrule
    \VariantShared{} & Assertion & AssertionType & 122,166 & 92,996\; (76.1\%) & - & 29,170\; (23.9\%) \\
    \VariantShared{} & Assertion & ExcludedTest & 122,166 & 101,519\; (83.1\%) & - & 20,647\; (16.9\%) \\
    \VariantShared{} & Assertion & MissingValue & 122,166 & 51,430\; (42.1\%) & - & 70,736\; (57.9\%) \\
    \VariantShared{} & Assertion & ParameterType & 122,166 & 5,393\; (\phantom{0}4.4\%) & 56,484\; (46.2\%) & 60,289\; (49.4\%) \\
    \VariantShared{} & Assertion & ReturnType & 122,166 & 11,650\; (\phantom{0}9.5\%) & 70,736\; (57.9\%) & 39,780\; (32.6\%) \\
    \midrule
    \VariantImprovedC{} & Generalization & NonPassingTest & 229 & 206\; (90.0\%) & - & 23\; (10.0\%) \\
    \bottomrule
  \end{tabular}
\end{table}

The \texttt{AssertionType} filter rejects 23.9\% of real-world assertions
versus 2.6\% under controlled conditions.
This increase stems from more diverse assertion usage in real-world test code.
For example, assertEquals accounts for 83.6\% of assertions
in controlled conditions, primarily due to the large number
of \ToolEvoSuite{}-generated tests which use assertEquals in 84.7\% of cases.
In contrast, assertEquals accounts for only 54.5\% of assertions
in the RepoReapers projects.
The most common unsupported assertion types are assertThat (8.3\%),
assertNotNull (6.1\%), fail (3.3\%), and assertNull (3.1\%).
In controlled conditions, these four types collectively account for only 0.8\% of assertions.

The \texttt{ExcludedTest} filter shows a 3$\times$ increase
from 5.5\% under controlled conditions to 16.9\% in real-world projects,
reflecting the corresponding increase in test-level exclusions
from 16.9\% to 59.2\%.

\paragraph{Generalization-level Exclusions}
\label{par:real-world-generalization-exclusions}

Of 239 generalization attempts in the RepoReapers projects,
206~(86.2\%) succeed.
The 33 exclusions result from
\texttt{NonPassingTest} filter rejections (23 cases, 9.6\%)
and test report detection failures (10 cases, 4.2\%).
The 86.2\% inclusion rate is comparable to the 83.8\% rate under controlled conditions.
However, only 0.2\% of all assertions result in an included generalization
(206 of 122,153), compared to 40.1\% under controlled conditions (11,597 of 28,923).
These results demonstrate that
the core generalization mechanism operates reliably when applicable,
but real-world applicability is limited by three primary barriers:
limited type and assertion pattern support
(99.4\% of assertions excluded by filters at earlier stages),
non-standard project structures
(output detection failures exclude 14.7\% of projects at Stages~1+2 and~5),
and resource constraints
(timeout exclusions affect 14.1\% of projects across all stages).

\rqanswerbox{6}{
  Fully automated generalization of real-world test suites encounters significant challenges.
  Only 206 of 122,153 assertions (0.2\%) are successfully generalized
  (compared to 40.1\% under controlled conditions)
  and only 11 of 632 projects (1.7\%) complete all processing stages.
  The core generalization mechanism operates reliably when applicable:
  generalization-level success rates are comparable (86.2\% real-world vs 83.8\% controlled),
  and projects that pass early filtering also complete specification extraction (90.0\%) and generalized test creation (97.4\%).
  However, three barriers prevent higher overall success rates:
  limited type and assertion support causes 99.4\% of assertions to be filtered
  (e.g., \texttt{MissingValue}: 57.9\%, \texttt{ParameterType}: 49.4\%, \texttt{ReturnType}: 32.6\%),
  non-standard project structures prevent output detection in 14.7\% of projects,
  and execution timeouts exclude 14.1\% of projects.
}
