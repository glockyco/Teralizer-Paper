\subsection{RQ6: What are the causes of unsuccessful generalization attempts under real-world conditions?}
\label{sec:limitations-eval-extended}

The extended evaluation applies \ToolTeralizer{} to 632 real-world Java projects
from the RepoReapers dataset (Section~\ref{sec:experimental-framework}),
thereby moving beyond the controlled conditions of the primary dataset
to examine more challenging, practical application scenarios.
Table~\ref{tab:processing-failures}
reveals progressive attrition throughout the processing pipeline:
only 10 projects (0.9\%) successfully complete all processing stages.
While many exclusions are still caused by filtering,
real-world projects face additional barriers throughout the different processing stages,
from dependency resolution through final mutation testing.
This evaluation identifies which barriers are internal versus external to \ToolTeralizer{},
and which ones require engineering effort versus research advances to resolve.

\begin{table}[b]
\caption{%
    Project-level exclusions by stage and cause for the \VariantImprovedC{} generalization strategy in RepoReapers projects.
    Internal causes are due to configured resource limits
    or current limitations of \ToolTeralizer{}.
    External causes are due to \ToolTeralizer{}'s dependencies
    (i.e., JUnit, Spoon, \ToolJPF{} / \ToolSPF{}, \ToolJacoco{}, and \ToolPit{}).
    Mixed causes are influenced by both internal as well as external factors.
}
\label{tab:processing-failures}
\centering
\begin{tabular}{rllr}
\toprule
\# & Type & Cause of Project-level Exclusion & Count \\
\midrule
\multicolumn{4}{l}{\textit{\makebox[15em][l]{Stage 1 + 2 - Project Analysis:} 632 projects\quad{}130 inclusions\quad{}502 exclusions\quad{}20.6\% inclusion rate}} \\
\midrule
1 & Mixed & all assertions excluded due to filter rejections & 255 \\
2 & Mixed & all tests excluded due to filter rejections and failures & 129 \\
3 & Internal & timeout exceeded (60 seconds per \VariantOriginal{} test suite) & 48 \\
4 & Internal & JUnit reports not found & 31 \\
5 & Internal & compilation outputs not found & 18 \\
6 & External & JUnit execution error during test execution & 13 \\
7 & External & Spoon execution error during test analysis & 8 \\
\midrule
\multicolumn{4}{l}{\textit{\makebox[15em][l]{Stage 3 - Specification Extraction:} 130 projects\quad{}117 inclusions\quad{}\phantom{0}13 exclusions\quad{}90.0\% inclusion rate}} \\
\midrule
8 & Mixed & all assertions excluded due to earlier filter rejections and new failures & 11 \\
9 & External & Spoon execution error during test instrumentation & 1 \\
10 & Internal & timeout exceeded (60 seconds per \VariantInitial{} test suite) & 1 \\
\midrule
\multicolumn{4}{l}{\textit{\makebox[15em][l]{Stage 4 - Generalized Test Creation:} 117 projects\quad{}114 inclusions\quad{}\phantom{00}3 exclusions\quad{}97.4\% inclusion rate}} \\
\midrule
11 & Internal & all generalizations excluded due to filter rejections and failures & 3 \\
\midrule
\multicolumn{4}{l}{\textit{\makebox[15em][l]{Stage 5 - Test Suite Reduction:} 114 projects\quad{}\phantom{0}11 inclusions\quad{}103 exclusions\quad{}\phantom{0}9.6\% inclusion rate}} \\
\midrule
12 & Internal & JaCoCo outputs not found & 40 \\
13 & Internal & timeout exceeded (300 seconds per test suite variant) & 40 \\
14 & External & \ToolPit{} execution error during mutation testing & 16 \\
15 & Internal & \ToolPit{} reports not found & 4 \\
16 & Internal & failed to process \ToolPit{} reports & 2 \\
17 & External & JaCoCo execution error during coverage collection & 1 \\
\midrule
\multicolumn{4}{l}{\textit{\makebox[15em][l]{Overall:} 632 projects\quad{}\phantom{0}11 inclusions\quad{}621 exclusions\quad{}\phantom{0}1.7\% inclusion rate}} \\
\bottomrule
\end{tabular}
\end{table}
% \begin{table}[H]
  \caption{Number of processing failures and remaining projects per processing stage.}
  \label{tab:processing-failures-per-stage}
  \begin{tabular}{l r r}
    \toprule
    Processing Stage & Failures & Remaining Projects \\
    \midrule
    Total projects & - & 1160\; (\phantom{.}100 \%) \\
    SETUP\_PROJECT & 355 & 805\; (69.4 \%) \\
    BUILD\_PROJECT\_ORIGINAL & 189 & 616\; (53.1 \%) \\
    BUILD\_SPOON\_MODEL & 8 & 608\; (52.4 \%) \\
    EXECUTE\_TESTS\_ORIGINAL & 61 & 547\; (47.2 \%) \\
    COLLECT\_JUNIT\_REPORTS\_ORIGINAL & 31 & 516\; (44.5 \%) \\
    BUILD\_PROJECT\_INSTRUMENTED & 1 & 515\; (44.4 \%) \\
    EXECUTE\_TESTS\_INITIAL & 130 & 385\; (33.2 \%) \\
    COLLECT\_JACOCO\_DATA\_INITIAL & 41 & 344\; (29.7 \%) \\
    COLLECT\_PIT\_DATA\_INITIAL & 64 & 280\; (24.1 \%) \\
    COLLECT\_PIT\_DATA\_GENERALIZED & 270 & 10\; (\phantom{0}0.9 \%) \\
    Successfully processed & - & 10\; (\phantom{0}0.9 \%) \\
    \bottomrule
  \end{tabular}
\end{table}
% \begin{table}[H]
  \caption{Causes of processing failures per processing stage.}
  \label{tab:processing-failure-causes}
  \begin{tabularx}{\textwidth}{l X}
    \toprule
    Processing Stage & Causes of Processing Failures \\
    \midrule
    SETUP\_PROJECT & dependency resolution error (329), sources / tests not found (26) \\
    BUILD\_PROJECT\_ORIGINAL & compilation error (171), compilation outputs not found (18) \\
    BUILD\_SPOON\_MODEL & Spoon execution error (8) \\
    EXECUTE\_TESTS\_ORIGINAL & JUnit execution error (13), timeout exceeded (48) \\
    COLLECT\_JUNIT\_REPORTS\_ORIGINAL & JUnit outputs not found (31) \\
    BUILD\_PROJECT\_INSTRUMENTED & compilation error (1) \\
    EXECUTE\_TESTS\_INITIAL & all tests excluded (129), timeout exceeded (1) \\
    COLLECT\_JACOCO\_DATA\_INITIAL & JaCoCo execution error (1), JaCoCo outputs not found (40) \\
    COLLECT\_PIT\_DATA\_INITIAL & PIT execution error (16), PIT outputs not found (4), all classes excluded (3), failed to map PIT data to a test (1), timeout exceeded (40) \\
    COLLECT\_PIT\_DATA\_GENERALIZED & all generalizations excluded (269), failed to map PIT data to a generalization (1) \\
    \bottomrule
  \end{tabularx}
\end{table}


\paragraph{Processing Pipeline Failures}

While filter rejections only cause
individual tests, assertions, or generalizations
to be excluded from further processing,
processing failures immediately stop
the full processing pipeline
because no further progress can be made
(Tables~\ref{tab:exclusions-breakdown-extended} and \ref{tab:exclusions-filtering-extended}).
The observed failures reveal three actionable improvement categories.
First, reducing exclusions caused by filtering
would recover 401 projects
(129 from all tests excluded,
3 from all classes excluded from mutation testing,
269 from all generalizations excluded).
Second, adding support for more varied project structures
would recover 119 projects
(26 missing source/test directories,
18 missing compilation outputs,
31 missing JUnit reports,
40 missing \ToolJacoco{} data,
4 missing \ToolPit{} data).
Third, increasing timeouts or reducing processing times
would recover 89 projects
(48 during \VariantOriginal{} test execution,
1 during \VariantInitial{} test execution,
40 during \ToolPit{} analysis).
In contrast, dependency resolution and compilation errors
in the original project code (500 projects total)
as well as external tool errors (38 projects) are less actionable,
as they reflect issues outside \ToolTeralizer{}'s control.
\ToolTeralizer{} implementation errors affect only 3 projects,
offering minimal improvement opportunity.

\paragraph{Filter-based Exclusions}

For projects reaching the filtering stages,
Table~\ref{tab:exclusions-filtering-extended}
reveals substantially higher exclusion rates
compared to the primary dataset.
NoAssertions filtering rejects 41.3\% of tests versus 10.3\% in the primary dataset,
reflecting real-world testing practices where assertions often reside in helper methods
or tests only check whether execution completes without crashing.
At the assertion level, MissingValue rejections increase to 57.9\% from 24.7\%,
indicating greater difficulty in resolving tested methods.
ParameterType rejections rise to 49.4\% from 15.4\%,
while ReturnType introduces a new category at 32.6\%
that accounts for void return types as well as unsupported non-void return types.
The primary dataset showed no (non-void) ReturnType exclusions because
target methods were pre-selected to have numeric or boolean return types,
whereas the extended dataset includes all methods regardless of return type.
AssertionType exclusions reach 23.9\% compared to 2.6\%,
demonstrating the larger diversity of assertions used in real-world tests.

% | Category                     | Count  | Where It Appears |
% |------------------------------|--------|------------------|
% | Included                     | 33,390 | Included column  |
% | Filtering (REJECT decisions) | 40,602 | Filtering column |
% | Before filtering             | 7,502  | Failures column  |
% | TestAnalysisTask failures    | 321    | Failures column  |
% | Filtering exceptions         | 19     | Failures column  |
% | Total                        | 81,834 | Total column     |

\begin{table}
  \caption{Exclusion results for \VariantImprovedC{} in the RepoReapers projects.}
  \label{tab:exclusions-breakdown-extended}
  \begin{tabular}{lrrrr}
    \toprule
    & & & \multicolumn{2}{c}{Excluded} \\
    \cmidrule(lr){4-5}
    Level & Total & Included & Filtering & Failures \\
    \midrule
    Test & 81,834 & \phantom{}33,390\; (40.8\%) & \phantom{0,0}40,621\; (49.6\%) & \phantom{}7,823\; (\phantom{0}9.6\%) \\
    Assertion & 122,166 & \phantom{0,}711\; (\phantom{0}0.6\%) & \phantom{}121,073\; (99.1\%) & \phantom{0,0}382\; (\phantom{0}0.3\%) \\
    Generalization & 239 & \phantom{0,}206\; (86.2\%) & \phantom{0,0,}23\; (\phantom{0}9.6\%) & \phantom{0,}10\; (\phantom{0}4.2\%) \\
    \bottomrule
  \end{tabular}
\end{table}
\begin{table}[H]
  \caption{Filtering results of the extended dataset for tests, assertions, and generalizations.}
  \label{tab:exclusions-filtering-extended}
  \begin{tabular}{lllrrrr}
    \toprule
    Variant & Type & Filter Name & Total & \multicolumn{1}{c}{Accept} & \multicolumn{1}{c}{Defer} & \multicolumn{1}{c}{Reject} \\
    \midrule
    \VariantShared{} & Test & NonPassingTest & 74,332 & 65,579\; (88.2\%) & - & 8,753\; (11.8\%) \\
    \VariantShared{} & Test & TestType & 74,332 & 65,052\; (87.5\%) & - & 9,280\; (12.5\%) \\
    \VariantShared{} & Test & NoAssertions & 56,853 & 33,390\; (58.7\%) & - & 23,463\; (41.3\%) \\
    \midrule
    \VariantShared{} & Assertion & AssertionType & 122,166 & 92,996\; (76.1\%) & - & 29,170\; (23.9\%) \\
    \VariantShared{} & Assertion & ExcludedTest & 122,166 & 101,519\; (83.1\%) & - & 20,647\; (16.9\%) \\
    \VariantShared{} & Assertion & MissingValue & 122,166 & 51,430\; (42.1\%) & - & 70,736\; (57.9\%) \\
    \VariantShared{} & Assertion & ParameterType & 122,166 & 5,393\; (\phantom{0}4.4\%) & 56,484\; (46.2\%) & 60,289\; (49.4\%) \\
    \VariantShared{} & Assertion & ReturnType & 122,166 & 11,650\; (\phantom{0}9.5\%) & 70,736\; (57.9\%) & 39,780\; (32.6\%) \\
    \midrule
    \VariantImprovedC{} & Generalization & NonPassingTest & 229 & 206\; (90.0\%) & - & 23\; (10.0\%) \\
    \bottomrule
  \end{tabular}
\end{table}

These filtering patterns suggest targeted improvement strategies.
Enhanced static analysis could reduce the 57.9\% MissingValue exclusions
by better resolving method calls in the presence of complex control flow,
while also reducing the 41.3\% NoAssertions rate by detecting assertions in helper methods.
Improvements to specification extraction could reduce the 32.6\% ReturnType exclusions
by modeling side effects as symbolic outputs,
enabling generalization of void methods that only modify object state.
Methods with non-numeric/non-boolean return types,
as well as the 49.4\% ParameterType and 23.9\% AssertionType exclusions
would all benefit from improved support
for non-numeric/non-boolean types in \ToolSPF{}.
While there already is some support
for strings, arrays, and objects in \ToolSPF{},
the provided information is insufficient
for accurate extraction of input/output specifications
as required for \ToolTeralizer{}'s automated test generalization.
For the 119 projects failing due to project structure detection,
improved heuristics for locating source directories, test directories,
and build outputs across diverse project layouts could improve achievable results.

% [PLACEHOLDER: Analysis of specific parameter types, return types, and assertion patterns
% observed in the extended dataset. Include distribution of non-generalizable types
% (e.g., String: X%, List: Y%, custom objects: Z%) and assertion methods
% (e.g., assertThat: X%, Mockito.verify: Y%, custom assertions: Z%).]

Within the 10 successfully processed projects,
\ToolTeralizer{} creates 229 property-based tests using the \VariantImprovedC{} variant.
The generalization-level filtering rate of 10.0\%
is comparable to the 16.0\% observed in the primary dataset.
This indicates similar failure patterns
from complex and implicit constraints
as in the primary dataset.
However, even though generalization-level filtering
leaves 206 successfully generalized tests,
none of these tests improve mutation detection rates
beyond the results achieved by the original test suites.

% [PLACEHOLDER: Test quality metrics for the 10 successful projects.
% Include original mutation scores, statement coverage, and mutation coverage.
% Likely finding: These projects already have strong test suites (>80\% mutation score),
% leaving limited room for improvement through generalization.
% Alternative explanation: The small number of successful generalizations (229)
% may be insufficient to detect marginal improvements.]

% \paragraph{Summary}

% The extended evaluation reveals that practical applicability faces challenges
% at every stage of the processing pipeline,
% with only 0.9\% of projects successfully completing all stages.
% Failures stem from three categories of barriers:
% (i) internal engineering challenges that \ToolTeralizer{} could address
% (filtering exclusions, project structure detection, and static analysis limitations),
% (ii) external research challenges requiring advances in \ToolSPF{}
% (extended type support beyond numeric and boolean),
% and (iii) external issues beyond our control
% (dependency resolution and compilation errors in original projects).
% The similar generalization-level failure rates (10\% and 16\%)
% and systematic filtering patterns across both evaluations
% suggest that the core generalization approach remains sound.
% Addressing the internal engineering barriers could potentially
% enable processing for 520 additional projects (45\% of the dataset),
% while the external research challenge of expanding type support
% remains crucial for broader applicability.
% The gap between controlled (50\%) and real-world (0.9\%) success rates
% underscores that practical applicability requires addressing both
% the actionable engineering improvements and the fundamental type support limitations.

% Data sources for RQ4 answer:
% - Static analysis failures: 24.7% MissingValue (line 99), 41.3% NoAssertions in extended (line 328)
% - Type support: 15.4% ParameterType primary (line 108), 49.4% extended (line 333)
% - Constraint handling: 14.6-28.4% from Improved vs Naive variants (lines 209-210)
% - Success rates: ~50% from "approx half of assertions" (line 244), 0.9% from extended (line 271)
% - 45% recovery: 520 projects = 401 (filtering) + 119 (structure) out of 1,160 total (lines 291-304, 400)

TODO: Answer to RQ5.
