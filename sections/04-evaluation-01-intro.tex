\newpage{}
\section{Evaluation}
\label{sec:evaluation}

Test generalization strengthens existing test suites by exploring additional inputs
within already-covered execution paths. Our approach automates this transformation from
conventional unit tests to property-based tests, reducing the manual effort
traditionally required to implement such a transformation.
We empirically evaluate the potential and limitations of semantics-based test generalization
through five research questions:

\begin{itemize}
  \item \textbf{RQ1:} How much does test generalization improve mutation detection?
  \item \textbf{RQ2:} How does constraint complexity affect constraint-aware versus random input generation?
  \item \textbf{RQ3:} To which degree does generalization affect the size and runtime of the target test suites?
  \item \textbf{RQ4:} How efficient is test generalization compared to extended test generation?
  \item \textbf{RQ5:} What are the causes of unsuccessful generalization attempts?
\end{itemize}

These questions progress from measuring direct effects to understanding practical constraints.
RQ1 establishes effectiveness through mutation score improvements.
RQ2 explores how constraint complexity affects mutation score improvements achieved by the \VariantNaive{} and \VariantImproved{} generation variants.
RQ3 quantifies effects on test suite size and execution time.
RQ4 examines the runtime costs and efficiency of test generalization compared to extended test generation via \ToolEvoSuite{}
and RQ5 analyzes failure cases to identify current limitations and guide future improvements.

Our experimental framework (Section~\ref{sec:experimental-framework}) establishes the methodology:
mutation testing measures effectiveness, three complementary datasets progressively reveal
capabilities and limitations, and nine test variants isolate different generalization effects.
Results (Sections~\ref{sec:primary-effects-eval}--\ref{sec:filtering-eval-extended})
demonstrate mutation score improvements under favorable conditions,
quantify resource trade-offs, and systematically map barriers to broader applicability.
