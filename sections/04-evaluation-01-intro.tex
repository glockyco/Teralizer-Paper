\section{Evaluation}
\label{sec:evaluation}

Test generalization strengthens the mutation scores of existing test suites
by exploring additional inputs within already-covered execution paths.
Our approach automates the transformation from
conventional unit tests to property-based tests, reducing the manual effort
required to perform such a transformation.
We empirically evaluate the potential and limitations of semantics-based test generalization
through six research questions:

\begin{itemize}
  \item \textbf{RQ1:} How much does test generalization improve the mutation score of existing unit test suites?
  \item \textbf{RQ2:} How does constraint complexity affect constraint-aware versus random input generation?
  \item \textbf{RQ3:} To which degree does generalization affect the size and runtime of the target test suites?
  \item \textbf{RQ4:} How efficient is test generalization compared to test generation?
  \item \textbf{RQ5:} What are the causes of unsuccessful generalization attempts under controlled conditions?
  \item \textbf{RQ6:} What are the causes of unsuccessful generalization attempts under real-world conditions?
\end{itemize}

These questions progress from measuring direct effects to understanding practical constraints.
RQ1 establishes effectiveness through mutation score improvements.
RQ2 explores how constraint complexity affects mutation score improvements of \VariantNaive{} and \VariantImproved{}.
RQ3 quantifies effects on test suite size and execution time.
RQ4 examines the runtime cost and efficiency of test generalization compared to test generation via \ToolEvoSuite{}.
RQ5 and RQ6 analyze failure cases to identify current limitations and guide future improvements.
Section~\ref{sec:experimental-framework} describes our experimental setup and methodology.
Sections~\ref{sec:primary-effects-eval}--\ref{sec:limitations-eval-extended} present results.
All experiments were run on a MacBook Air (M2, 24~GB RAM) with default JVM settings.
All collected data is available in our replication package~\cite{replicationpackage}.
