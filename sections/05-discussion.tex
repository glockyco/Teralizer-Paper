\section{Discussion}
\label{sec:discussion}

% \subsection{Benefits of the Approach}
% \label{sec:benefits-of-the-approach}

% \subsection{Potential for Future Improvements}
% \label{sec:potential-for-improvements}

% \subsubsection{Improving the Mutation Score of Generalized Tests}
% \label{sec:improving-the-mutation-score}

% \subsubsection{Improving the Size of Generalized Tests}
% \label{sec:improving-the-test-size}

% \subsubsection{Improving the Runtime of Test Generalization}
% \label{sec:improving-the-runtime}

% \subsubsection{Reducing the Number of Unsuccessful Generalizations}
% \label{sec:reducing-unsuccessful-generalizations}

% \subsubsection{Using Test Generalization for Test Suite Reduction}
% \label{sec:test-suite-reduction}

% \subsubsection{Dealing with Overfitting}
% \label{sec:overfitting}


\subsection{Key Findings Across Research Questions}
\label{sec:key-findings}

Our evaluation reveals several unexpected patterns that challenge
conventional assumptions about test generalization and property-based testing.

\subsubsection{Mutation Type Determines Optimal Generation Strategy}

The superiority of \VariantNaive{} over \VariantImproved{} on \DatasetsEqBenchEs{} projects,
despite the opposite pattern on \DatasetsCommonsEs{},
reveals that different fault types require different detection strategies.
Math mutations, constituting 59\% of all mutants in our evaluation,
show 3.99\% improvement with random generation versus 3.37\% with constraint-aware generation.
% Data source: detections-per-mutator-data.csv
Conversely, ConditionalsBoundary mutations (11\% of mutants)
achieve 2.55\% improvement with constraint-aware generation versus only 1.21\% with random generation.

This dichotomy reflects the spatial distribution of faults:
Math mutations (e.g., changing \texttt{+} to \texttt{-}) typically create
incorrect behavior across most of the input space,
making them amenable to random sampling.
Boundary mutations concentrate faults at partition edges,
where constraint-aware generation focuses its testing effort.
The implication: effective property-based testing requires
a portfolio of generation strategies matched to expected fault distributions,
not a single sophisticated approach.

\subsubsection{Test Architecture Creates Trade-offs with Generalizability}

The stark difference in test replaceability between \ToolEvoSuite{} and developer tests
stems not from quality but from architectural choices.
\ToolEvoSuite{}'s single-assertion-per-test structure enables complete replacement
when generalization succeeds, achieving near-zero net test count increase (0--1 tests added).
% Data source: tests-per-project-data.csv
Developer tests, averaging multiple assertions per test method,
prevent removal even when individual assertions generalize successfully.

This pattern reveals a tension between test maintainability and analyzability.
Developers consolidate related assertions to reduce duplication and improve readability---sound
software engineering practice that inadvertently blocks automated enhancement.
The finding suggests that test design guidelines should consider
not just immediate quality attributes but also amenability to future automated analysis.

\subsubsection{Validation Dominates Cost, Not Core Generalization}

Runtime analysis reveals that the generalization approach itself is efficient,
consuming less than 10\% of total processing time for specification extraction and transformation.
% Data source: runtime-breakdown-data.csv
The dominant cost (40--78\%) comes from validation through mutation testing,
which is required to identify which generalizations actually improve fault detection.
This distribution suggests that selective generalization of high-value tests
could achieve most benefits at a fraction of the cost.

Notably, combining 1-second \ToolEvoSuite{} generation with \ToolTeralizer{} generalization
can achieve better mutation scores than 60-second \ToolEvoSuite{} generation alone,
suggesting that staged approaches (quick generation followed by selective generalization)
may be more efficient than monolithic extended generation.

\subsubsection{Clear Feasibility Boundaries Emerge from Filtering Analysis}

The filtering cascade reveals that even under favorable conditions,
only about 25\% of assertions successfully generalize
(47.8\% pass assertion-level filters, of which roughly half succeed at generalization).
% Data source: exclusions-summary-data.csv, exclusions-filtering-data.csv
The jump from 1.6\% success in buildable real-world projects
to approximately 50\% in favorable conditions
identifies specific barriers: method resolution failures increase from 24.7\% to 57.9\%,
and type support issues increase from 15.4\% to 49.4\%.

These numbers establish realistic expectations:
test generalization is not a universal solution but rather a targeted technique
most effective for numeric computations with simple test structures.

\subsection{Implications for Testing Research and Practice}
\label{sec:implications}

\subsubsection{Rethinking Property-Based Testing Strategies}

The failure of sophisticated constraint-aware generation to universally outperform random generation
challenges the property-based testing community's focus on "smart" generation.
Our results suggest that generation strategy effectiveness depends on mutation characteristics:
random generation for pervasive faults (Math mutations),
constraint-aware for boundary faults (ConditionalsBoundary),
and potentially other strategies for different fault patterns.

Future property-based testing frameworks could adapt their generation strategies
based on the code under test, using static analysis to predict likely mutation types
and selecting appropriate generation strategies accordingly.

\subsubsection{Test Design for Future Automation}

While developer tests prioritize maintainability through consolidated assertions,
our results suggest value in designing tests with future automation in mind.
A preprocessing step that automatically refactors multi-assertion tests
into single-assertion equivalents could enable generalization
while preserving the original consolidated version for maintenance.
This dual representation---consolidated for humans, decomposed for tools---could
reconcile the competing demands of maintainability and analyzability.

\subsubsection{Test Suite Reduction Through Partition Analysis}

Our observation that multiple tests often cover the same execution partition,
particularly in \ToolEvoSuite{}-generated suites,
suggests an opportunity for test suite reduction.
Rather than generalizing every test, we could identify partition equivalence classes,
retain one representative test per partition, and generalize only these representatives.
This approach could yield smaller, more efficient test suites
with equivalent or better fault detection.

\subsection{Opportunities for Practical Improvement}
\label{sec:opportunities}

Several engineering improvements could substantially enhance practical deployment:

\paragraph{Runtime Optimization}
The commercial PIT accelerator plugin claims 3$\times$ speedup for mutation testing,
which would directly reduce our dominant cost.
Additionally, parallelization of both \ToolTeralizer{} tasks and PIT execution
(currently single-threaded by default) could provide further improvements.
The upcoming jqwik 2.0's planned parallelization support would also reduce
property-based test execution time.

% primary runtime costs are from external tools (i.e., jqwik + PIT)
%
% mutation testing runtimes could likely be improved
% through the use of the pitest accelerator plugin (\url{https://docs.arcmutate.com/docs/accelerator.html}),
% thus, further improving results of Teralizer relative to EvoSuite
% (website claims "analysis time for Commons Lang is reduced from over 55 minutes to under 18");
% however, the plugin is not available without a license
%
% also, parallelization would be possible rather easily for many Teralizer tasks
% (and also for PIT, which only runs single-threaded by default)
% (supposedly, jqwik will get parallelization support in major version 2, source (2024-05-14): \url{https://github.com/jqwik-team/jqwik/issues/45#issuecomment-2109949764})
%
% for practical application scenarios, could add configuration to target only specific tests, rather than the full test suite

\paragraph{Enhanced Method Resolution}
The 24.7\% method resolution failure rate in favorable conditions
represents low-hanging fruit for engineering improvement.
More sophisticated static analysis handling inheritance hierarchies,
lambda expressions, and method references could substantially reduce
these failures without requiring fundamental advances.

\paragraph{Selective Targeting}
Given that validation dominates cost,
selective application to high-value tests (e.g., those covering critical components
or showing low initial mutation scores) could achieve
most benefits at fraction of the computational cost.

\subsection{Fundamental Limitations}
\label{sec:limitations}

\subsubsection{Type Support Remains a Research Challenge}

The restriction to numeric and boolean types stems from fundamental limitations
in symbolic execution, not implementation choices.
All major symbolic execution engines face similar constraints,
and alternative specification extraction approaches
produce different specification types unsuitable for test input generation.
Supporting strings would require string constraint solvers,
while collections and objects need theories for data structure properties---active
research areas without mature solutions.

\subsubsection{Diminishing Returns with Test Maturity}

The inverse relationship between initial test strength and improvement potential
(3--4pp for 48\% baseline, 0.07pp for 80\% baseline)
% Data source: mutation-detection-figure-data.csv
reflects a fundamental characteristic:
mature test suites have already caught the "easy" mutants that generalization typically detects.
The remaining undetected mutants often involve subtle semantic issues
or non-numeric computations outside our approach's scope.

\subsection{Threats to Validity}
\label{sec:threats-to-validity}

\subsubsection{Construct Validity}
\label{sec:construct-validity}

We use mutation score as our primary effectiveness measure,
which may not perfectly correlate with real fault detection.
While the \ToolPit{} DEFAULTS operators are empirically validated,
they may not represent all fault types in practice.
Our filtering of parameterized tests and certain assertion types
could bias results toward simpler test patterns.
The decision to exclude tests that fail after removing \ToolEvoSuite{}'s isolation features
may underestimate the challenges of real-world application.

\subsubsection{Internal Validity}
\label{sec:internal-validity}

We control for randomness using fixed seeds and equal \tries{} parameters across variants.
However, the effectiveness of different generation strategies
inherently depends on program characteristics,
introducing variability beyond our control.
Runtime measurements were collected on a single machine configuration
and may vary with different hardware or JVM settings.
The comparison between \ToolEvoSuite{} and developer tests
involves different projects (EqBench vs Commons),
potentially confounding test quality analysis with project characteristics.

\subsubsection{External Validity}
\label{sec:external-validity}

Our evaluation is limited to Java 5--8 projects using JUnit 4/5 and Maven/Gradle builds.
The \DatasetEqBench{} benchmark's focus on mathematical programs
may not represent typical software projects.
The 0.9\% success rate on RepoReapers projects
reflects current implementation limitations
and may improve with engineering enhancements.
Modern Java features, alternative testing frameworks,
and non-numeric domains remain unexplored.
