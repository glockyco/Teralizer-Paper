\section{Discussion}
\label{sec:discussion}

\subsection{Benefits of the Approach}
\label{sec:benefits-of-the-approach}

\subsection{Potential for Future Improvements}
\label{sec:potential-for-improvements}

\subsubsection{Improving the Mutation Score of Generalized Tests}
\label{sec:improving-the-mutation-score}

\subsubsection{Improving the Size of Generalized Tests}
\label{sec:improving-the-test-size}

\subsubsection{Improving the Runtime of Test Generalization}
\label{sec:improving-the-runtime}

primary runtime costs are from external tools (i.e., jqwik + PIT)

mutation testing runtimes could likely be improved
through the use of the pitest accelerator plugin (\url{https://docs.arcmutate.com/docs/accelerator.html}),
thus, further improving results of Teralizer relative to EvoSuite
(website claims "analysis time for Commons Lang is reduced from over 55 minutes to under 18");
however, the plugin is not available without a license

also, parallelization would be possible rather easily for many Teralizer tasks
(and also for PIT, which only runs single-threaded by default)
(supposedly, jqwik will get parallelization support in major version 2, source (2024-05-14): \url{https://github.com/jqwik-team/jqwik/issues/45#issuecomment-2109949764})

for practical application scenarios, could add configuration to target only specific tests, rather than the full test suite

\subsubsection{Reducing the Number of Unsuccessful Generalizations}
\label{sec:reducing-unsuccessful-generalizations}

\subsubsection{Using Test Generalization for Test Suite Reduction}
\label{sec:test-suite-reduction}

\subsubsection{Dealing with Overfitting}
\label{sec:overfitting}

We do not generate assertions, so overfitting is less pronounced than if we did.
However, we still rely on the current implementation as the source of truth,
encoding the exact behavior of it in the generalized tests (at least as far as
the behavior is tested by assertions in the original, non-generalized test suite).

\subsection{Threats to Validity}
\label{sec:threats-to-validity}

\subsubsection{Construct Validity}
\label{sec:construct-validity}

\subsubsection{Internal Validity}
\label{sec:internal-validity}

\subsubsection{External Validity}
\label{sec:external-validity}
