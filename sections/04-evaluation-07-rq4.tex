\subsection{RQ4: Causes of Unsuccessful Generalizations}
\label{sec:limitations-eval}

[TODO: Intro]

% --

\subsubsection{Unsuccessful Generalizations in the Primary Evaluation Dataset}
\label{sec:filtering-eval}

\begin{table}[H]
  \caption{Included and excluded counts by variant and level.}
  \label{tab:exclusions-summary}
  \begin{tabular}{llrrr}
    \toprule
    Variant & Type & Total & \multicolumn{1}{c}{Included} & \multicolumn{1}{c}{Excluded} \\
    \midrule
    \VariantShared{} & Test & 23,246 & 19,306\; (83.1\%) & 3,940\; (16.9\%) \\
    \midrule
    \VariantShared{} & Assertion & 28,923 & 13,836\; (47.8\%) & 15,087\; (52.2\%) \\
    \midrule
    \VariantBaseline{} & Generalization & 13,836 & 13,814\; (99.8\%) & 22\; (\phantom{0}0.2\%) \\
    \VariantNaiveA{} & Generalization & 13,836 & 10,743\; (77.6\%) & 3,093\; (22.4\%) \\
    \VariantNaiveB{} & Generalization & 13,836 & 9,964\; (72.0\%) & 3,872\; (28.0\%) \\
    \VariantNaiveC{} & Generalization & 13,836 & 9,881\; (71.4\%) & 3,955\; (28.6\%) \\
    \VariantImprovedA{} & Generalization & 13,836 & 11,788\; (85.2\%) & 2,048\; (14.8\%) \\
    \VariantImprovedB{} & Generalization & 13,836 & 11,660\; (84.3\%) & 2,176\; (15.7\%) \\
    \VariantImprovedC{} & Generalization & 13,836 & 11,597\; (83.8\%) & 2,239\; (16.2\%) \\
    \bottomrule
  \end{tabular}
\end{table}
\begin{table}
  \caption{Filtering results for tests and assertions in the \DatasetsCommons{} and \DatasetsEqBenchEs{} projects.}
  \label{tab:exclusions-filtering}
  \begin{tabular}{llrrrr}
    \toprule
    Level & Filter Name & Total & \multicolumn{1}{c}{Accept} & \multicolumn{1}{c}{Defer} & \multicolumn{1}{c}{Reject} \\
    \midrule
    Test & NonPassingTest & 23,246 & 21,719\; (93.4\%) & - & 1,527\; (\phantom{0}6.6\%) \\
    Test & TestType & 23,246 & 23,066\; (99.2\%) & - & 180\; (\phantom{0}0.8\%) \\
    Test & NoAssertions & 21,532 & 19,306\; (89.7\%) & - & 2,226\; (10.3\%) \\
    \midrule
    Assertion & AssertionType & 28,923 & 28,180\; (97.4\%) & - & 743\; (\phantom{0}2.6\%) \\
    Assertion & ExcludedTest & 28,923 & 27,326\; (94.5\%) & - & 1,597\; (\phantom{0}5.5\%) \\
    Assertion & MissingValue & 28,923 & 21,766\; (75.3\%) & - & 7,157\; (24.7\%) \\
    Assertion & ParameterType & 28,923 & 17,835\; (61.7\%) & 6,630\; (22.9\%) & 4,458\; (15.4\%) \\
    Assertion & VoidReturnType & 28,923 & 21,763\; (75.2\%) & 7,157\; (24.7\%) & 3\; (\phantom{0}0.0\%) \\
    \bottomrule
  \end{tabular}
\end{table}
%\begin{table}[H]
  \caption{Test execution failure analysis by project.}
  \label{tab:exclusions-test-fails-by-project}
  \begin{tabular}{lrrrr}
    \toprule
    Project & Total & \multicolumn{1}{c}{No Error} & \multicolumn{1}{c}{TooManyFilterMisses} & \multicolumn{1}{c}{Inaccurate Specification} \\
    \midrule
    \DatasetEqBenchA{} & 57137 & 54998\; (96.3\%) & 1906\; (\phantom{0}3.3\%) & 233\; (\phantom{0}0.4\%) \\
    \DatasetEqBenchB{} & 59033 & 56643\; (96.0\%) & 2148\; (\phantom{0}3.6\%) & 242\; (\phantom{0}0.4\%) \\
    \DatasetEqBenchC{} & 60231 & 57680\; (95.8\%) & 2279\; (\phantom{0}3.8\%) & 272\; (\phantom{0}0.5\%) \\
    \midrule
    \DatasetCommonsA{} & 27598 & 25119\; (91.0\%) & 1376\; (\phantom{0}5.0\%) & 1103\; (\phantom{0}4.0\%) \\
    \DatasetCommonsB{} & 30968 & 27892\; (90.1\%) & 1742\; (\phantom{0}5.6\%) & 1334\; (\phantom{0}4.3\%) \\
    \DatasetCommonsC{} & 31572 & 28160\; (89.2\%) & 1895\; (\phantom{0}6.0\%) & 1517\; (\phantom{0}4.8\%) \\
    \midrule
    \DatasetCommonsDev{} & 18367 & 17107\; (93.1\%) & 507\; (\phantom{0}2.8\%) & 753\; (\phantom{0}4.1\%) \\
    \bottomrule
  \end{tabular}
\end{table} % TODO: Cover in text only (?).
%\begin{table}[H]
  \caption{Test execution failure analysis by variant.}
  \label{tab:exclusions-test-fails-by-variant}
  \begin{tabular}{lrrrr}
    \toprule
    Variant & Total & \multicolumn{1}{c}{No Error} & \multicolumn{1}{c}{TooManyFilterMisses} & \multicolumn{1}{c}{Inaccurate Specification} \\
    \midrule
    \VariantOriginal{} & 39,014 & 38,882\; (99.7\%) & 0\; (\phantom{0}0.0\%) & 132\; (\phantom{0}0.3\%) \\
    \VariantInitial{} & 19,306 & 19,306\; (100.0\%) & 0\; (\phantom{0}0.0\%) & 0\; (\phantom{0}0.0\%) \\
    \VariantBaseline{} & 33,142 & 33,120\; (99.9\%) & 0\; (\phantom{0}0.0\%) & 22\; (\phantom{0}0.1\%) \\
    \midrule
    \VariantNaiveA{} & 33,110 & 30,049\; (90.8\%) & 2,233\; (\phantom{0}6.7\%) & 828\; (\phantom{0}2.5\%) \\
    \VariantNaiveB{} & 33,110 & 29,270\; (88.4\%) & 2,938\; (\phantom{0}8.9\%) & 902\; (\phantom{0}2.7\%) \\
    \VariantNaiveC{} & 33,110 & 29,187\; (88.2\%) & 3,005\; (\phantom{0}9.1\%) & 918\; (\phantom{0}2.8\%) \\
    \midrule
    \VariantImprovedA{} & 33,110 & 31,094\; (93.9\%) & 1,189\; (\phantom{0}3.6\%) & 827\; (\phantom{0}2.5\%) \\
    \VariantImprovedB{} & 33,110 & 30,966\; (93.5\%) & 1,223\; (\phantom{0}3.7\%) & 921\; (\phantom{0}2.8\%) \\
    \VariantImprovedC{} & 33,110 & 30,903\; (93.3\%) & 1,265\; (\phantom{0}3.8\%) & 942\; (\phantom{0}2.8\%) \\
    \bottomrule
  \end{tabular}
\end{table} % TODO: Cover in text only (?).
%\begin{table}[H]
  \caption{Number of SPF execution failures by error type.}
  \label{tab:exclusions-spf}
  \begin{tabular}{lrr}
    \toprule
    Error Type & Total & Percent \\
    \midrule
    SPF exception & 1540 & 51.42 \\
    PC size limit exceeded & 790 & 26.38 \\
    Depth limit exceeded & 524 & 17.50 \\
    Teralizer exception & 97 & 3.24 \\
    Execution timeout & 28 & 0.93 \\
    OutOfMemoryError & 16 & 0.53 \\
    \bottomrule
  \end{tabular}
\end{table} % TODO: Cover in text only (?).

\ToolTeralizer{} determines generalization feasibility
through a three-level filtering cascade.
Table~\ref{tab:exclusions-filtering} quantifies this filtering cascade:
test-level filtering excludes 16.9\% of tests,
assertion-level filtering excludes 52.2\% of assertions within remaining tests,
and generalization-level filtering excludes 0.2--28.4\% of generated tests depending on the variant
(Table~\ref{tab:exclusions-summary}).
The vast majority of exclusions stem from systematic filtering that identifies
cases the current \ToolTeralizer{} implementation cannot handle,
rather than unexpected failures during processing.
Throughout the remainder of this section,
we provide detailed descriptions of the different failure causes
responsible for these exclusions.

\paragraph{Test-Level Filtering}

Test-level filtering excludes 3,940 tests (16.9\%),
with 3,933 excluded through filtering and 7 from \ToolTeralizer{} exceptions
during test analysis and filtering phases.

NonPassingTest rejections account for 1,527 exclusions (6.6\% of total tests).
The mechanism behind these failures reveals an interesting cascade effect.
132 of the 1,527 exclusions are caused by test that fail
after disabling \ToolEvoSuite{}'s isolation and reproducibility features.
These features had to be disabled to prevent \ToolPit{} crashes during mutation testing with \ToolJqwik{}.
However, their removal causes tests relying on system time or specific environmental conditions to fail.
Because \ToolPit{} only supports class-level exclusion,
1,395 additional passing tests in the same classes must also be excluded,
amplifying the impact of individual test failures by over 10$\times$.

Another 180 tests (0.8\% of total tests) are excluded due to TestType rejections.
All of these exclusions are caused by ParameterizedTests in the \DatasetCommonsDev{} project.
\ToolTeralizer{} currently supports only standard @Test annotations,
excluding parameterized, repeated, and other specialized test types (Section~\ref{sec:test-analysis}).

The NoAssertions filter applies after the two preceding filters,
operating on 21,532 remaining tests from the initial 23,246.
From this subset of tests, 10.3\% (2,226 tests) are excluded by the filter
because they do not contain any assertions in the main test method.
In \ToolEvoSuite{}-generated test suites, all NoAssertions exclusions
are genuinely assertion-free tests the simply pass if no exception occurs.
In contrast, 69 of 80 NoAssertions exclusions (86.3\%) in the developer-written
\DatasetCommonsDev{} test suite are false positves: these tests
contain assertions in helper methods called from the main test method.
However, \ToolTeralizer{}'s current static analysis
only examines the top-level test method,
which causes it to miss these delegated assertion calls.

\paragraph{Assertion-Level Filtering}

Assertion-level filtering examines 28,923 assertions within passing tests,
excluding 15,087 (52.2\%) that cannot be generalized.
Of these exclusions, 12,092 stem from filtering decisions
while 2,995 result from unsuccessful specification extraction
during \ToolSPF{} execution.

The MissingValue filter excludes 24.7\% of assertions.
A rejection occurs when \ToolTeralizer{}'s static analysis
cannot identify which method call represents the tested method
or when the declaration of the tested method cannot be resolved by Spoon.
Missing values also cause ParameterType and VoidReturnType filters
to report a filtering result of \textit{defer},
which indicates that these filters do not have enough information
to make a definitive \textit{accept} or \textit{reject} decision.

The ParameterType filter rejects 15.4\% of assertions
when none of the tested method's parameters
have generalizable types (numeric or boolean).
Deferral numbers are lower than rejections of MissingValue filters
because we can sometimes infer parameter types from the call site
of the tested method if the full method declaration cannot be resolved.

The VoidReturnType filter rejects 3 assertions
for tested methods with void return types.
The currently implementation of \ToolTeralizer{}
does not support such methods because \ToolSPF{}
does not report any symbolic outputs for them.

Another 5.5\% of assertions (1,597) are rejected by the ExcludedTest filter
because they belong to tests already excluded at the test level.
This cascading effect ensures consistency between filtering stages.

Finally, 2.6\% of assertions (743) are rejected by the AssertionType filter
because they are assertions for which \ToolTeralizer{}
does not currently offer generalization support.
Specifically, excluded assertions in the dataset are
assertArrayEquals (207 instances),
assertSame (142),
assertNull (124),
assertNotNull (86),
assertNotSame (79),
assertNotEquals (54),
fail (33),
and assertInstanceOf (18).
These assertions primarily consist of checks for reference equality,
null values, or array comparisons, i.e., involve data types
that are not supported by \ToolTeralizer{}'s current specification extraction.

\paragraph{Specification Extraction Failures}

As mentioned above, 2,995 assertion-level exclusions are due to
unsuccessful specification extraction attempts during \ToolSPF{} execution.
Causes for this include \ToolSPF{} errors, \ToolTeralizer{} errors,
and exceeded analysis limits (timeout, specification size limit, depth limit, etc.).

SPF exceptions constitute 51.4\% of extraction failures (1,540 cases).
They occur primarily because of missing models for native methods in the current implementation \ToolSPF{}
and to a lesser extent due to implementation bugs in \ToolSPF{}/\ToolJPF{}.

Exceeded analysis limits account for another 45.3\% of failed extractions (1358 cases).
For example, 790 (26.4\%) are interrupted
after exceeding the configured maximum specification size,
and 524 (17.5\%) additional cases exceed the configured maximum depth limit.
Both of these aim to avoid timeouts and memory exhaustion
in the presence of complex control flows or complex constraints.
The remaining failures are due to timeouts (0.9\%, 28 cases) and out-of-memory errors (0.5\%, 16 cases)
which evade preemptive detection via the two preceding measures.

Exceptions due to known bugs
in \ToolTeralizer{}'s current specification extraction implementation
represent the remaining 3.2\% of failures (97 cases).

\paragraph{Generalization-Level Filtering}

After generating property-based tests, generalization validation
identifies generalized tests that fail execution (Table~\ref{tab:exclusions-filtering}).
Low failure rates for \VariantBaseline{} confirm that
transformation from JUnit tests to jqwik tests is reliable.
However, failure rates of up to 28.4\% for \VariantNaiveC{}
suggest that actual generalization of inputs and outputs presents more challenges.

\VariantBaseline{} generalized tests fail in 0.2\% of cases (22 out of 13,836 tests).
This low failure rate indicates
that the transformation to property-based \ToolJqwik{} tests
is largely successful in producing compilable code that matches the behavior of the original JUnit tests.
The few failures stem from inaccurate specifications caused by
implicit preconditions, assertions in loops,
assertions in transitively called methods,
or tested methods called within loops.
Neither of these patterns is properly accounted for
by \ToolTeralizer{}'s current specification extraction.

\VariantNaive{} and \VariantImproved{} variants
process 32 fewer tests than \VariantBaseline{}
to avoid Java's "code too large" compilation error.
This error occurs when a method's bytecode exceeds 64KB,
Java's hard limit for method size.
Generated property-based tests with large input/output specifications
can approach this limit when specifications contain
numerous complex constraints,
necessitating preemptive exclusion of the largest specifications.

\VariantNaive{} variants show substantially higher failure rates:
22.2\% for \VariantNaiveA{},
27.8\% for \VariantNaiveB{},
and 28.4\% for \VariantNaiveC{}.
The primary failure cause is TooManyFilterMissesExceptions,
with specific counts of 2,233 for \VariantNaiveA{},
2,938 for \VariantNaiveB{},
and 3,005 for \VariantNaiveC{}.
These exceptions occur when \ToolJqwik{} cannot generate
inputs satisfying path constraints within its retry limit.
The increasing failure count with higher \tries{} values
demonstrates the challenge of generating more diverse valid inputs
when using naive random generation with complex constraints.

\VariantImproved{} variants reduce overall failure rates to 14.6--16.0\%
through constraint-aware input value generation (Section~\ref{sec:boundary-detection-effectiveness}).
Specifically, TooManyFilterMissesExceptions decrease to
1,189 for \VariantImprovedA{},
1,223 for \VariantImprovedB{},
and 1,265 for \VariantImprovedC{}, i.e.,
approximately half the counts seen in naive variants.
However, failures from inaccurate specifications remain consistent
across both variant types at approximately 2.5--2.8\% of tests.

The persistence of specification-related failures stems from
limitations in \ToolTeralizer{}'s current specification extraction approach:
implicit preconditions not visible in the code
(e.g., "inputs do not cause overflows",
"inputs do not cause division by zero",
"array indices remain within bounds"),
assertions within loops that execute varying numbers of times,
assertions in methods called transitively from the test,
and tested methods invoked within loops where iteration counts depend on inputs.
These patterns produce incorrect expected values or
invalid test configurations when different inputs are tested.

Project-specific patterns show how test characteristics affect generalization.
\DatasetsCommons{} projects exhibit 4.99--6.00\% TooManyFilterMissesException rates
and 4.00--4.31\% from inaccurate specifications,
while \DatasetsEqBenchEs{} projects show 3.34--3.78\% and 0.41--0.45\% respectively.
The higher TooManyFilterMissesException rates in \DatasetsCommons{} projects
result from their more complex constraints (Section~\ref{sec:boundary-detection-effectiveness}).
The 10$\times$ higher rate of inaccurate specification failures
in \DatasetsCommons{} projects reflects fundamental differences in test construction:
\ToolEvoSuite{}-generated tests avoid loops and never invoke assertions through helper methods,
while developer-written tests commonly use both patterns.

\paragraph{Summary}

The three-level filtering cascade reveals that even under favorable conditions,
i.e., primarily \ToolEvoSuite{}-generated test suites
that avoid complex patterns like helper methods and loops,
\ToolTeralizer{} excludes significant portions at each stage:
16.9\% at test level,
52.2\% at assertion level,
and 14.6--28.4\% at generalization level
for \VariantNaive{} and \VariantImproved{} generalization variants.
Each level presents distinct challenges:
support for non-@Test test and no-assertion tests at test level,
specification extraction and type support at assertion level,
and input generation at generalization level.
Constraint-aware generation (\VariantImproved{}) reduces generalization-level failures by approximately 50\%
compared to naive approaches (14.6--16.0\% vs. 22.2--28.4\%),
while project-specific patterns show developer-written tests pose greater challenges than synthetic tests.
Detailed insights and implications are discussed following the extended dataset analysis
in Section~\ref{sec:filtering-eval-extended}.

% --

[TODO: Answer to RQ4]
