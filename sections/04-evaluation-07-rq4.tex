\subsection{RQ4 (Part 1 of 2): Causes of Unsuccessful Generalizations in the Evaluation Dataset}
\label{sec:filtering-eval}

In this section, we describe causes of unsuccessful generalizations in the main
evaluation dataset (commons-utils + evosuite-variants, eqbench + evosuite variants).
To get more generalizable results that enable better informed decisions about future
research directions, we also evaluated generalization success vs. failure in <number>
additional open source projects beyond the main evaluation dataset. Results of the
extended evaluation are discussed in Section~\ref{sec:filtering-eval-extended}.

overall test / assertion / generalization exclusions

\begin{table}[H]
  \caption{Included and excluded counts by variant and level.}
  \label{tab:exclusions-summary}
  \begin{tabular}{llrrr}
    \toprule
    Variant & Type & Total & \multicolumn{1}{c}{Included} & \multicolumn{1}{c}{Excluded} \\
    \midrule
    \VariantShared{} & Test & 23,246 & 19,306\; (83.1\%) & 3,940\; (16.9\%) \\
    \midrule
    \VariantShared{} & Assertion & 28,923 & 13,836\; (47.8\%) & 15,087\; (52.2\%) \\
    \midrule
    \VariantBaseline{} & Generalization & 13,836 & 13,814\; (99.8\%) & 22\; (\phantom{0}0.2\%) \\
    \VariantNaiveA{} & Generalization & 13,836 & 10,743\; (77.6\%) & 3,093\; (22.4\%) \\
    \VariantNaiveB{} & Generalization & 13,836 & 9,964\; (72.0\%) & 3,872\; (28.0\%) \\
    \VariantNaiveC{} & Generalization & 13,836 & 9,881\; (71.4\%) & 3,955\; (28.6\%) \\
    \VariantImprovedA{} & Generalization & 13,836 & 11,788\; (85.2\%) & 2,048\; (14.8\%) \\
    \VariantImprovedB{} & Generalization & 13,836 & 11,660\; (84.3\%) & 2,176\; (15.7\%) \\
    \VariantImprovedC{} & Generalization & 13,836 & 11,597\; (83.8\%) & 2,239\; (16.2\%) \\
    \bottomrule
  \end{tabular}
\end{table}

filtering-based test / assertion / generalization exclusions

\begin{table}
  \caption{Filtering results for tests and assertions in the \DatasetsCommons{} and \DatasetsEqBenchEs{} projects.}
  \label{tab:exclusions-filtering}
  \begin{tabular}{llrrrr}
    \toprule
    Level & Filter Name & Total & \multicolumn{1}{c}{Accept} & \multicolumn{1}{c}{Defer} & \multicolumn{1}{c}{Reject} \\
    \midrule
    Test & NonPassingTest & 23,246 & 21,719\; (93.4\%) & - & 1,527\; (\phantom{0}6.6\%) \\
    Test & TestType & 23,246 & 23,066\; (99.2\%) & - & 180\; (\phantom{0}0.8\%) \\
    Test & NoAssertions & 21,532 & 19,306\; (89.7\%) & - & 2,226\; (10.3\%) \\
    \midrule
    Assertion & AssertionType & 28,923 & 28,180\; (97.4\%) & - & 743\; (\phantom{0}2.6\%) \\
    Assertion & ExcludedTest & 28,923 & 27,326\; (94.5\%) & - & 1,597\; (\phantom{0}5.5\%) \\
    Assertion & MissingValue & 28,923 & 21,766\; (75.3\%) & - & 7,157\; (24.7\%) \\
    Assertion & ParameterType & 28,923 & 17,835\; (61.7\%) & 6,630\; (22.9\%) & 4,458\; (15.4\%) \\
    Assertion & VoidReturnType & 28,923 & 21,763\; (75.2\%) & 7,157\; (24.7\%) & 3\; (\phantom{0}0.0\%) \\
    \bottomrule
  \end{tabular}
\end{table}

failing tests

\begin{table}[H]
  \caption{Number of test execution failures by exception type and (generalization) variant.}
  \label{tab:exclusions-test-fails}
  \begin{tabular}{lrrrrrrrr}
    \toprule
    Variant & ORIGINAL & BASELINE & \multicolumn{3}{c}{NAIVE} & \multicolumn{3}{c}{IMPROVED} \\
    \cmidrule(lr){4-6}
    \cmidrule(lr){7-9}
    Tries & - & - & 10 & 50 & 200 & 10 & 50 & 200 \\
    \midrule
    ArithmeticException & 0 & 0 & 99 & 99 & 99 & 57 & 58 & 58 \\
    AssertionFailedError & 132 & 22 & 729 & 803 & 819 & 752 & 845 & 866 \\
    NumberFormatException & 0 & 0 & 0 & 0 & 0 & 18 & 18 & 18 \\
    TooManyFilterMissesException & 0 & 0 & 2233 & 2938 & 3005 & 1189 & 1223 & 1265 \\
    \bottomrule
  \end{tabular}
\end{table}


@TODO: Remove Table~\ref{tab:exclusions-spf}. Describe SPF execution failures in text only.

\begin{table}[H]
  \caption{Number of SPF execution failures by error type.}
  \label{tab:exclusions-spf}
  \begin{tabular}{lrr}
    \toprule
    Error Type & Total & Percent \\
    \midrule
    SPF exception & 1540 & 51.42 \\
    PC size limit exceeded & 790 & 26.38 \\
    Depth limit exceeded & 524 & 17.50 \\
    Teralizer exception & 97 & 3.24 \\
    Execution timeout & 28 & 0.93 \\
    OutOfMemoryError & 16 & 0.53 \\
    \bottomrule
  \end{tabular}
\end{table}

\subsection{RQ4 (Part 2 of 2): Causes of Unsuccessful Generalizations in Other Open Source Projects}
\label{sec:filtering-eval-extended}

This evaluation only uses the IMPROVED$_{200}$
generalization variant, but results also apply to all other variants.

\begin{table}[H]
  \caption{Number of processing failures and remaining projects per processing stage.}
  \label{tab:processing-failures-per-stage}
  \begin{tabular}{l r r}
    \toprule
    Processing Stage & Failures & Remaining Projects \\
    \midrule
    Total projects & - & 1160\; (\phantom{.}100 \%) \\
    SETUP\_PROJECT & 355 & 805\; (69.4 \%) \\
    BUILD\_PROJECT\_ORIGINAL & 189 & 616\; (53.1 \%) \\
    BUILD\_SPOON\_MODEL & 8 & 608\; (52.4 \%) \\
    EXECUTE\_TESTS\_ORIGINAL & 61 & 547\; (47.2 \%) \\
    COLLECT\_JUNIT\_REPORTS\_ORIGINAL & 31 & 516\; (44.5 \%) \\
    BUILD\_PROJECT\_INSTRUMENTED & 1 & 515\; (44.4 \%) \\
    EXECUTE\_TESTS\_INITIAL & 130 & 385\; (33.2 \%) \\
    COLLECT\_JACOCO\_DATA\_INITIAL & 41 & 344\; (29.7 \%) \\
    COLLECT\_PIT\_DATA\_INITIAL & 64 & 280\; (24.1 \%) \\
    COLLECT\_PIT\_DATA\_GENERALIZED & 270 & 10\; (\phantom{0}0.9 \%) \\
    Successfully processed & - & 10\; (\phantom{0}0.9 \%) \\
    \bottomrule
  \end{tabular}
\end{table}

\begin{table}[H]
  \caption{Causes of processing failures per processing stage.}
  \label{tab:processing-failure-causes}
  \begin{tabularx}{\textwidth}{l X}
    \toprule
    Processing Stage & Causes of Processing Failures \\
    \midrule
    SETUP\_PROJECT & dependency resolution error (329), sources / tests not found (26) \\
    BUILD\_PROJECT\_ORIGINAL & compilation error (171), compilation outputs not found (18) \\
    BUILD\_SPOON\_MODEL & Spoon execution error (8) \\
    EXECUTE\_TESTS\_ORIGINAL & JUnit execution error (13), timeout exceeded (48) \\
    COLLECT\_JUNIT\_REPORTS\_ORIGINAL & JUnit outputs not found (31) \\
    BUILD\_PROJECT\_INSTRUMENTED & compilation error (1) \\
    EXECUTE\_TESTS\_INITIAL & all tests excluded (129), timeout exceeded (1) \\
    COLLECT\_JACOCO\_DATA\_INITIAL & JaCoCo execution error (1), JaCoCo outputs not found (40) \\
    COLLECT\_PIT\_DATA\_INITIAL & PIT execution error (16), PIT outputs not found (4), all classes excluded (3), failed to map PIT data to a test (1), timeout exceeded (40) \\
    COLLECT\_PIT\_DATA\_GENERALIZED & all generalizations excluded (269), failed to map PIT data to a generalization (1) \\
    \bottomrule
  \end{tabularx}
\end{table}


Broadly speaking, there are three main ways to increase the number of projects
that can be successfully processed: (i)~reducing exclusions caused by filtering
(129 + 3 + 269 = 401 projects), (ii)~adding support for more varied project structures
(26 + 18 + 31 + 40 + 4 = 119 projects), and (iii)~increasing timeouts (48 + 1 + 40 = 89 projects).
Depedency resolution and compilation errors in the original project code (329 + 171 = 500 projects)
as well as external tool errors (8 + 13 + 1 + 16 = 38 projects) are less actionable.
Furthermore, Teralizer errors only occur in a small number of cases (1 + 1 + 1 = 3 projects),
so also offer comparatively little opportunity for improvements.

\begin{table}[H]
  \caption{Filtering results of the extended dataset for tests, assertions, and generalizations.}
  \label{tab:exclusions-filtering-extended}
  \begin{tabular}{lllrrrr}
    \toprule
    Variant & Type & Filter Name & Total & \multicolumn{1}{c}{Accept} & \multicolumn{1}{c}{Defer} & \multicolumn{1}{c}{Reject} \\
    \midrule
    \VariantShared{} & Test & NonPassingTest & 74,332 & 65,579\; (88.2\%) & - & 8,753\; (11.8\%) \\
    \VariantShared{} & Test & TestType & 74,332 & 65,052\; (87.5\%) & - & 9,280\; (12.5\%) \\
    \VariantShared{} & Test & NoAssertions & 56,853 & 33,390\; (58.7\%) & - & 23,463\; (41.3\%) \\
    \midrule
    \VariantShared{} & Assertion & AssertionType & 122,166 & 92,996\; (76.1\%) & - & 29,170\; (23.9\%) \\
    \VariantShared{} & Assertion & ExcludedTest & 122,166 & 101,519\; (83.1\%) & - & 20,647\; (16.9\%) \\
    \VariantShared{} & Assertion & MissingValue & 122,166 & 51,430\; (42.1\%) & - & 70,736\; (57.9\%) \\
    \VariantShared{} & Assertion & ParameterType & 122,166 & 5,393\; (\phantom{0}4.4\%) & 56,484\; (46.2\%) & 60,289\; (49.4\%) \\
    \VariantShared{} & Assertion & ReturnType & 122,166 & 11,650\; (\phantom{0}9.5\%) & 70,736\; (57.9\%) & 39,780\; (32.6\%) \\
    \midrule
    \VariantImprovedC{} & Generalization & NonPassingTest & 229 & 206\; (90.0\%) & - & 23\; (10.0\%) \\
    \bottomrule
  \end{tabular}
\end{table}

The most common filtering reasons are unsupported parameter and return types.

@TODO: Better explain these filtering results by prooviding information about:
(i) which types are identified in the tested method signatures, and
(ii) which assertions are used in the tests.

Successful generalizations did not improve mutation scores in any of the successfully
processed projects. However, some generalizations indicate that there weak
preconditions are used in the source / test code of the original implementation.

@TODO: Check which generalizations fail the "NonPassingTest" filter due to (i) bugs in
our generalization implemenation vs. (ii) weak preconditions.

% Performance improvements of Teralizer itself can NOT be used as a substitute for
% (i) because the timeouts occur for tasks that are performed by external tools.
