\subsection{RQ2: Effects on Test Suite Size and Runtime}
\label{sec:ancillary-effects-eval}

In this section, we describe the secondary effects caused by the generalization
in terms of changes in test suite size and test suite runtime.
To describe these changes, we employ the following four metrics: 
\begin{enumerate}
  \item the number of tests in the test suite (Section~\ref{sec:test-suite-test-count}),
  \item the number of lines of code in the test suite (Section~\ref{sec:test-suite-line-count}),
  \item the execution time of the test suite (Section~\ref{sec:test-suite-execution-time}).
\end{enumerate}

We focus primarily on the results
of the \VariantNaiveC{} and \VariantImprovedC{} generalization variants.
The results for the remaining variants follow similar trends
but are omitted for brevity.
The full results for all generalization variants
are available in our replication package~\cite{replicationpackage}.

\subsubsection{Number of Tests in the Test Suite}
\label{sec:test-suite-test-count}

Table~\ref{tab:tests-per-project} shows how many tests are added
by the \VariantNaiveC{} and \VariantImprovedC{} generalization variants,
how many can be removed after generalization,
and how this changes the total number of tests
relative to the \VariantOriginal{} test suites
across the \DatasetsEqBenchEs{} and \DatasetsCommons{} projects.
The number of added tests is between 174--211
(3.5--4.3~\% of \VariantOriginal{} test suite size)
for the \DatasetsEqBenchEs{} projects,
between 60--75 (2.2--2.9\%)
for the \DatasetsCommonsEs{} projects,
and 3 (0.4~\%) for the \DatasetCommonsDev{} project.
\VariantImprovedC{} generally adds a larger number of tests than \VariantNaiveC{} 
because more of the tests that are created  by \VariantNaiveC{}
are excluded due to \texttt{Too\-Many\-Filter\-Misses\-Exception}s
(as described in Section~\ref{sec:naive-generalization}
and evaluated in Section~\ref{sec:filtering-eval}).
Furthermore, the number of added tests is much smaller
than the total number of generalization that are created and evaluated by \ToolTeralizer{}
because only tests that measurably increase the mutation score of the test suite are retained.

Added tests are largely compensated by removed tests
in the \DatasetsEqBenchEs{} and \DatasetsCommonsEs{} projects
which use \ToolEvoSuite{} generated test suites.
As a result, total test suite size only increases by 0--1 test cases
(0--0.04\% of \VariantOriginal{} test suite size)
for these projects.
For the \DatasetCommonsDev{} project,
none of the tests for which generalizations are added can be removed.
This is because an \VariantOriginal{} test can only be removed
if generalized tests are successfully created for all assertions in the test
(as described in Section~\ref{sec:baseline-generalization}).
In the projects with \ToolEvoSuite{} generated tests,
this requirement is generally satisfied because most tests only contain a single assertion.
However, this requirement is much more difficult to satisfy for \DatasetCommonsDev{} 
because the developer written tests often contain multiple assertions.
More generally, increases in the number of tests are
expected to be smaller for projects with more focused tests
that each contain only a small number of assertions
than for less focused ones with a large number of assertions per test.

In future work, we also plan to explore opportunities
for generalization to decrease overall test suite size.
This can be achieved
by replacing multiple \VariantOriginal{} tests
that all cover the same input partition
with a single property-based test that covers the same partition instead.
For further discussion and related work about this, see Section~\ref{sec:test-suite-reduction}.

\begin{table}[H]
\caption{Number of tests before and after generalization, with changes, per project.}
\label{tab:tests-per-project}
\begin{tabular}{llrrrrrr}
\toprule
 & & \multicolumn{6}{c}{Tests} \\
\cmidrule(lr){3-8}
Project & Variant & Before & Added & Removed & After & Delta & Delta \% \\
\midrule
\DatasetEqBenchA{} & \VariantNaiveC{} & 4718 & 177 & 177 & 4718 & -- & -- \\
\DatasetEqBenchB{} & \VariantNaiveC{} & 4875 & 174 & 173 & 4876 & +1 & +0.02 \% \\
\DatasetEqBenchC{} & \VariantNaiveC{} & 4974 & 174 & 174 & 4974 & -- & -- \\
\midrule
\DatasetEqBenchA{} & \VariantImprovedC{} & 4718 & 206 & 206 & 4718 & -- & -- \\
\DatasetEqBenchB{} & \VariantImprovedC{} & 4875 & 211 & 210 & 4876 & +1 & +0.02 \% \\
\DatasetEqBenchC{} & \VariantImprovedC{} & 4974 & 210 & 210 & 4974 & -- & -- \\
\midrule
\DatasetCommonsA{} & \VariantNaiveC{} & 2481 & 60 & 59 & 2482 & +1 & +0.04 \% \\
\DatasetCommonsB{} & \VariantNaiveC{} & 2738 & 63 & 62 & 2739 & +1 & +0.04 \% \\
\DatasetCommonsC{} & \VariantNaiveC{} & 2735 & 60 & 59 & 2736 & +1 & +0.04 \% \\
\midrule
\DatasetCommonsA{} & \VariantImprovedC{} & 2481 & 69 & 68 & 2482 & +1 & +0.04 \% \\
\DatasetCommonsB{} & \VariantImprovedC{} & 2738 & 70 & 69 & 2739 & +1 & +0.04 \% \\
\DatasetCommonsC{} & \VariantImprovedC{} & 2735 & 75 & 74 & 2736 & +1 & +0.04 \% \\
\midrule
\DatasetCommonsDev{} & \VariantNaiveC{} & 725 & 3 & 0 & 728 & +3 & +0.41 \% \\
\midrule
\DatasetCommonsDev{} & \VariantImprovedC{} & 725 & 3 & 0 & 728 & +3 & +0.41 \% \\
\bottomrule
\end{tabular}
\end{table}

\subsubsection{Lines of Code in the Test Suite}
\label{sec:test-suite-line-count}

While generalization has only minor effects
on the number of tests in the evaluated test suites,
the lines of code in the test suites are more strongly affected.
As shown in Table~\ref{tab:lines-per-project},
overall lines of code (LOC) in the test suites increase
by 31.5--58.7~\% for the \DatasetsEqBenchEs{} projects,
by 18.8--29.9~\% for the \DatasetsEqBenchEs{} projects,
and by 4.9--5.3~\% for the \DatasetCommonsDev{} project.

Increases in test suite size are strongly correlated with
the number of tests added to the test suite (see Table~\ref{tab:tests-per-project}),
even if added tests are compensated for by test removals.
For example, \VariantNaive{} adds 177 tests to \DatasetEqBenchA{}
and allows the same number of \VariantOriginal{} tests to be removed.
Despite this, 11780 LOC are added
but only 1127 LOC are removed
(as shown in Table~\ref{tab:lines-per-project}).
This is expected for two reasons:
(i) generalized test classes
contain all of the code that is needed
to set valid input ranges for \ToolPit{}'s input value generators
(as described in Section~\ref{sec:test-generalization}), and
(ii) when creating the classes
that contain the generalized test methods,
\ToolTeralizer{} copies all shared code
(i.e., setup/teardown methods, non-test methods, class fields, ...)
from the \VariantOriginal{} test classes
into each generalized test class
(as described in Section~\ref{sec:test-generalization})
which results in additional LOC increases due to duplicated code.

The effects of (i) are more pronounced
for \VariantImproved{} generalization variants
than for \VariantNaive{} ones
because of the more precise (but also more complex)
identification of input ranges used by \VariantImproved{} variants.
For example, the generalized test suite created by 
\VariantImprovedC{} for \DatasetCommonsDev{}
has 36 additional LOC compared to \VariantNaiveC{}
(9018 vs.\ 8982~LOC, see Table~\ref{tab:lines-per-project})
even though the same three tests are added as a result of generalization
(as shown in Table~\ref{tab:tests-per-project}).
This difference in LOC of individual tests generalized
via \VariantImproved{} vs.\ \VariantNaive{} variants
is generally larger for tested methods
with a larger number of input parameters.
The reason for this is that each input parameter
needs to be separately restricted
to match the constraints that involve that parameter
in the input specification.

Both of the causes mentioned above
could be resolved with additional engineering effort.
For example, generalized test methods created by \ToolTeralizer{}
could be added to the \VariantOriginal{} test classes
instead of creating a new classes for each one, thus resolving cause (ii).
The primary reason we avoid this in the current implementation
is to keep \ToolTeralizer{}'s changes as unintrusive and isolated
from the \VariantOriginal{} test suite (and other generalized tests) as possible
to avoid unintended adverse side-effects on developer written code
as well as unintended interactions between generalized tests.
However, there is no technical reason
that strictly requires generalized test methods
to be added to newly created classes.
Similarly, the code to set input value ranges
that satisfy the extracted input specifications
could be extracted to a library
that abstracts away the corresponding implementation details.
In the test code of the target projects,
only minor changes would remain then
(i.e., modified test annotations, inputs, and expected outputs),
thus resolving cause (i).

\begin{table}[H]
\caption{Number of test lines before and after generalization, with changes, per project.}
\label{tab:lines-per-project}
\begin{tabular}{llrrrrrr}
\toprule
 & & \multicolumn{6}{c}{Lines} \\
\cmidrule(lr){3-8}
Project & Variant & Before & Added & Removed & After & Delta & Delta \% \\
\midrule
\DatasetEqBenchA{} & \VariantNaiveC{} & 30989 & 11780 & 1127 & 41642 & +10653 & +34.4 \% \\
\DatasetEqBenchB{} & \VariantNaiveC{} & 32503 & 11520 & 1061 & 42962 & +10459 & +32.2 \% \\
\DatasetEqBenchC{} & \VariantNaiveC{} & 33510 & 11623 & 1069 & 44064 & +10554 & +31.5 \% \\
\midrule
\DatasetEqBenchA{} & \VariantImprovedC{} & 30989 & 19019 & 1302 & 48706 & +17717 & +57.2 \% \\
\DatasetEqBenchB{} & \VariantImprovedC{} & 32503 & 20353 & 1284 & 51572 & +19069 & +58.7 \% \\
\DatasetEqBenchC{} & \VariantImprovedC{} & 33510 & 20288 & 1285 & 52513 & +19003 & +56.7 \% \\
\midrule
\DatasetCommonsA{} & \VariantNaiveC{} & 16563 & 3733 & 359 & 19937 & +3374 & +20.4 \% \\
\DatasetCommonsB{} & \VariantNaiveC{} & 18124 & 3942 & 379 & 21687 & +3563 & +19.7 \% \\
\DatasetCommonsC{} & \VariantNaiveC{} & 17886 & 3723 & 361 & 21248 & +3362 & +18.8 \% \\
\midrule
\DatasetCommonsA{} & \VariantImprovedC{} & 16563 & 5261 & 413 & 21411 & +4848 & +29.3 \% \\
\DatasetCommonsB{} & \VariantImprovedC{} & 18124 & 5423 & 421 & 23126 & +5002 & +27.6 \% \\
\DatasetCommonsC{} & \VariantImprovedC{} & 17886 & 5801 & 452 & 23235 & +5349 & +29.9 \% \\
\midrule
\DatasetCommonsDev{} & \VariantNaiveC{} & 8561 & 421 & 0 & 8982 & +421 & +4.9 \% \\
\midrule
\DatasetCommonsDev{} & \VariantImprovedC{} & 8561 & 457 & 0 & 9018 & +457 & +5.3 \% \\
\bottomrule
\end{tabular}
\end{table}

\subsubsection{Execution Time of the Test Suite}
\label{sec:test-suite-execution-time}

As shown in Table~\ref{tab:runtime-per-project},
generalization with \VariantNaiveC{} and \VariantImprovedC{}
increases overall test suite runtimes
for all \DatasetsEqBenchEs{} and \DatasetsCommonsEs{} projects.
More specifically, test suite runtimes
show increases of 574.5--1210.0~\% for the \DatasetsEqBenchEs{} projects,
444.1--2651.5~\% for the \DatasetsCommonsEs{} projects,
and 9.4--57.1~\% for the \DatasetCommonsDev{} project.
Runtime increases are primarily affected by the following factors:

\begin{enumerate}
  \item the number of tests added by the generalization, even if they are compensated for by removed tests,
  \item the number of \tries{} used during property-based testing,
  \item the used generalization approach, i.e., \VariantNaive{} vs. \VariantImproved{} generalization,
  \item the complexity of input partition constraints.
\end{enumerate}

In the following paragraphs,
we discuss how each of the above factors
influences overall test suite execution times
and how these factors interact with each other.

\paragraph{Added Tests}

Execution of conventional JUnit tests has a lower runtime cost
than execution of corresponding property-based jqwik tests.
More specifically, individual property-based tests
created with the \VariantBaseline{} generalization variant 
take, on average, 149.56 milliseconds (ms) longer to execute
than the corresponding \VariantOriginal{} tests
(as shown in the left plot of Figure~\ref{fig:test-runtime-differences})
which have a mean execution time of only 3.6~ms. % 0.0035570506476290337
% Exact number = 0.0035570506476290337 via:
% SELECT AVG(runtime)
% FROM junit_test_report jtr
% JOIN generalization g ON g.test_id = jtr.test_id
% WHERE stage = 'COLLECT_JUNIT_REPORTS_ORIGINAL'
% AND g.is_included;
Therefore, overall test suite execution time increases,
on average, by at least 149.56~ms per jqwik test
that is added during the generalization process,
even if no new test inputs are exercised
and the added tests are compensated for by removed JUnit tests.
Since these runtime increases are inherent to the use of jqwik,
they are orthogonal to our specific generalization approach.
That is, any manual or automated transformation of JUnit tests
to jqwik tests incurs the same runtime overhead,
and this overhead can only be reduced
if fewer jqwik tests are created
(e.g., through test suite reduction, as discussed in Section~\ref{sec:test-suite-reduction})
or if the runtime performance of jqwik is improved.

\paragraph{Number of \tries{}}

Increasing the number of \tries{}
directly increases the number test inputs
that need to be generated and exercised
during the execution of the property-based tests.
For example, as shown in Figure~\ref{fig:test-runtime-differences},
 execution time of \VariantNaive{} tests
is, on average, 286.13 milliseconds (ms) longer per test
than for \VariantOriginal{} tests when using 10 \tries{}
(a +91.3~\% increase compared to the \VariantBaseline{} overhead of 149.56~ms),
348.56~ms (+118.5~\%) longer with 50 \tries{},
and 1136.21~ms (+659.7~\%) longer with 200 \tries{}.
Similarly, execution time of \VariantImproved{} tests
is 189.17~ms (+26.5~\%) longer with 10 \tries{},
246.85~ms (+65.1~\%) longer with 50 \tries{},
and 395.26~ms (+164.3~\%) longer with 200 \tries{}.
While overall runtime increases
get larger as \texttt{tries} increase,
runtime increases per \texttt{try}
get smaller as \texttt{tries} increase.
More specifically, \VariantNaive{} variants
show per-\texttt{try} increases
of 28.61~ms, 6.97~ms, and 5.68~ms.
\VariantImproved{} variants show
per-\texttt{try} increases 
of 18.92~ms, 4.94~ms, and 1.98~ms.
However, to attribute these observed
improvements in per-\texttt{try} efficiency to any specific causes
would require a more thorough microbenchmarking setup
that properly accounts for confounding factors such JVM warmup,
which is beyond the scope of this evaluation.

\paragraph{\VariantNaive{} vs.\ \VariantImproved{} Generalization}

Runtime increases are generally larger
for \VariantNaive{} than for \VariantImproved{}
when comparing the same tests.
For example, \VariantNaiveC{} and \VariantImprovedC{}
both generalize the same three tests of \DatasetCommonsDev{}
(see Table~\ref{tab:tests-per-project}).
Nevertheless, as shown in Table~\ref{tab:runtime-per-project},
generalization with \VariantNaiveC{}
increases overall test suite runtime by 4.54 seconds
(+57.1~\% compared to \VariantOriginal{}).
\VariantImprovedC{}, on the other hand,
only increases runtime by 0.74 seconds (+9.4\%).
The more pronounced runtime impact of \VariantNaive{} variants is caused by 
the input value generation approach used by these variants.
As described in Section~\ref{sec:naive-generalization},
\VariantNaive{} generalization selects inputs for test execution
by first randomly generating inputs
that match the parameter types of the tested method,
and then discarding any inputs that do not satisfy the input specification.
Especially for cases with restrictive input specifications (e.g., \texttt{a~==~b~\&\&~b~==~c}),
this causes a large runtime overhead
because many discard-and-regenerate cycles are required
until a valid set of inputs is identified
(or \ToolJqwik{} aborts the process due to \texttt{TooManyFilterMisses}).
\VariantImproved{} variants are less affected by this
because they already consider (some) input constraints
during initial input value generation
(as described in Section~\ref{sec:improved-generalization}).
As a result, fewer inputs need to be discarded and regenerated,
which lessens the overall runtime impact of \VariantImproved{} generalization
despite the more involved input value generation process.

\paragraph{Complexity of Constraints}

As complexity of input specifications increases,
required runtime also increases.
This is because more complex constraints
cause more filter-and-regenerate cycles to occur
during execution of property-based tests.
While \VariantNaive{} generalization
is more strongly affected by this than \VariantImproved{}
because it does not consider any constraints during value generation
(as discussed in the previous paragraph),
the issue also affects \VariantImproved{} generalization
in cases where less than 100~\% of constraints can be used
(for further details on constraint use,
see Sections~\ref{sec:improved-generalization} and \ref{sec:boundary-detection-effectiveness}).
For example, as shown in Table~\ref{tab:tests-per-project},
\VariantImprovedC{} adds 3 times as many tests (206 vs.\ 69) to \DatasetEqBenchA{}
as it adds to \DatasetCommonsA{}.
Nevertheless, the difference in runtime increases
is much less pronounced at +578.7~\% vs.\ +680.5~\%.
This is because input specifications in \DatasetEqBenchA{}
have fewer operations (mean: 159.1 vs.\ 208.0, median: 10 vs.\ 15)
and constraints (mean: 6.5 vs.\ 7.5, median: 2 vs.\ 5)
than in \DatasetCommonsA{}, 
and fewer of these constraints
can be used during input value generation
(mean: 42.9~\% vs.61.5~\%, median: 80~\% vs.\ 100~\%).
This indicates that better support for more complex input constraints
not only holds the potential to increase detection rates
(as suggested in Section~\ref{sec:boundary-detection-effectiveness})
but could also reduce the runtime cost of generalized tests at the same time.

@TODO: Add answer for RQ2.

\begin{table}[H]
\caption{Test suite runtime before and after generalization, with changes, per project.}
\label{tab:runtime-per-project}
\begin{tabular}{llrrrrrr}
\toprule
 & & \multicolumn{6}{c}{Runtime (in seconds)} \\
\cmidrule(lr){3-8}
Project & Variant & Before & Added & Removed & After & Delta & Delta \% \\
\midrule
\DatasetEqBenchA{} & \VariantNaiveC{} & 17.44 & 100.94 & 0.74 & 117.65 & +100.20 & +574.5 \% \\
\DatasetEqBenchB{} & \VariantNaiveC{} & 16.70 & 106.19 & 0.66 & 122.23 & +105.53 & +632.0 \% \\
\DatasetEqBenchC{} & \VariantNaiveC{} & 18.21 & 221.07 & 0.76 & 238.52 & +220.31 & +1210.0 \% \\
\midrule
\DatasetEqBenchA{} & \VariantImprovedC{} & 17.44 & 101.62 & 0.68 & 118.38 & +100.94 & +578.7 \% \\
\DatasetEqBenchB{} & \VariantImprovedC{} & 16.70 & 139.64 & 0.56 & 155.77 & +139.08 & +832.9 \% \\
\DatasetEqBenchC{} & \VariantImprovedC{} & 18.21 & 124.68 & 0.69 & 142.20 & +123.99 & +681.0 \% \\
\midrule
\DatasetCommonsA{} & \VariantNaiveC{} & 4.31 & 114.42 & 0.09 & 118.64 & +114.33 & +2651.5 \% \\
\DatasetCommonsB{} & \VariantNaiveC{} & 7.40 & 148.54 & 0.14 & 155.80 & +148.40 & +2005.2 \% \\
\DatasetCommonsC{} & \VariantNaiveC{} & 6.30 & 122.11 & 0.07 & 128.34 & +122.04 & +1936.2 \% \\
\midrule
\DatasetCommonsA{} & \VariantImprovedC{} & 4.31 & 29.49 & 0.14 & 33.66 & +29.35 & +680.5 \% \\
\DatasetCommonsB{} & \VariantImprovedC{} & 7.40 & 71.50 & 0.21 & 78.70 & +71.30 & +963.3 \% \\
\DatasetCommonsC{} & \VariantImprovedC{} & 6.30 & 28.07 & 0.08 & 34.29 & +27.99 & +444.1 \% \\
\midrule
\DatasetCommonsDev{} & \VariantNaiveC{} & 7.95 & 4.54 & 0.00 & 12.48 & +4.54 & +57.1 \% \\
\midrule
\DatasetCommonsDev{} & \VariantImprovedC{} & 7.95 & 0.74 & 0.00 & 8.69 & +0.74 & +9.4 \% \\
\bottomrule
\end{tabular}
\end{table}

\begin{figure}[H]
  \centering
  \includegraphics[width=\linewidth]{fig_test_runtime_differences}
  \caption{Runtime Comparison: Test vs. Generalization.}
  %\Description{@TODO}
  \label{fig:test-runtime-differences}
\end{figure}

@TODO: Split Figure~\ref{fig:test-runtime-differences} into two (sub-)figures to make each one individually referenceable.
