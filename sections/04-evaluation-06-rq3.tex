\subsection{RQ3: Runtime Requirements}
\label{sec:runtime-eval}

As previously described in Section~\ref{sec:evaluation-setup},
we measure the runtime requirements of \ToolTeralizer{}
by executing the tool once for each of the seven \DatasetsEqBenchEs{} and \DatasetsCommons{} projects
using a MacBook Air with M2 processor and 24~GB of memory.
Since all generalization variants require the same input data
(i.e., source code of target projects as well as extracted input/output specifications),
\ToolTeralizer{} executes the corresponding processing steps
(i.e., project instrumentation, test analysis, and specification extraction)
only once and then reuses the collected specifications
across all generalizations of the project.
Because no other tools exist
that transform conventional unit tests to property-based tests,
we measure the efficiency of \ToolTeralizer{}
by comparing the mutation score increase that \ToolTeralizer{} achieves 
when increasing the used \tries{} (thus increasing its runtime)
to the corresponding increase that \ToolEvoSuite{} achieves
when increasing the test generation timeout.

\subsubsection{Execution Time of \ToolTeralizer{}}
\label{sec:execution-time}

\subsubsection{Efficiency of \ToolTeralizer{} vs.\ \ToolEvoSuite{}}
\label{sec:execution-efficience}

\begin{table}[H]
  \caption{Total runtimes of Teralizer for all evaluated projects.}
  \label{tab:teralizer-runtimes}
  \begin{tabular}{lr}
    \toprule
    Project & Runtime \\
    \midrule
    \DatasetEqBenchA{} & 24h 46min 27s \\
    \DatasetEqBenchB{} & 28h 16min 32s \\
    \DatasetEqBenchC{} & 30h 53min 22s \\
    \midrule
    \DatasetCommonsA{} & 8h 13min 55s \\
    \DatasetCommonsB{} & 9h 46min 48s \\
    \DatasetCommonsC{} & 9h 04min 32s \\
    \midrule
    \DatasetCommonsDev{} & 3h 23min 22s \\
    \bottomrule
  \end{tabular}
\end{table}

\begin{figure}[H]
  \centering
  \includegraphics[width=\linewidth]{figures/fig_teralizer_runtimes}
  \caption{Teralizer runtimes per project, processing stage, and generalization variant.}
  %\Description{@TODO}
  \label{fig:teralizer-runtimes}
\end{figure}


\begin{samepage}
  \begin{figure}[H]
    \centering
    \includegraphics[width=\textwidth]{figures/fig_teralizer_efficiency}
    \caption{Pareto fronts for EvoSuite and Teralizer variants across projects.}
    \label{fig:teralizer-efficiency}
  \end{figure}
  \begin{table}[H]
    \begin{minipage}[t]{0.48\textwidth}
      \centering
      \begin{table}[H]
  \caption{Pareto points for project: eqbench.}
  \label{tab:pareto-eqbench}
  \begin{tabular}{rrlrr}
    \toprule
    Pt. & EvoSuite & Teralizer & Det. \% & Runtime (s) \\
    \midrule
    1 & 1s & - & 48.1 & 26,479 \\
    2 & 10s & - & 50.6 & 29,861 \\
    3 & 1s & NAIVE$_{10}$ & 50.7 & 36,728 \\
    4 & 1s & IMPROVED$_{50}$ & 51.4 & 37,457 \\
    5 & 1s & NAIVE$_{50}$ & 51.7 & 37,532 \\
    6 & 10s & IMPROVED$_{10}$ & 51.9 & 41,525 \\
    7 & 10s & IMPROVED$_{50}$ & 53.6 & 42,256 \\
    8 & 10s & NAIVE$_{50}$ & 53.8 & 45,398 \\
    9 & 10s & IMPROVED$_{200}$ & 53.8 & 48,269 \\
    10 & 10s & NAIVE$_{200}$ & 54.1 & 62,938 \\
    11 & 60s & IMPROVED$_{50}$ & 54.5 & 68,093 \\
    12 & 60s & NAIVE$_{50}$ & 54.7 & 68,782 \\
    13 & 60s & IMPROVED$_{200}$ & 54.8 & 75,081 \\
    14 & 60s & NAIVE$_{200}$ & 55.0 & 93,017 \\
    \bottomrule
  \end{tabular}
\end{table}
    \end{minipage}
    \hfill
    \begin{minipage}[t]{0.48\textwidth}
      \centering
      \begin{table}[H]
  \caption{Pareto points for project: commons-utils.}
  \label{tab:pareto-commons}
  \begin{tabular}{rrlrr}
    \toprule
    Pt. & EvoSuite & Teralizer & Det. \% & Runtime (s) \\
    \midrule
    1 & 1s & - & 56.8 & 4648.7 \\
    2 & 10s & - & 57.3 & 5597.3 \\
    3 & 1s & IMPROVED$_{10}$ & 57.9 & 7294.5 \\
    4 & 60s & - & 58.1 & 10239.8 \\
    5 & 10s & NAIVE$_{10}$ & 58.1 & 10445.1 \\
    6 & 10s & IMPROVED$_{50}$ & 58.4 & 10603.4 \\
    7 & 10s & IMPROVED$_{10}$ & 58.4 & 11081.5 \\
    8 & 10s & IMPROVED$_{200}$ & 58.5 & 13270.4 \\
    9 & 60s & IMPROVED$_{10}$ & 59.3 & 13938.7 \\
    10 & 60s & IMPROVED$_{50}$ & 59.4 & 14727.5 \\
    11 & 60s & IMPROVED$_{200}$ & 59.5 & 15735.7 \\
    \bottomrule
  \end{tabular}
\end{table}
    \end{minipage}
  \end{table}
\end{samepage}



