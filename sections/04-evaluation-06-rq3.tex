\subsection{RQ3: Runtime Requirements}
\label{sec:runtime-eval}

We measured \ToolTeralizer{}'s runtime across seven project variants
to assess viability compared to existing automated testing tools.
Since no existing tools perform automated test generalization from unit tests to property-based tests,
we compared efficiency against \ToolEvoSuite{}, the established test generation tool.
Runtime results were collected on a MacBook Air with M2 processor and 24~GB of memory.
Shared processing stages were executed only once per project by \ToolTeralizer{}
and collected specifications were then reused across all seven generalization variants.
Processing required a total time of 3.4 hours for \DatasetCommonsDev{},
8.2--9.8 hours for \DatasetsCommonsEs{} projects,
and 24.8--30.9 hours for \DatasetsEqBenchEs{} projects.
Despite the relatively high runtime cost,
Pareto analysis (Section~\ref{sec:evosuite-efficiency}) confirms
that combining low search budget \ToolEvoSuite{} generation and \ToolTeralizer{} generalization
often achieves better detection-to-runtime ratios than simply running \ToolEvoSuite{} with higher search budgets.
Within the overall processing pipeline,
validation through mutation testing consumes 40.5--81.1\% of total processing time.
This concentration of computational cost in validation
indicates that future optimizations should target
the mutation testing phase rather than transformation algorithms.


\subsubsection{Execution Time of \ToolTeralizer{}}
\label{sec:execution-time}

Runtime distribution across processing stages reveals
that validation dominates computational cost.
Figure~\ref{fig:teralizer-runtimes} shows
generalization validation consuming 1,110--35,538 seconds (40.5--78.5\% of total processing time)
while specification extraction requires 65--2,793 seconds (0.7--8.2\%)
and test transformation takes 5--173 seconds (0.1--0.5\%).
This resembles \ToolEvoSuite{}'s distribution,
where the JUnit checking phase,
which is most comparable to \ToolTeralizer{}'s
generalization validation stage,
consumes 86\%, 72\%, and 34\% of median runtime
for 1s, 10s, and 60s search budgets respectively.

\paragraph{Non-Validation Runtimes}

Non-validation stages complete efficiently
despite containing the core generalization logic.
Specification extraction (65--2,793 seconds) uses \ToolSPF{}
to concretely execute tests in constraint collection mode,
extracting path conditions and symbolic outputs
without requiring any constraint solving calls
(Section~\ref{sec:specification-extraction}).
While this introduces some overhead
because the JVM implementation of \ToolJPF / \ToolSPF{}
is less optimized than production-ready JVM implementations
(\ToolJPF{} itself runs inside of a host JVM process \cite{TODO}),
the cost is comparatively small, representing a one-time cost
of ca.\ 100$\times$ the runtime of an original test suite execution.
Test transformation cost is almost negligible at a one-time runtime cost
of only 5--173 seconds per generalization variant across all projects.
The reason for this low cost is that test transformation
only performs syntactic replacements, e.g.,
converting JUnit annotations to \ToolJqwik{} ones,
wrapping input constraint encodings in \texttt{TestParameters} classes,
and replacing expected values in assertions 
(Section~\ref{sec:test-transformation}).
Consequently, neither specification extraction nor test transformation
offer much opportunity for further runtime improvements.

\paragraph{Validation Runtimes}

Validation costs are substantially larger than non-validation costs
due to the high runtime cost of mutation testing,
which is by far the largest contributing factor in all validation stages.
For example, even on the \VariantOriginal{} test suite
without any added generalized tests,
execution of \VariantOriginal{} validation (701--3,898 seconds)
generally requires 1.5--10$\times$ as much runtime as specification extraction,
with mutation testing representing 88\% of the total runtime cost of this stage on average.
Execution times for \VariantInitial{} validation (652--4,131 seconds)
are generally slightly lower than for \VariantOriginal{} validation
because of the filtering that takes place
between the two stages (Section~\ref{sec:all-filtering}).
Validation of generalized test suites has higher runtime requirements (1,110--35,538 seconds)
than for the \VariantOriginal{} and \VariantInitial{} ones
for largely the same reasons that individual property-based \ToolJqwik{} tests
take longer to execute than conventional JUnit tests, i.e.,
\ToolJqwik{} overhead, larger number of tested inputs,
as well as filter-and-regenerate cycles
which occur more often for \VariantNaive{} variants
and in the presence of more complex input constraints (Section~\ref{sec:test-suite-execution-time}).

While it might seem tempting to forgo any validation,
this would result in considerable increases to the execution times of generalized test suites.
After all, one of the primary purpose of validation stages is
(i)~to provide a baseline for mutation detection comparisons
(\VariantOriginal{} / \VariantInitial{} validation),
and (ii)~to identify which generalized tests do not
measurably increase mutation detection rates (generalization validation),
thereby enabling \ToolTeralizer{} to remove non-contributing tests from the final generalized test suite.
In our evaluation, this filtering reduces the number of retained generalized tests
from a total of 65,633 to 4,240 across all projects and variants
(Section~\ref{sec:test-suite-test-count}),
thus demonstrating the impact
that validation-based filtering has on generalization outcomes.
Even though there is a non-negligible one-time cost associated with this,
that cost amortizes over time compared a longer-running test suite
that incurs further cost with every repeated execution.
Future optimization efforts could focus on validation efficiency
through faster mutation testing approaches
or lightweight pre-filtering heuristics
that identify likely-beneficial candidates before full validation.

\subsubsection{Efficiency of \ToolTeralizer{} vs.\ \ToolEvoSuite{}}
\label{sec:execution-efficiency}

\begin{table}[H]
  \caption{Total runtimes of Teralizer for all evaluated projects.}
  \label{tab:teralizer-runtimes}
  \begin{tabular}{lr}
    \toprule
    Project & Runtime \\
    \midrule
    \DatasetEqBenchA{} & 24h 46min 27s \\
    \DatasetEqBenchB{} & 28h 16min 32s \\
    \DatasetEqBenchC{} & 30h 53min 22s \\
    \midrule
    \DatasetCommonsA{} & 8h 13min 55s \\
    \DatasetCommonsB{} & 9h 46min 48s \\
    \DatasetCommonsC{} & 9h 04min 32s \\
    \midrule
    \DatasetCommonsDev{} & 3h 23min 22s \\
    \bottomrule
  \end{tabular}
\end{table}

\begin{figure}[H]
  \centering
  \includegraphics[width=\linewidth]{figures/fig_teralizer_runtimes}
  \caption{Teralizer runtimes per project, processing stage, and generalization variant.}
  %\Description{@TODO}
  \label{fig:teralizer-runtimes}
\end{figure}


\begin{samepage}
  \begin{figure}[H]
    \centering
    \includegraphics[width=\textwidth]{figures/fig_teralizer_efficiency}
    \caption{Pareto fronts for EvoSuite and Teralizer variants across projects.}
    \label{fig:teralizer-efficiency}
  \end{figure}
  \begin{table}[H]
    \begin{minipage}[t]{0.48\textwidth}
      \centering
      \begin{table}[H]
  \caption{Pareto points for project: eqbench.}
  \label{tab:pareto-eqbench}
  \begin{tabular}{rrlrr}
    \toprule
    Pt. & EvoSuite & Teralizer & Det. \% & Runtime (s) \\
    \midrule
    1 & 1s & - & 48.1 & 26,479 \\
    2 & 10s & - & 50.6 & 29,861 \\
    3 & 1s & NAIVE$_{10}$ & 50.7 & 36,728 \\
    4 & 1s & IMPROVED$_{50}$ & 51.4 & 37,457 \\
    5 & 1s & NAIVE$_{50}$ & 51.7 & 37,532 \\
    6 & 10s & IMPROVED$_{10}$ & 51.9 & 41,525 \\
    7 & 10s & IMPROVED$_{50}$ & 53.6 & 42,256 \\
    8 & 10s & NAIVE$_{50}$ & 53.8 & 45,398 \\
    9 & 10s & IMPROVED$_{200}$ & 53.8 & 48,269 \\
    10 & 10s & NAIVE$_{200}$ & 54.1 & 62,938 \\
    11 & 60s & IMPROVED$_{50}$ & 54.5 & 68,093 \\
    12 & 60s & NAIVE$_{50}$ & 54.7 & 68,782 \\
    13 & 60s & IMPROVED$_{200}$ & 54.8 & 75,081 \\
    14 & 60s & NAIVE$_{200}$ & 55.0 & 93,017 \\
    \bottomrule
  \end{tabular}
\end{table}
    \end{minipage}
    \hfill
    \begin{minipage}[t]{0.48\textwidth}
      \centering
      \begin{table}[H]
  \caption{Pareto points for project: commons-utils.}
  \label{tab:pareto-commons}
  \begin{tabular}{rrlrr}
    \toprule
    Pt. & EvoSuite & Teralizer & Det. \% & Runtime (s) \\
    \midrule
    1 & 1s & - & 56.8 & 4648.7 \\
    2 & 10s & - & 57.3 & 5597.3 \\
    3 & 1s & IMPROVED$_{10}$ & 57.9 & 7294.5 \\
    4 & 60s & - & 58.1 & 10239.8 \\
    5 & 10s & NAIVE$_{10}$ & 58.1 & 10445.1 \\
    6 & 10s & IMPROVED$_{50}$ & 58.4 & 10603.4 \\
    7 & 10s & IMPROVED$_{10}$ & 58.4 & 11081.5 \\
    8 & 10s & IMPROVED$_{200}$ & 58.5 & 13270.4 \\
    9 & 60s & IMPROVED$_{10}$ & 59.3 & 13938.7 \\
    10 & 60s & IMPROVED$_{50}$ & 59.4 & 14727.5 \\
    11 & 60s & IMPROVED$_{200}$ & 59.5 & 15735.7 \\
    \bottomrule
  \end{tabular}
\end{table}
    \end{minipage}
  \end{table}
\end{samepage}



