\subsection{Evaluation Setup}
\label{sec:evaluation-setup}

To collect the necessary data to answer RQ1-RQ4,
we executed \ToolTeralizer{} once for every project in the evaluation dataset
on a MacBook Air with M2 processor and 24~GB of memory
(for threats to validity resulting from this setup, see Section~\ref{sec:threats-to-validity}).
The JVM settings for \texttt{InitialHeapSize} and \texttt{MaxHeapSize}
were kept at their default values, i.e., 384~MB (${1/64}$th of memory) for \texttt{InitialialHeapSize} and 6~GB (${1/4}$th of memory) for \texttt{MaxHeapSize}.
Further details about the used execution environments and settings are provided
in the \textit{Setup} sections of the the individual research questions.
Since there are no existing approaches
that automatically create property-based tests from existing (unit) tests,
no other tools are included in the evaluation.
Instead, we compare the original,
human written and \ToolEvoSuite{} generated test suites in the evaluation dataset
to the augmented / generalized test suites created by \ToolTeralizer{}.
As described in Section~\ref{sec:approach},
\ToolTeralizer{} creates a total of nine variants
of each target project / test suite
throughout its execution
to analyze how different settings or combinations thereof
affect the generalization results.
We refer to these variants as:

\begin{itemize}
  \item \VariantOriginal{}:
    The original project without any modifications applied by \ToolTeralizer{}.
  \item \VariantInitial{}:
    The state of the project
    after project instrumentation, test analysis, and specification extraction have concluded 
    (see Sections~\ref{sec:project-instrumentation}--\ref{sec:specification-extraction})
    but before any tests have been generalized.
    Tests for which these steps were not successful
    are excluded from this project variant and,
    therefore, are not included in any further generalization steps.
    An overview of excluded tests is provided in Section~\ref{sec:filtering-eval}
    when discussing the results of RQ4.
  \item \VariantBaseline{}:
    The state of the project
    after \VariantBaseline{} generalization has been applied
    (see Section~\ref{sec:baseline-generalization}).
  \item \VariantNaiveA{}, \VariantNaiveB{}, \VariantNaiveC{}:
    The state of the project
    after \VariantNaive{} generalization has been applied.
    As described in Section~\ref{sec:naive-generalization},
    the subscript values indicate the values used for \ToolJqwik{}'s \tries{} setting.
  \item \VariantImprovedA{}, \VariantImprovedB{}, \VariantImprovedC{}:
    The state of the project
    after \VariantImproved{} generalization has been applied
    (see Section~\ref{sec:improved-generalization}).
    The subscript values indicate the values used for \ToolJqwik{}'s \tries{} setting.
\end{itemize}

In addition to the source code of the two pre-generalization variants
(\VariantOriginal{} and \VariantInitial)
as well as the seven post-generalization variants
(\VariantBaseline{},
\VariantNaive{} with 10 / 50 / 200 \tries{},
and \VariantImproved{} with 10 / 50 / 200 \tries{}),
\ToolTeralizer{} also creates and stores (meta) data and (intermediate) processing results
at various stages throughout its execution
in a PostgreSQL database and in log files.
This data includes, for example,
(i) descriptions of identified tests and assertions
(e.g., names, involved data types, lines of code),
(ii) information about executed processing tasks
(e.g., processing status, causes of failures, execution time),
and (iii) raw tool outputs
(e.g., console output logs, JUnit / \ToolJacoco{} / \ToolPit{} reports , extracted input-/output-specifications).
For a more detailed overview of the collected data, see Section~\ref{sec:collected-data}.
The evaluation results presented in the following sections
are generated from the collected data via Jupyter notebooks.
To aid independent validation and replication of our results,
all of the collected data, the full Java implementation of \ToolTeralizer{},
and the Jupyter notebooks used for the evaluation
are publicly available in our replication package~\cite{replicationpackage}.
