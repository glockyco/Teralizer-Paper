\section{Conclusions}
\label{sec:conclusions}

This paper introduced a semantics-based approach
for automated test generalization,
using specifications extracted through single-path symbolic analysis
to transform conventional unit tests into property-based tests.
We implemented this approach in a prototype tool called \ToolTeralizer{}.
Under controlled conditions matching current symbolic analysis capabilities,
\ToolTeralizer{} achieves mutation score improvements of 1--4 percentage points
compared to \ToolEvoSuite{}-generated baselines.
Pareto analysis further showed that combining short test generation with test generalization
can outperform longer generation alone. For example,
1-second generation plus generalization
achieves a higher mutation score on \DatasetEqBench{}
than 60-second generation (51.7\% vs 51.6\%)
while requiring 32\% less total runtime.

However, our evaluation across 632 real-world Java projects
from the RepoReapers dataset reveals substantial barriers
to fully automated generalization under real-world conditions:
only 1.7\% of projects complete the processing pipeline,
and 98.3\% of assertions are excluded before reaching generalized test creation.
By analyzing these exclusions in detail,
we distinguish implementation limitations of our prototype
from fundamental research challenges in specification extraction,
providing concrete guidance for advancing the field.
The primary barrier to fully automated generalization
is limited type support in existing symbolic analysis tools and approaches:
current tools cannot precisely encode constraints for strings, arrays, and objects,
causing the majority of assertion-level exclusions.

As symbolic analysis improves to support additional types,
semantics-based test generalization would directly benefit.
Other limitations of \ToolTeralizer{}
are addressable through engineering improvements
without requiring research advances:
interprocedural analysis would recover assertions in helper methods,
broader framework support would reduce test-level exclusions,
and extended constraint encoding in generated tests would improve effectiveness and efficiency
by reducing filter-and-regenerate cycles.
Our complete implementation and replication package
are publicly available to support reproduction and extension of this work~\cite{replicationpackage}.
