\section{Conclusions}
\label{sec:conclusions}

We have presented \ToolTeralizer{},
a semantics-based approach for automated transformation
of conventional unit tests into property-based tests.
Our work demonstrates the feasibility of semantics-based test generalization
while systematically mapping the boundaries between current capabilities
and broader practical applicability.

\subsection{Summary of Contributions}

This research makes four key contributions to the testing community:

\textbf{Semantics-based test generalization.}
Unlike prior approaches such as JARVIS that generalize from examples using predefined abstraction templates,
\ToolTeralizer{} extracts path-exact specifications
directly from the implementation through single-path symbolic analysis.
Our evaluation shows consistent mutation score improvements:
3--4 percentage points on \DatasetsEqBenchEs{} projects
and 1--2 percentage points on \DatasetsCommonsEs{} projects,
showing that semantics-based generalization achieves measurable improvements
with stronger effects on automatically-generated tests.
% Data source: mutation-detection-figure-data.csv

\textbf{Comprehensive empirical dataset characterizing test generalizability.}
Beyond demonstrating feasibility, our analysis produces
a systematic characterization of which test structures and program types
support semantics-based generalization.
The three-tier evaluation design reveals success rates from 50\% under favorable conditions
to 0.9\% in practical application,
with detailed metrics on filtering outcomes and failure causes.
This dataset, available in our replication package, provides
the community with empirical foundations for future research.

\textbf{Systematic mapping of applicability barriers.}
Our analysis distinguishes between fundamental research challenges
and engineering barriers that could be addressed through implementation effort.
Type support limitations stem from the difficulty of representing and solving constraints over non-numeric types,
while method resolution failures (24.7--57.9\% of exclusions)
represent addressable engineering challenges.
This mapping provides concrete guidance for prioritizing future improvements.

\textbf{Open implementation and replication package.}
We provide a complete implementation of our approach
along with comprehensive experimental data and analysis scripts.
This enables both reproduction of our results
and extension to additional domains,
supporting continued research in automated test generalization.

\subsection{Key Empirical Findings}

Our evaluation reveals several unexpected patterns
that challenge conventional assumptions about property-based testing.
Most notably, random input generation outperforms constraint-aware generation
for Math mutations (59\% of all mutants),
while the opposite holds for boundary-related mutations.
This finding suggests that effective property-based testing requires
a portfolio of generation strategies rather than sophisticated single approaches.

The contrast between automatically-generated and developer-written tests
illuminates design trade-offs in test construction.
While \ToolEvoSuite{}'s single-assertion-per-test structure enables
near-complete test replacement after generalization,
developer tests' multi-assertion structure blocks such consolidation
despite successful individual assertion generalization.
This reveals a tension between test maintainability and amenability to automated analysis.

Runtime analysis shows that the generalization approach itself is efficient,
consuming less than 10\% of total processing time.
The dominant cost comes from validation through mutation testing,
suggesting that selective application to high-value tests
could achieve most benefits at substantially reduced computational cost.

\subsection{Implications for Testing Practice}

Our results have several implications for testing research and practice.
Test generation tools could consider future generalizability
when designing test architectures,
potentially creating more modular tests that facilitate later enhancement.
The demonstrated efficiency of staged approaches
(quick initial generation followed by selective generalization)
suggests alternatives to monolithic extended test generation.

For practitioners, our findings establish realistic expectations:
test generalization is most effective for automatically-generated test suites
on numeric computations, with diminishing returns on mature developer test suites.
The 0.9\% success rate on real-world projects reflects current implementation limitations
rather than fundamental impossibility,
pointing toward specific engineering improvements that could enhance practical applicability.

\subsection{Future Research Directions}

Several research directions emerge from our analysis.
The mutation-type-specific effectiveness of generation strategies
suggests investigating adaptive property-based testing frameworks
that select generation approaches based on code characteristics.
The test architecture insights point toward automated refactoring techniques
that could decompose multi-assertion tests to enable generalization
while preserving consolidated versions for maintenance.

The systematic barrier analysis identifies clear engineering priorities:
enhanced method resolution for complex inheritance patterns,
runtime optimization through parallelization and commercial acceleration tools,
and exploration of test suite reduction through partition equivalence analysis.
Each of these represents concrete improvement opportunities
with measurable impact on practical applicability.

Looking beyond our current scope,
extending generalization to richer data types remains
a fundamental research challenge requiring advances in symbolic reasoning about complex data structures.
The integration of machine learning approaches for constraint satisfaction
could address the high failure rates we observe for complex constraints,
while alternative specification extraction techniques
might enable generalization beyond the numeric domain.

The test generalization problem space, as characterized by our systematic evaluation,
provides a foundation for continued research in automated testing enhancement.
Our open implementation and comprehensive dataset
enable the community to build upon these initial findings,
working toward more broadly applicable automated test improvement techniques.
