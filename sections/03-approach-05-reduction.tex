\subsection{Test Suite Reduction}
\label{sec:test-suite-reduction}

The test suite reduction stage uses mutation testing to identify and retain
only generalized tests that improve fault detection capability
beyond what the original test suite provides.
Since generalized tests exercise more inputs than their original counterparts,
they may detect additional mutants that the original test suite missed.
However, not all generalized tests provide this improvement:
some may only detect mutants already caught by other tests in the suite.

\ToolTeralizer{} uses \ToolPit{} to execute mutation testing
on three test suite variants:
\VariantOriginal{} (the original test suite),
\VariantInitial{} (the subset with successful specification extraction),
and the generalized test variants (\VariantBaseline{}, \VariantNaive{}, \VariantImproved{}).
By comparing which mutants each suite detects,
we identify generalized tests that catch mutants not detected by \VariantOriginal{}.

The selection criterion is straightforward:
retain only generalized tests that detect mutants
not caught by the original test suite.
This ensures that every generalized test in the final suite
contributes unique fault detection capability.
Generalized tests that only detect mutants already caught by existing tests
are excluded as redundant.

The final test suite combines the retained generalized tests
with the original test suite,
providing both the original developer-validated test cases
and the additional fault detection capability from generalization.
This approach maintains backward compatibility
while strengthening the test suite's ability to detect regressions
within the execution paths already covered by the original tests.
