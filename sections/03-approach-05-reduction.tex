\subsection{Test Suite Reduction}
\label{sec:test-suite-reduction}

Not all generalized tests provide value to the test suite.
Despite covering many more inputs than the original tests,
some property-based tests still only detect the same faults
as the corresponding conventional unit tests from which they were created.
Thus, these generalized tests
increase runtime costs of overall test suite execution
but do not provide any tangible benefits in exchange for this.
Similarly, successful generalization may render some original tests redundant.
This is because tests created by all three of \ToolTeralizer{}'s generalization variants
are designed to use the original input values as the first
set of inputs exercised during property-based test execution.
To address these inefficiencies,
\ToolTeralizer{} performs test suite reduction
as the final stage of the generalization pipeline,
using mutation testing to measure each test's contribution to fault detection
and retaining only those tests that strengthen the suite's effectiveness.

\ToolTeralizer{} evaluates fault detection capability
using the \texttt{DEFAULTS} group of mutation operators
provided by \ToolPit{}~\cite{coles_2016_pit}.
This group provides a stable set of operators that minimize equivalent mutants
and avoid subsumption~\cite{coles_2021_less_is_more, coles_pit_mutators}.
Thus, the group excludes mutations
that would be redundant or trivially detected,
ensuring that each generated mutant represents a meaningful potential fault.
The full set of 12 mutation operators from the \texttt{DEFAULTS} group
is listed in Table~\ref{tab:pit-mutators}.
Each row in the table shows the name of the mutator,
a short description of its behavior,
and an example that shows a source code representation
of the mutator's effects.
The operators can be roughly categorized by the type of mutation they produce.
The first subgroup modifies arithmetic operations.
The second one replaces return values.
The third one modifies conditionals,
and the fourth one removes calls to methods
that have \texttt{void} as their return type.

To perform test suite reduction, \ToolTeralizer{}
first executes mutation testing on the original
test suite as well as the non-reduced test suites
created by the three test generalization variants.
By comparing which mutants each configuration detects,
\ToolTeralizer{} identifies generalized tests
that catch mutants not detected by the original test suite.
The selection criterion is straightforward:
retain only generalized tests that detect at least one mutant
not caught by the original test suite.
This ensures that every generalized test in the final suite
contributes unique fault detection capability,
while those that only detect already-caught mutants are excluded as redundant.

Beyond filtering generalized tests, \ToolTeralizer{} identifies original tests
that can be removed without loss of test suite effectiveness.
An original test is removable when all of its assertions
have been successfully generalized into property-based tests.
For tests containing a single assertion, successful generalization means
the property-based test validates that assertion across the entire input partition,
making the original single-input validation redundant.
For tests containing multiple assertions, removal requires that
every assertion has been successfully transformed.
If any assertion cannot be generalized
(due to type limitations, failed MUT identification, or other filtering criteria)
the original test must be retained to preserve that validation.
The final test suite of each variant
combines the retained generalized tests
with the necessary original tests.

\begin{table}[t]
  \caption{Mutation operators from \ToolPit{}'s DEFAULTS group used in our evaluation.}
  \label{tab:pit-mutators}
  \begin{tabular}{l l l l}
    \toprule
    &&\multicolumn{2}{l}{Example} \\
    \cmidrule{3-4}
    Mutator & Description  & Before & After\\
    \midrule
    Math                       & Replaces arithmetic operations            & \texttt{x + y}       & \texttt{x - y} \\
    Increments                 & Replaces increment/decrement              & \texttt{i++}         & \texttt{i{-}{-}} \\
    InvertNegs                 & Inverts negation of variables             & \texttt{return -x}   & \texttt{return x} \\
    \midrule
    BooleanTrueReturnVals      & Returns \texttt{true} for booleans        & \texttt{return b}    & \texttt{return true} \\
    BooleanFalseReturnVals     & Returns \texttt{false} for booleans       & \texttt{return b}    & \texttt{return false} \\
    PrimitiveReturns           & Returns \texttt{0} for numeric primitives & \texttt{return a}    & \texttt{return 0} \\
    EmptyObjectReturnVals      & Returns empty for strings                 & \texttt{return s}    & \texttt{return ""} \\
    NullReturnVals             & Returns \texttt{null} for objects         & \texttt{return o}    & \texttt{return null} \\
    \midrule
    RemoveConditionalEqualElse & Forces else for equality checks           & \texttt{if (a == b)} & \texttt{if (false)} \\
    RemoveConditionalOrderElse & Forces else for inequality checks         & \texttt{if (a < b)}  & \texttt{if (false)} \\
    ConditionalsBoundary       & Changes boundary of inequalities          & \texttt{if (a < b)}  & \texttt{if (a <= b)} \\
    \midrule
    VoidMethodCall             & Removes void method calls                 & \texttt{foo(...)}    & \texttt{/* removed */} \\
    \bottomrule
  \end{tabular}
\end{table}
