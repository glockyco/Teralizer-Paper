\subsection{Experimental Framework}
\label{sec:experimental-framework}

\subsubsection{Mutation Testing Methodology}
\label{sec:mutation-testing-methodology}

Mutation testing provides our primary measure of test effectiveness
by systematically injecting faults into the code and measuring whether tests detect them~\cite{jia2011analysis}.
A test suite that detects more mutants demonstrates stronger fault-finding capability.
We use \ToolPit{}~\cite{coles2016pit} with its DEFAULTS mutation operator group,
the recommended set for Java projects.
Understanding which mutations each generalization strategy detects is essential
for interpreting our results, as different strategies show distinct strengths.
Table~\ref{tab:pit-mutators} describes the specific operators we apply.

\begin{table}[t]
  \caption{Mutation operators from \ToolPit{}'s DEFAULTS group used in our evaluation.}
  \label{tab:pit-mutators}
  \begin{tabular}{l l l l}
    \toprule
    &&\multicolumn{2}{l}{Example} \\
    \cmidrule{3-4}
    Mutator & Description  & Before & After\\
    \midrule
    Math                       & Replaces arithmetic operations            & \texttt{x + y}       & \texttt{x - y} \\
    Increments                 & Replaces increment/decrement              & \texttt{i++}         & \texttt{i{-}{-}} \\
    InvertNegs                 & Inverts negation of variables             & \texttt{return -x}   & \texttt{return x} \\
    \midrule
    BooleanTrueReturnVals      & Returns \texttt{true} for booleans        & \texttt{return b}    & \texttt{return true} \\
    BooleanFalseReturnVals     & Returns \texttt{false} for booleans       & \texttt{return b}    & \texttt{return false} \\
    PrimitiveReturns           & Returns \texttt{0} for numeric primitives & \texttt{return a}    & \texttt{return 0} \\
    EmptyObjectReturnVals      & Returns empty for strings                 & \texttt{return s}    & \texttt{return ""} \\
    NullReturnVals             & Returns \texttt{null} for objects         & \texttt{return o}    & \texttt{return null} \\
    \midrule
    RemoveConditionalEqualElse & Forces else for equality checks           & \texttt{if (a == b)} & \texttt{if (false)} \\
    RemoveConditionalOrderElse & Forces else for inequality checks         & \texttt{if (a < b)}  & \texttt{if (false)} \\
    ConditionalsBoundary       & Changes boundary of inequalities          & \texttt{if (a < b)}  & \texttt{if (a <= b)} \\
    \midrule
    VoidMethodCall             & Removes void method calls                 & \texttt{foo(...)}    & \texttt{/* removed */} \\
    \bottomrule
  \end{tabular}
\end{table}

\subsubsection{Datasets}
\label{sec:datasets}

Our evaluation employs three complementary datasets that progressively
reveal the gap between ideal and real-world conditions for test generalization.

\paragraph{EqBench}
The \DatasetEqBench{} benchmark~\cite{badihi_2021_eqbench} provides ideal conditions
for automated test generalization.
Originally designed for equivalence checking research,
its 652 Java classes implement equivalent and non-equivalent program pairs
focusing on numeric computations while deliberately avoiding
features that complicate automated reasoning (recursion, reflection, complex object graphs).
Since EqBench provides only implementation code without tests,
we generated test suites using \ToolEvoSuite{}~\cite{fraser2011evosuite}
with three search budgets (1s, 10s, 60s per class),
creating datasets \DatasetEqBenchA{}, \DatasetEqBenchB{}, and \DatasetEqBenchC{}.
This design explores the relationship between initial test suite characteristics
and generalization effectiveness, allowing us to observe whether the diminishing
returns from stronger baselines outweigh the benefits of better test diversity.

\paragraph{Apache Commons}
To bridge toward real-world complexity,
we systematically extracted numeric utility methods from Apache Commons projects.
Using Sourcegraph's code search,
we identified public static methods with numeric or boolean parameters and return values,
the types currently supported by \ToolTeralizer{}'s constraint extraction.
This yielded 247 classes from 106 source files,
including all transitively called methods and dependencies
to ensure compilation (19,709 LOC total).
We created four dataset variants:
\DatasetCommonsDev{} preserves the 725 original developer-written tests (14,389 LOC),
while \DatasetCommonsA{}, \DatasetCommonsB{}, and \DatasetCommonsC{}
use \ToolEvoSuite{}-generated tests with 1s, 10s, and 60s search budgets respectively.
This enables direct comparison of generalization effectiveness between
mature developer tests that combine multiple scenarios per method
and generated tests that follow a one-scenario-per-test approach.

\paragraph{RepoReapers}
To understand current limitations in practical settings,
we selected 1,160 projects from RepoReapers~\cite{munaiah_2017_reporeapers},
a curated collection of 2.9 million GitHub repositories
specifically filtered for software engineering research quality.
Our selection criteria balanced technical constraints with evaluation goals:
projects needed Java 5--8 (for \ToolSPF{} compatibility),
JUnit 4/5 with Maven builds (for our prototype),
standard directory structures (for automated processing),
mid-size codebases (5,000--50,000 LOC),
and substantial test suites (20--80\% of total code).
The selected projects collectively comprise 95,404 implementation classes 
and 53,039 test classes across diverse domains and coding styles.
While \ToolTeralizer{} succeeds on EqBench and partially on Apache Commons,
the RepoReapers projects expose fundamental barriers to real-world deployment
analyzed in RQ4 (Section~\ref{sec:filtering-eval-extended}).

Table~\ref{tab:dataset-statistics} reveals important structural differences:
EvoSuite plateaus quickly at ~4,900 tests regardless of budget,
while developer tests achieve similar coverage with 3.5$\times$ fewer
but more complex tests combining multiple scenarios per method.

\begin{table}
  \caption{Number of files, classes, source lines of code (SLOC), and test methods per project.}
  \label{tab:dataset-statistics}
  \begin{tabular}{lrrrrrrr}
    \toprule
    & \multicolumn{3}{r}{Implementation} & \multicolumn{4}{r}{Test} \\
    \cmidrule(lr){2-4} \cmidrule(lr){5-8}
    Project & Files & Classes & SLOC & Files & Classes & SLOC & Methods \\
    \midrule
    \DatasetEqBenchA{} & 544 & 652 & 27,871 & 544 & 544 & 35,666 & 4,718 \\
    \DatasetEqBenchB{} & 544 & 652 & 27,871 & 543 & 543 & 36,937 & 4,875 \\
    \DatasetEqBenchC{} & 544 & 652 & 27,871 & 544 & 544 & 37,836 & 4,974 \\
    \midrule
    \DatasetCommonsA{} & 106 & 247 & 19,709 & 103 & 103 & 17,524 & 2,481 \\
    \DatasetCommonsB{} & 106 & 247 & 19,709 & 103 & 103 & 19,082 & 2,738 \\
    \DatasetCommonsC{} & 106 & 247 & 19,709 & 102 & 102 & 18,839 & 2,735 \\
    \midrule
    \DatasetCommonsDev{} & 106 & 247 & 19,709 & 80 & 119 & 14,389 & 725 \\
    \midrule
    \DatasetRepoReapers{} (total) & 41,292 & 50,474 & 2,735,127 & 22,281 & 30,894 & 2,012,601 & 81,810 \\
    \DatasetRepoReapers{} (mean) & 65 & 79 & 4,320 & 35 & 48 & 3,179 & 162 \\
    \DatasetRepoReapers{} (median) & 49 & 56 & 3,253 & 23 & 26 & 2,107 & 86 \\
    \bottomrule
  \end{tabular}
\end{table}

\subsubsection{Test Variants and Environment}
\label{sec:test-variants}

We evaluate nine test suite variants per project to isolate different aspects
of the generalization approach:

\begin{itemize}
  \item \VariantOriginal{}: Original project before any processing
  \item \VariantInitial{}: After filtering unsuitable tests (baseline for comparisons)
  \item \VariantBaseline{}: Property-based format with original inputs only (isolates framework overhead)
  \item \VariantNaiveA{}, \VariantNaiveB{}, \VariantNaiveC{}: Random generation with constraint filtering (10/50/200 \tries{})
  \item \VariantImprovedA{}, \VariantImprovedB{}, \VariantImprovedC{}: Constraint-aware generation (10/50/200 \tries{})
\end{itemize}

All experiments ran on a MacBook Air with M2 processor and 24~GB memory
using default JVM settings.
\ToolTeralizer{} captures comprehensive metrics throughout execution,
available in our replication package~\cite{replicationpackage}.