%%
%% This is file `sample-manuscript.tex',
%% generated with the docstrip utility.
%%
%% The original source files were:
%%
%% samples.dtx  (with options: `all,proceedings,bibtex,manuscript')
%% 
%% IMPORTANT NOTICE:
%% 
%% For the copyright see the source file.
%% 
%% Any modified versions of this file must be renamed
%% with new filenames distinct from sample-manuscript.tex.
%% 
%% For distribution of the original source see the terms
%% for copying and modification in the file samples.dtx.
%% 
%% This generated file may be distributed as long as the
%% original source files, as listed above, are part of the
%% same distribution. (The sources need not necessarily be
%% in the same archive or directory.)
%%
%%
%% Commands for TeXCount
%TC:macro \cite [option:text,text]
%TC:macro \citep [option:text,text]
%TC:macro \citet [option:text,text]
%TC:envir table 0 1
%TC:envir table* 0 1
%TC:envir tabular [ignore] word
%TC:envir displaymath 0 word
%TC:envir math 0 word
%TC:envir comment 0 0
%%
%% The first command in your LaTeX source must be the \documentclass
%% command.
%%
%% For submission and review of your manuscript please change the
%% command to \documentclass[manuscript, screen, review]{acmart}.
%%
%% When submitting camera ready or to TAPS, please change the command
%% to \documentclass[sigconf]{acmart} or whichever template is required
%% for your publication.
%%
%%
\documentclass[manuscript,screen,review]{acmart}
%%
%% \BibTeX command to typeset BibTeX logo in the docs
\AtBeginDocument{%
  \providecommand\BibTeX{{%
    Bib\TeX}}}

%% Rights management information.  This information is sent to you
%% when you complete the rights form.  These commands have SAMPLE
%% values in them; it is your responsibility as an author to replace
%% the commands and values with those provided to you when you
%% complete the rights form.
\setcopyright{acmlicensed}
\copyrightyear{2025}
\acmYear{2025}
\acmDOI{XXXXXXX.XXXXXXX}
%% These commands are for a PROCEEDINGS abstract or paper.
% \acmConference[Conference acronym 'XX]{Make sure to enter the correct
%   conference title from your rights confirmation email}{June 03--05,
%   2018}{Woodstock, NY}
%%
%%  Uncomment \acmBooktitle if the title of the proceedings is different
%%  from ``Proceedings of ...''!
%%
%\acmBooktitle{ACM Transactions on Software Engineering and Methodology}
% \acmISBN{978-1-4503-XXXX-X/2018/06}
\acmJournal{TOSEM}

%%
%% Submission ID.
%% Use this when submitting an article to a sponsored event. You'll
%% receive a unique submission ID from the organizers
%% of the event, and this ID should be used as the parameter to this command.
%%\acmSubmissionID{123-A56-BU3}

%%
%% For managing citations, it is recommended to use bibliography
%% files in BibTeX format.
%%
%% You can then either use BibTeX with the ACM-Reference-Format style,
%% or BibLaTeX with the acmnumeric or acmauthoryear sytles, that include
%% support for advanced citation of software artefact from the
%% biblatex-software package, also separately available on CTAN.
%%
%% Look at the sample-*-biblatex.tex files for templates showcasing
%% the biblatex styles.
%%

%%
%% The majority of ACM publications use numbered citations and
%% references.  The command \citestyle{authoryear} switches to the
%% "author year" style.
%%
%% If you are preparing content for an event
%% sponsored by ACM SIGGRAPH, you must use the "author year" style of
%% citations and references.
%% Uncommenting
%% the next command will enable that style.
%%\citestyle{acmauthoryear}


%%
%% end of the preamble, start of the body of the document source.
\begin{document}

%%
%% The "title" command has an optional parameter,
%% allowing the author to define a "short title" to be used in page headers.
% \title[short title]{full title}
\title{Teralizer: Something About Automated Test Generalization}

%%
%% The "author" command and its associated commands are used to define
%% the authors and their affiliations.
%% Of note is the shared affiliation of the first two authors, and the
%% "authornote" and "authornotemark" commands
%% used to denote shared contribution to the research.
\author{Johann Glock}
\email{johann.glock@aau.at}
\orcid{0000-0002-0152-8611}
\affiliation{%
  \institution{University of Klagenfurt}
  \city{Klagenfurt}
  \country{Austria}
}

\author{Clemens Bauer}
\email{clemens.bauer@aau.at}
\orcid{0009-0000-9199-8563}
\affiliation{%
  \institution{University of Klagenfurt}
  \city{Klagenfurt}
  \country{Austria}
}

\author{Martin Pinzger}
\email{martin.pinzger@aau.at}
\orcid{0000-0002-5536-3859}
\affiliation{%
  \institution{University of Klagenfurt}
  \city{Klagenfurt}
  \country{Austria}
}

%%
%% By default, the full list of authors will be used in the page
%% headers. Often, this list is too long, and will overlap
%% other information printed in the page headers. This command allows
%% the author to define a more concise list
%% of authors' names for this purpose.
\renewcommand{\shortauthors}{Glock et al.}

%%
%% The abstract is a short summary of the work to be presented in the
%% article.
\begin{abstract}
  Abstract
\end{abstract}

%%
%% The code below is generated by the tool at http://dl.acm.org/ccs.cfm.
%% Please copy and paste the code instead of the example below.
%%
\begin{CCSXML}
<ccs2012>
 <concept>
  <concept_id>00000000.0000000.0000000</concept_id>
  <concept_desc>Do Not Use This Code, Generate the Correct Terms for Your Paper</concept_desc>
  <concept_significance>500</concept_significance>
 </concept>
 <concept>
  <concept_id>00000000.00000000.00000000</concept_id>
  <concept_desc>Do Not Use This Code, Generate the Correct Terms for Your Paper</concept_desc>
  <concept_significance>300</concept_significance>
 </concept>
 <concept>
  <concept_id>00000000.00000000.00000000</concept_id>
  <concept_desc>Do Not Use This Code, Generate the Correct Terms for Your Paper</concept_desc>
  <concept_significance>100</concept_significance>
 </concept>
 <concept>
  <concept_id>00000000.00000000.00000000</concept_id>
  <concept_desc>Do Not Use This Code, Generate the Correct Terms for Your Paper</concept_desc>
  <concept_significance>100</concept_significance>
 </concept>
</ccs2012>
\end{CCSXML}

\ccsdesc[500]{Do Not Use This Code~Generate the Correct Terms for Your Paper}
\ccsdesc[300]{Do Not Use This Code~Generate the Correct Terms for Your Paper}
\ccsdesc{Do Not Use This Code~Generate the Correct Terms for Your Paper}
\ccsdesc[100]{Do Not Use This Code~Generate the Correct Terms for Your Paper}

%%
%% Keywords. The author(s) should pick words that accurately describe
%% the work being presented. Separate the keywords with commas.
\keywords{Do, Not, Us, This, Code, Put, the, Correct, Terms, for,
  Your, Paper}

% \received{20 February 2007}
% \received[revised]{12 March 2009}
% \received[accepted]{5 June 2009}

\received{n/a}
\received[revised]{n/a}
\received[accepted]{n/a}

%%
%% A "teaser figure" is an image, or set of images in one figure,
%% that are placed after all author and affiliation information,
%% and before the body of the article, spanning the page. If you
%% wish to have such a figure in your article, place the command
%% immediately before the \maketitle command:
%% \begin{teaserfigure}
%%   \includegraphics[width=\textwidth]{sampleteaser}
%%   \caption{figure caption}
%%   \Description{figure description}
%% \end{teaserfigure}

%%
%% This command processes the author and affiliation and title
%% information and builds the first part of the formatted document.
\maketitle

% ------------------------------------------------------------------------------
% INTRODUCTION
% ------------------------------------------------------------------------------

\section{Introduction}
\label{sec:introduction}

% ------------------------------------------------------------------------------
% BACKGROUND
% ------------------------------------------------------------------------------

\section{Background}
\label{sec:background}

% ------------------------------------------------------------------------------
% APPROACH
% ------------------------------------------------------------------------------

\section{Approach}
\label{sec:approach}

\subsection{Specification Extraction}

\subsubsection{Code Instrumentation}
\subsubsection{JPF Execution}

\subsection{Test Generalization}

\subsubsection{"NAIVE" Generalization}
\subsubsection{"IMPROVED" Generalization}

\subsection{Filtering}

\subsubsection{Test-level Filtering}
\subsubsection{Assertion-level Filtering}
\subsubsection{Generalization-level Filtering}

% ------------------------------------------------------------------------------
% EVALUATION
% ------------------------------------------------------------------------------

\section{Evaluation}
\label{sec:evaluation}

primary effects, ancillary effects, runtime requirements, limitations

\begin{itemize}
  \item RQ1: To which degree does generalization affect the mutation score of the target test suites?
  \item RQ2: To which degree does generalization affect the size and runtime of the target test suites?
  \item RQ3: What are the runtime requirements of the generalization approach?
  \item RQ4: What are the causes of unsuccessful generalization attempts?
\end{itemize}

\subsection{Target Programs}

selection criteria: can be processed by SPF (=> java 8!, numeric inputs + outputs, no randomness), use junit 4 or 5 for testing, use maven or gradle as build system, ...?

\subsubsection{EqBench}

well-suited for processing with SPF, but no test suite available;

tests generated with EvoSuite;
came "third" in SBFT Tool Competition 2025 - Java Test Case Generation Track (https://arxiv.org/pdf/2504.09168), but differences between top 3 not statistically significant per the paper;
can't find a summary paper for SBFT 2024;
performed best on the SBFT Tool Competition 2023 - Java Test Case Generation Track for line and branch coverage metrics, and second-best for understandability metric;
also the overall winner for SBST Tool Competition 2022 and SBST Tool Competition 2021 which use line + branch (via JaCoCo) and mutation coverage (via PIT) as metrics;

3 different test suites with search budgets: 1s, 10s, 60s (= EvoSuite default);
no other changes to default EvoSuite configuration

descriptive statistics (evosuite runtime, number of classes, number of tests, LOC, ...)

\subsubsection{Apache Commons Utils}

manually collected;
utility classes from apache commons projects;
classes identified via SourceGraph (based on SPF limitations);
corresponding test classes manually identified;
additionally same 3 EvoSuite generated test suites as above

descriptive statistics (evosuite runtime, number of classes, number of tests, LOC, ...)

\subsection{Evaluation Setup}

Hardware: MacBook Air, M2, 24 GB memory

target programs as above

variants:

\begin{itemize}
  \item ORIGINAL: before any processing
  \item INITIAL: after SPF execution
  \item BASELINE: after generalization (only original test inputs)
  \item NAIVE (with 10 / 50 / 200 tries): after generalization (naive input selection)
  \item IMPROVED (with 10 / 50 / 200 tries): after generalization (improved input selection)
\end{itemize}

\subsection{RQ1: Effects on Mutation Score}

number of mutants per project + mutator (for variant INITIAL?)

number of covered / uncovered mutants per project (for variant INITIAL?)
mention degree of coverage change as a sidenote (no (large) change is by design)

percentage of survived / detected / ... mutants per project + variant (relative to number of covered mutants)

number of newly killed mutants  per project + variant

number / percentage of killing generalizations

differences between killed / unkilled mutants (here or in RQ4?)

\subsection{RQ2: Effects on Test Suite Size and Runtime}

effects on the number of tests in the test suite

effects on the number of lines of code in the test suite

effects on the execution time of individual tests

effects on the total execution time of the test suite

\subsection{RQ3: Runtime Requirements}

only one run for each configuration (due to high runtime requirements)
investigation for a subset of configurations confirmed that the trends hold

total runtime (+ runtime per project?)

runtime per project + processing stage + variant

efficiency relative to higher EvoSuite search budgets

\subsection{RQ4: Causes of Unsuccessful Generalizations}

overall test / assertion / generalization exclusions

filtering-based test / assertion / generalization exclusions

exclusions caused by JFP execution failures

differences between killed / unkilled mutants (here or in RQ1?)

% ------------------------------------------------------------------------------
% DISCUSSION
% ------------------------------------------------------------------------------

\section{Discussion}
\label{sec:discussion}

\subsection{Benefits of the Approach}

\subsection{Potential for Future Improvements}

\subsubsection{Improving the Mutation Score of Generalized Tests}
\subsubsection{Improving the Size of Generalized Tests}
\subsubsection{Using Test Generalization for Test Suite Reduction}

\subsection{Threats to Validity}

\subsubsection{Construct Validity}
\subsubsection{Internal Validity}
\subsubsection{External Validity}

% ------------------------------------------------------------------------------
% RELATED WORK
% ------------------------------------------------------------------------------

\section{Related Work}
\label{sec:related-work}

% ------------------------------------------------------------------------------
% CONCLUSIONS
% ------------------------------------------------------------------------------

\section{Conclusions}
\label{sec:conclusions}

%%
%% The acknowledgments section is defined using the "acks" environment
%% (and NOT an unnumbered section). This ensures the proper
%% identification of the section in the article metadata, and the
%% consistent spelling of the heading.
\begin{acks}
This research was funded in whole or in part by the Austrian Science Fund (FWF) 10.55776/P36698. For open access purposes, the author has applied a CC BY public copyright license to any author accepted manuscript version arising from this submission.
\end{acks}

%%
%% The next two lines define the bibliography style to be used, and
%% the bibliography file.
\bibliographystyle{ACM-Reference-Format}
\bibliography{main}

\end{document}
\endinput
%%
%% End of file `sample-manuscript.tex'.
