%% 
%% Copyright 2007-2025 Elsevier Ltd
%% 
%% This file is part of the 'Elsarticle Bundle'.
%% ---------------------------------------------
%% 
%% It may be distributed under the conditions of the LaTeX Project Public
%% License, either version 1.3 of this license or (at your option) any
%% later version.  The latest version of this license is in
%%    http://www.latex-project.org/lppl.txt
%% and version 1.3 or later is part of all distributions of LaTeX
%% version 1999/12/01 or later.
%% 
%% The list of all files belonging to the 'Elsarticle Bundle' is
%% given in the file `manifest.txt'.
%% 
%% Template article for Elsevier's document class `elsarticle'
%% with harvard style bibliographic references

%\documentclass[preprint,12pt,authoryear]{elsarticle}
\documentclass[preprint,5p,times,twocolumn,authoryear]{elsarticle}

%% Use the option review to obtain double line spacing
%% \documentclass[authoryear,preprint,review,12pt]{elsarticle}

%% Use the options 1p,twocolumn; 3p; 3p,twocolumn; 5p; or 5p,twocolumn
%% for a journal layout:
%% \documentclass[final,1p,times,authoryear]{elsarticle}
%% \documentclass[final,1p,times,twocolumn,authoryear]{elsarticle}
%% \documentclass[final,3p,times,authoryear]{elsarticle}
%% \documentclass[final,3p,times,twocolumn,authoryear]{elsarticle}
%% \documentclass[final,5p,times,authoryear]{elsarticle}
%% \documentclass[final,5p,times,twocolumn,authoryear]{elsarticle}

%% For including figures, graphicx.sty has been loaded in
%% elsarticle.cls. If you prefer to use the old commands
%% please give \usepackage{epsfig}

%% The amssymb package provides various useful mathematical symbols
\usepackage{amssymb}
%% The amsmath package provides various useful equation environments.
\usepackage{amsmath}
%% The amsthm package provides extended theorem environments
%% \usepackage{amsthm}

%% The lineno packages adds line numbers. Start line numbering with
%% \begin{linenumbers}, end it with \end{linenumbers}. Or switch it on
%% for the whole article with \linenumbers.
%% \usepackage{lineno}

\journal{Journal of Systems and Software}

\begin{document}

\begin{frontmatter}

%% Title, authors and addresses

%% use the tnoteref command within \title for footnotes;
%% use the tnotetext command for theassociated footnote;
%% use the fnref command within \author or \affiliation for footnotes;
%% use the fntext command for theassociated footnote;
%% use the corref command within \author for corresponding author footnotes;
%% use the cortext command for theassociated footnote;
%% use the ead command for the email address,
%% and the form \ead[url] for the home page:
%% \title{Title\tnoteref{label1}}
%% \tnotetext[label1]{}
%% \author{Name\corref{cor1}\fnref{label2}}
%% \ead{email address}
%% \ead[url]{home page}
%% \fntext[label2]{}
%% \cortext[cor1]{}
%% \affiliation{organization={},
%%            addressline={}, 
%%            city={},
%%            postcode={}, 
%%            state={},
%%            country={}}
%% \fntext[label3]{}

\title{} %% Article title

%% use optional labels to link authors explicitly to addresses:
%% \author[label1,label2]{}
%% \affiliation[label1]{organization={},
%%             addressline={},
%%             city={},
%%             postcode={},
%%             state={},
%%             country={}}
%%
%% \affiliation[label2]{organization={},
%%             addressline={},
%%             city={},
%%             postcode={},
%%             state={},
%%             country={}}

\author{} %% Author name

%% Author affiliation
\affiliation{organization={},%Department and Organization
            addressline={}, 
            city={},
            postcode={}, 
            state={},
            country={}}

%% Abstract
\begin{abstract}
%% Text of abstract
Abstract text.
\end{abstract}

%%Graphical abstract
%\begin{graphicalabstract}
%\includegraphics{grabs}
%\end{graphicalabstract}

%%Research highlights
%\begin{highlights}
%\item Research highlight 1
%\item Research highlight 2
%\end{highlights}

%% Keywords
\begin{keyword}
%% keywords here, in the form: keyword \sep keyword

%% PACS codes here, in the form: \PACS code \sep code

%% MSC codes here, in the form: \MSC code \sep code
%% or \MSC[2008] code \sep code (2000 is the default)

\end{keyword}

\end{frontmatter}

%% Add \usepackage{lineno} before \begin{document} and uncomment 
%% following line to enable line numbers
%% \linenumbers

%% main text
%%

% ----------------------------------------------------------------------------------------------------------------------
% INTRODUCTION
% ----------------------------------------------------------------------------------------------------------------------

\section{Introduction}
\label{sec:introduction}

% ----------------------------------------------------------------------------------------------------------------------
% BACKGROUND
% ----------------------------------------------------------------------------------------------------------------------

\section{Background}
\label{sec:background}

% ----------------------------------------------------------------------------------------------------------------------
% APPROACH
% ----------------------------------------------------------------------------------------------------------------------

\section{Approach}
\label{sec:approach}

\subsection{Specification Extraction}

\subsubsection{Code Instrumentation}
\subsubsection{JPF Execution}

\subsection{Test Generalization}

\subsubsection{"NAIVE" Generalization}
\subsubsection{"IMPROVED" Generalization}

\subsection{Filtering}

\subsubsection{Test-level Filtering}
\subsubsection{Assertion-level Filtering}
\subsubsection{Generalization-level Filtering}

% ----------------------------------------------------------------------------------------------------------------------
% EVALUATION
% ----------------------------------------------------------------------------------------------------------------------

\section{Evaluation}
\label{sec:evaluation}

primary effects, ancillary effects, runtime requirements, limitations

\begin{itemize}
  \item RQ1: To which degree does generalization affect the mutation score of the target test suites?
  \item RQ2: To which degree does generalization affect the size and runtime of the target test suites?
  \item RQ3: What are the runtime requirements of the generalization approach?
  \item RQ4: What are the causes of unsuccessful generalization attempts?
\end{itemize}

\subsection{Target Programs}

\subsubsection{EqBench}

no test suite available;
tests generated with EvoSuite;
3 different test suites with search budgets: 1s, 10s, 60s (= EvoSuite default);
no other changes to default EvoSuite configuration

descriptive statistics (evosuite runtime, number of classes, number of tests, LOC, ...)

\subsubsection{Apache Commons Utils}

manually collected;
utility classes from apache commons projects;
classes identified via SourceGraph (based on SPF limitations);
corresponding test classes manually identified;
additionally same 3 EvoSuite generated test suites as above

descriptive statistics (evosuite runtime, number of classes, number of tests, LOC, ...)

\subsection{Evaluation Setup}

Hardware: MacBook Air, M2, 24 GB memory

target programs as above

variants:

\begin{itemize}
  \item ORIGINAL: before any processing
  \item INITIAL: after SPF execution
  \item BASELINE: after generalization (only original test inputs)
  \item NAIVE (with 10 / 50 / 200 tries): after generalization (naive input selection)
  \item IMPROVED (with 10 / 50 / 200 tries): after generalization (improved input selection)
\end{itemize}

\subsection{RQ1: Effects on Mutation Score}

number of mutants per project + mutator (for variant INITIAL?)

number of covered / uncovered mutants per project (for variant INITIAL?)
mention degree of coverage change as a sidenote (no (large) change is by design)

percentage of survived / detected / ... mutants per project + variant (relative to number of covered mutants)

number of newly killed mutants  per project + variant

number / percentage of killing generalizations

differences between killed / unkilled mutants (here or in RQ4?)

\subsection{RQ2: Effects on Test Suite Size and Runtime}

effects on the number of tests in the test suite

effects on the number of lines of code in the test suite

effects on the execution time of individual tests

effects on the total execution time of the test suite

\subsection{RQ3: Runtime Requirements}

only one run for each configuration (due to high runtime requirements)
investigation for a subset of configurations confirmed that the trends hold

total runtime (+ runtime per project?)

runtime per project + processing stage + variant

efficiency relative to higher EvoSuite search budgets

\subsection{RQ4: Causes of Unsuccessful Generalizations}

overall test / assertion / generalization exclusions

filtering-based test / assertion / generalization exclusions

exclusions caused by JFP execution failures

differences between killed / unkilled mutants (here or in RQ1?)

% ----------------------------------------------------------------------------------------------------------------------
% DISCUSSION
% ----------------------------------------------------------------------------------------------------------------------

\section{Discussion}
\label{sec:discussion}

\subsection{Benefits of the Approach}

\subsection{Potential for Future Improvements}

\subsubsection{Improving the Mutation Score of Generalized Tests}
\subsubsection{Improving the Size of Generalized Tests}
\subsubsection{Using Test Generalization for Test Suite Reduction}

\subsection{Threats to Validity}

\subsubsection{Construct Validity}
\subsubsection{Internal Validity}
\subsubsection{External Validity}

% ----------------------------------------------------------------------------------------------------------------------
% RELATED WORK
% ----------------------------------------------------------------------------------------------------------------------

\section{Related Work}
\label{sec:related-work}

% ----------------------------------------------------------------------------------------------------------------------
% CONCLUSIONS
% ----------------------------------------------------------------------------------------------------------------------

\section{Conclusions}
\label{sec:conclusions}

%% If you have bib database file and want bibtex to generate the
%% bibitems, please use
%%
%%  \bibliographystyle{elsarticle-harv} 
%%  \bibliography{<your bibdatabase>}

%% else use the following coding to input the bibitems directly in the
%% TeX file.

%% Refer following link for more details about bibliography and citations.
%% https://en.wikibooks.org/wiki/LaTeX/Bibliography_Management

\begin{thebibliography}{00}

%% For authoryear reference style
%% \bibitem[Author(year)]{label}
%% Text of bibliographic item

% \bibitem[Lamport(1994)]{lamport94}
%   Leslie Lamport,
%   \textit{\LaTeX: a document preparation system},
%   Addison Wesley, Massachusetts,
%   2nd edition,
%   1994.

\end{thebibliography}
\end{document}

\endinput
%%
%% End of file `elsarticle-template-harv.tex'.


