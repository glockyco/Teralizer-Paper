%%
%% This is file `sample-manuscript.tex',
%% generated with the docstrip utility.
%%
%% The original source files were:
%%
%% samples.dtx  (with options: `all,proceedings,bibtex,manuscript')
%% 
%% IMPORTANT NOTICE:
%% 
%% For the copyright see the source file.
%% 
%% Any modified versions of this file must be renamed
%% with new filenames distinct from sample-manuscript.tex.
%% 
%% For distribution of the original source see the terms
%% for copying and modification in the file samples.dtx.
%% 
%% This generated file may be distributed as long as the
%% original source files, as listed above, are part of the
%% same distribution. (The sources need not necessarily be
%% in the same archive or directory.)
%%
%%
%% Commands for TeXCount
%TC:macro \cite [option:text,text]
%TC:macro \citep [option:text,text]
%TC:macro \citet [option:text,text]
%TC:envir table 0 1
%TC:envir table* 0 1
%TC:envir tabular [ignore] word
%TC:envir displaymath 0 word
%TC:envir math 0 word
%TC:envir comment 0 0
%%
%% The first command in your LaTeX source must be the \documentclass
%% command.
%%
%% For submission and review of your manuscript please change the
%% command to \documentclass[manuscript, screen, review]{acmart}.
%%
%% When submitting camera ready or to TAPS, please change the command
%% to \documentclass[sigconf]{acmart} or whichever template is required
%% for your publication.
%%
%%
\documentclass[manuscript,screen,review]{acmart}
%%
%% \BibTeX command to typeset BibTeX logo in the docs
\AtBeginDocument{%
  \providecommand\BibTeX{{%
    Bib\TeX}}}

%% Rights management information.  This information is sent to you
%% when you complete the rights form.  These commands have SAMPLE
%% values in them; it is your responsibility as an author to replace
%% the commands and values with those provided to you when you
%% complete the rights form.
\setcopyright{acmlicensed}
\copyrightyear{2025}
\acmYear{2025}
\acmDOI{XXXXXXX.XXXXXXX}
%% These commands are for a PROCEEDINGS abstract or paper.
% \acmConference[Conference acronym 'XX]{Make sure to enter the correct
%   conference title from your rights confirmation email}{June 03--05,
%   2018}{Woodstock, NY}
%%
%%  Uncomment \acmBooktitle if the title of the proceedings is different
%%  from ``Proceedings of ...''!
%%
%\acmBooktitle{ACM Transactions on Software Engineering and Methodology}
% \acmISBN{978-1-4503-XXXX-X/2018/06}
\acmJournal{TOSEM}

%%
%% Submission ID.
%% Use this when submitting an article to a sponsored event. You'll
%% receive a unique submission ID from the organizers
%% of the event, and this ID should be used as the parameter to this command.
%%\acmSubmissionID{123-A56-BU3}

%%
%% For managing citations, it is recommended to use bibliography
%% files in BibTeX format.
%%
%% You can then either use BibTeX with the ACM-Reference-Format style,
%% or BibLaTeX with the acmnumeric or acmauthoryear sytles, that include
%% support for advanced citation of software artefact from the
%% biblatex-software package, also separately available on CTAN.
%%
%% Look at the sample-*-biblatex.tex files for templates showcasing
%% the biblatex styles.
%%

%%
%% The majority of ACM publications use numbered citations and
%% references.  The command \citestyle{authoryear} switches to the
%% "author year" style.
%%
%% If you are preparing content for an event
%% sponsored by ACM SIGGRAPH, you must use the "author year" style of
%% citations and references.
%% Uncommenting
%% the next command will enable that style.
%%\citestyle{acmauthoryear}

\usepackage{float}
\usepackage{tabularx}
\usepackage{listings}
\usepackage{algorithm}
\usepackage{algorithmic}

\newcommand{\ToolTeralizer}{\textsc{Teralizer}}

\newcommand{\ToolJPFLong}{Java PathFinder}
\newcommand{\ToolJPF}{JPF}

\newcommand{\ToolSPFLong}{Symbolic PathFinder}
\newcommand{\ToolSPF}{SPF}

\newcommand{\ToolJqwik}{jqwik}
\newcommand{\ToolJacoco}{JaCoCo}
\newcommand{\ToolPit}{PIT}

\newcommand{\ToolEvoSuite}{\textsc{EvoSuite}}

\newcommand{\DatasetEqBench}{\textsc{EqBench}}
\newcommand{\DatasetEqBenchA}{eqbench-es-1s}
\newcommand{\DatasetEqBenchB}{eqbench-es-10s}
\newcommand{\DatasetEqBenchC}{eqbench-es-60s}
\newcommand{\DatasetsEqBenchEs}{eqbench-es-$*$}

\newcommand{\DatasetCommonsDev}{commons-utils-dev}
\newcommand{\DatasetCommonsA}{commons-utils-es-1s}
\newcommand{\DatasetCommonsB}{commons-utils-es-10s}
\newcommand{\DatasetCommonsC}{commons-utils-es-60s}
\newcommand{\DatasetsCommons}{commons-utils-$*$}
\newcommand{\DatasetsCommonsEs}{commons-utils-es-$*$}

\newcommand{\DatasetRepoReapers}{repo-reapers}

\newcommand{\VariantOriginal}{\textsc{Original}}
\newcommand{\VariantInitial}{\textsc{Initial}}
\newcommand{\VariantBaseline}{\textsc{Baseline}}
\newcommand{\VariantShared}{\textsc{Shared}}

\newcommand{\VariantNaive}{\textsc{Naive}}
\newcommand{\VariantNaiveA}{\textsc{Naive\textsubscript{10}}}
\newcommand{\VariantNaiveB}{\textsc{Naive\textsubscript{50}}}
\newcommand{\VariantNaiveC}{\textsc{Naive\textsubscript{200}}}

\newcommand{\VariantImproved}{\textsc{Improved}}
\newcommand{\VariantImprovedA}{\textsc{Improved\textsubscript{10}}}
\newcommand{\VariantImprovedB}{\textsc{Improved\textsubscript{50}}}
\newcommand{\VariantImprovedC}{\textsc{Improved\textsubscript{200}}}

\newcommand{\tries}{\texttt{tries}}

\floatstyle{plain}
\newfloat{genericfloat}{tbph}{lop}

\lstset{
  language=Java,
  basicstyle=\ttfamily\footnotesize,
  keywordstyle=\color{blue}\bfseries,
  commentstyle=\color{green!60!black},
  stringstyle=\color{red},
  numbers=none,
  breaklines=true,
  breakatwhitespace=true,
  tabsize=2,
  showstringspaces=false,
  frame=none
}

%%
%% end of the preamble, start of the body of the document source.
\begin{document}

%%
%% The "title" command has an optional parameter,
%% allowing the author to define a "short title" to be used in page headers.
% \title[short title]{full title}
\title{Teralizer: Semantics-Based Test Generalization from Conventional Unit Tests to Property-Based Tests}

%%
%% The "author" command and its associated commands are used to define
%% the authors and their affiliations.
%% Of note is the shared affiliation of the first two authors, and the
%% "authornote" and "authornotemark" commands
%% used to denote shared contribution to the research.
\author{Johann Glock}
\email{johann.glock@aau.at}
\orcid{0000-0002-0152-8611}
\affiliation{%
  \institution{University of Klagenfurt}
  \city{Klagenfurt}
  \country{Austria}
}

\author{Clemens Bauer}
\email{clemens.bauer@aau.at}
\orcid{0009-0000-9199-8563}
\affiliation{%
  \institution{University of Klagenfurt}
  \city{Klagenfurt}
  \country{Austria}
}

\author{Rudolf Ramler (?)}
\email{rudolf.ramler@scch.at}
\orcid{0000-0001-9903-6107}
\affiliation{%
  \institution{Software Competence Center Hagenberg}
  \city{Hagenberg}
  \country{Austria}
}

\author{Martin Pinzger}
\email{martin.pinzger@aau.at}
\orcid{0000-0002-5536-3859}
\affiliation{%
  \institution{University of Klagenfurt}
  \city{Klagenfurt}
  \country{Austria}
}

%%
%% By default, the full list of authors will be used in the page
%% headers. Often, this list is too long, and will overlap
%% other information printed in the page headers. This command allows
%% the author to define a more concise list
%% of authors' names for this purpose.
\renewcommand{\shortauthors}{Glock et al.}

%%
%% The abstract is a short summary of the work to be presented in the
%% article.
\begin{abstract}
Unit tests validate single input-output pairs, leaving vast input spaces unexplored within execution paths.
Property-based testing addresses this by generating multiple inputs satisfying properties,
but requires significant manual effort to create generators and specifications.
We present \ToolTeralizer{},
a semantics-based approach for automated transformation
of conventional unit tests into property-based tests.
Unlike prior work such as JARVIS,
which generalizes from input-output examples using predefined abstraction templates,
\ToolTeralizer{} extracts path-exact specifications directly from the implementation
through single-path symbolic analysis using \ToolSPF{}'s constraint collection mode.
% TODO: Clarify "test-method mapping" (it's actually an "assertion-method mapping").
Our approach comprises test-method mapping via data flow analysis,
specification extraction along concrete test paths,
and property-based test generation with \ToolJqwik{}.
Evaluation on three datasets reveals both potential and limitations.
On \ToolEvoSuite{}-generated tests, mutation scores improve 1--4 percentage points
(\DatasetEqBench{}: 48--52\%$\rightarrow$52--55\%, Commons: 57--58\%$\rightarrow$58--59\%).
% Data source: mutation-detection-figure-data.csv - worst to best case improvements across budgets
Developer-written tests show minimal improvement (80.35\%$\rightarrow$80.42\%),
suggesting diminishing returns for mature suites.
Analysis of 1,160 real-world projects identifies deployment barriers:
only 0.9\% process successfully,
with test-method mapping failures (24.7\%),
type limitations (15.4\%),
and generation failures (14.6--16.2\%) as primary obstacles.
We contribute:
(1)~semantics-based test generalization via single-path symbolic analysis,
(2)~comprehensive empirical dataset characterizing test generalizability,
(3)~systematic mapping distinguishing engineering from fundamental barriers,
(4)~open implementation and replication package.
\end{abstract}

%%
%% The code below is generated by the tool at http://dl.acm.org/ccs.cfm.
%% Please copy and paste the code instead of the example below.
%%
\begin{CCSXML}
<ccs2012>
   <concept>
       <concept_id>10011007.10011074.10011099.10011102.10011103</concept_id>
       <concept_desc>Software and its engineering~Software testing and debugging</concept_desc>
       <concept_significance>500</concept_significance>
       </concept>
   <concept>
       <concept_id>10003752.10010124.10010138.10010143</concept_id>
       <concept_desc>Theory of computation~Program analysis</concept_desc>
       <concept_significance>300</concept_significance>
       </concept>
   <concept>
       <concept_id>10011007.10011074.10011099.10011693</concept_id>
       <concept_desc>Software and its engineering~Empirical software validation</concept_desc>
       <concept_significance>100</concept_significance>
       </concept>
 </ccs2012>
\end{CCSXML}

\ccsdesc[500]{Software and its engineering~Software testing and debugging}
\ccsdesc[300]{Theory of computation~Program analysis}
\ccsdesc[100]{Software and its engineering~Empirical software validation}

%%
%% Keywords. The author(s) should pick words that accurately describe
%% the work being presented. Separate the keywords with commas.
\keywords{Test Amplification, Test Generalization, Property-Based Testing, Symbolic Execution}

% \received{20 February 2007}
% \received[revised]{12 March 2009}
% \received[accepted]{5 June 2009}

\received{n/a}
\received[revised]{n/a}
\received[accepted]{n/a}

%%
%% A "teaser figure" is an image, or set of images in one figure,
%% that are placed after all author and affiliation information,
%% and before the body of the article, spanning the page. If you
%% wish to have such a figure in your article, place the command
%% immediately before the \maketitle command:
%% \begin{teaserfigure}
%%   \includegraphics[width=\textwidth]{sampleteaser}
%%   \caption{figure caption}
%%   \Description{figure description}
%% \end{teaserfigure}

%%
%% This command processes the author and affiliation and title
%% information and builds the first part of the formatted document.
\maketitle

% ------------------------------------------------------------------------------
% INTRODUCTION
% ------------------------------------------------------------------------------

\section{Introduction}
\label{sec:introduction}

Conventional unit tests validate software behavior
by checking specific input-output pairs~\cite{orso_2014_software,ammann_2016_intro,myers_2011_art},
but leave most inputs along the same execution path untested.
Property-based testing~\cite{claessen_2000_quickcheck,hughes_2007_quickcheck} instead generates many inputs
and checks whether specified properties hold across executions.
For example, given the $abs$ method in Figure~\ref{fig:regression-detection},
a unit test which asserts that $abs(0)$ returns $0$ would 
still pass after changing \texttt{x >= 0} to \texttt{x == 0}, 
whereas a property-based test which asserts $abs(x) = x$ for $x \geq 0$ 
would expose this regression.
Industrial experience reports suggest that property-based testing 
often uncovers edge cases and boundary conditions missed by unit tests~\cite{hughes_2016_experiences,goldstein_2024_pbt_practice}.
Adoption, however, remains limited because writing property-based tests 
requires manual effort to define both input constraints and suitable properties,
a task practitioners find challenging~\cite{goldstein_2024_pbt_practice}.
This motivates research into transformation approaches
that automatically generalize existing unit tests by deriving properties from program semantics.

\begin{figure}[t]
  \centering
  \includegraphics[width=.95\linewidth]{figures/fig_regression_detection}
  \caption{The conventional unit test misses a regression that the property-based test detects.}
  \Description{Regression detection comparison between unit test and property-based test.
  Left shows buggy abs() implementation with incorrect condition (x == 0 instead of x >= 0),
  highlighted with red background for removed line and green for added line.
  Middle shows original @Test that still passes (1 test run, 0 failures)
  because it only tests input 0.
  Right shows @Property test that fails (1 test run, 1 failure)
  detecting the regression with input 1,
  showing "expected <1> but was <-1>" error message.}
  \label{fig:regression-detection}
\end{figure}

\begin{figure}[t]
  \centering
  \includegraphics[width=.95\linewidth]{figures/fig_generalization}
  \caption{\ToolTeralizer{} takes implementation and test code as input, and produces property-based tests as output.}
  \Description{Transformation workflow showing Teralizer converting a unit test to a property-based test.
  Left side shows the original abs() implementation and a JUnit test with @Test annotation
  that asserts abs(0) equals 0.
  Center shows the Teralizer transformation arrow.
  Right side shows the generated jqwik property-based test with @Property annotation,
  parameterized input (min=0) int x,
  and generalized assertion assertEquals(x, abs(x)).
  Annotations indicate replaced annotations, added constraints, added parameters,
  replaced expected values, and replaced arguments.}
  \label{fig:generalization}
\end{figure}

We propose a semantics-based approach for automated test generalization
that analyzes both test and implementation code
to derive path-exact specifications through single-path symbolic analysis~\cite{pasareanu_2013_symbolic}.
Our method determines which inputs follow the same execution path as existing tests
and transforms unit tests into property-based tests
that validate the same assertions across entire input partitions.
Because specifications are extracted directly from program semantics,
the resulting properties are exact for each execution path
and preserve the developer-provided oracles encoded in assertions.
%This enables thorough testing within those validated behaviors.
To our knowledge, JARVIS~\cite{peleg_2018_jarvis} is the only prior work
that automatically generalizes unit tests into property-based tests.
However, JARVIS infers properties from input-output examples based solely on test code,
relying on predefined abstraction templates that yield overapproximations.
In contrast, our white-box approach leverages both static and dynamic program analysis
to extract exact specifications for the execution paths exercised by the original tests.

We implemented this approach in \ToolTeralizer{}, a prototype tool for Java
that transforms JUnit tests into property-based \ToolJqwik{}~\cite{link_2022_jqwik} tests.
\ToolTeralizer{} employs a five-stage pipeline:
(1)~analyzing tests and their assertions regarding suitability for generalization,
(2)~identifying tested methods through data flow analysis,
(3)~extracting specifications through single-path symbolic analysis~\cite{pasareanu_2013_symbolic},
(4)~creating generalized property-based tests, and
(5)~filtering generalized tests to retain only those that improve fault detection capability.
Figure~\ref{fig:generalization} illustrates the effects of this transformation,
showing how a simple equality assertion $abs(0) = 0$
becomes the property $abs(x) = x$,
valid for all non-negative values of $x$.

To evaluate our approach,
we applied \ToolTeralizer{} to three complementary datasets.
The \DatasetEqBench{} benchmark~\cite{badihi_2021_eqbench} provides controlled settings
with numeric-focused programs well-suited for symbolic analysis.
Because \DatasetEqBench{} lacks test suites, we generated tests using \ToolEvoSuite{}~\cite{fraser_2011_evosuite}.
Utility methods extracted from Apache Commons projects
offer a middle ground between controlled and real-world scenarios.
Here, we directly compared \ToolEvoSuite{}-generated and developer-written tests
on the same codebase, partially isolating the influence of test architecture on generalization outcomes.
Finally, we applied \ToolTeralizer{} to 632 real-world Java projects with developer-written tests
from the RepoReapers dataset~\cite{munaiah_2017_reporeapers}
to expose the full complexity of practical application scenarios.
This progression from controlled to real-world conditions highlights
both the potential and limitations of semantics-based test generalization.

Our evaluation shows modest yet consistent improvements under controlled conditions.
On \ToolEvoSuite{}-generated tests, mutation scores increased by 1--4 percentage points:
from 48--52\% to 52--55\% on \DatasetEqBench{},
and from 57--58\% to 58--59\% on Apache Commons utilities.
In contrast, generalization of developer-written tests for Apache Commons utilities
showed only 0.05--0.07 percentage points improvement from a baseline of 80.35\%.
% This shows that the tests of these projects already detected most of the faults targeted by automated generalization.
%Beyond mutation detection improvements,
Results from the RepoReapers projects reveal practical applicability barriers:
% our analysis of 632 real-world projects reveals practical applicability barriers:
only 1.7\% of projects successfully completed the generalization pipeline.
Failures primarily occurred due to 
type support limitations of symbolic analysis
as well as static analysis limitations
of our prototype.
To provide a roadmap for future work,
we classified these failures into
those that can be resolved through engineering effort
and those that represent deeper research challenges in specification extraction and encoding.

This paper makes the following contributions:

\begin{enumerate}
\item A \textbf{semantics-based test generalization approach} that extracts specifications via symbolic analysis to transform conventional unit tests into property-based tests.
\item A comprehensive \textbf{empirical evaluation} across three complementary datasets, demonstrating 1--4 percentage point mutation score improvements under controlled conditions.
\item A systematic \textbf{analysis of applicability barriers}, distinguishing addressable engineering limitations from fundamental research challenges in specification extraction and encoding.
\item An \textbf{open implementation and replication package} \cite{replicationpackage}, enabling reproduction and extension of our results.
\end{enumerate}

The paper is organized as follows.
Section~\ref{sec:background} introduces the technical foundations of test generalization.
Section~\ref{sec:approach} presents \ToolTeralizer{}'s five-stage pipeline.
Section~\ref{sec:evaluation} evaluates our approach through six research questions,
covering mutation score improvements, impact on test suite size and execution time,
runtime requirements, and causes of unsuccessful generalizations.
Section~\ref{sec:discussion} discusses the results, directions for future work, and threats to validity.
Section~\ref{sec:related-work} positions our work within the broader testing literature,
and Section~\ref{sec:conclusions} concludes the paper.

\newpage{}
\section{Background}
\label{sec:background}

\subsection{Property-based Testing}
\label{sec:property-based-testing}

\subsection{Test Generalization}
\label{sec:test-generalization}

\subsection{Mutation Testing}
\label{sec:mutation-testing}

Mutation testing measures test suite effectiveness by introducing artificial faults (mutants) into the code and checking whether tests detect these faults~\cite{TODO}.
A test suite that detects more mutants is considered more effective at finding real bugs.
The approach works by systematically applying mutation operators—small syntactic changes that mimic common programming errors—to the source code.
If a test fails when executed against a mutant, the mutant is considered ``killed''; otherwise, it ``survives,'' indicating a potential weakness in the test suite.

For \ToolTeralizer{}, mutation testing serves a crucial role beyond evaluation: it filters generalized tests to retain only those that improve fault detection.
Without this quality control, property-based test generation could produce thousands of tests that execute different inputs but detect no additional faults.
Section~\ref{sec:all-filtering} describes how mutation testing enables \ToolTeralizer{} to reduce 65,633 generated tests to 4,240 effective ones.

We use \ToolPit{}~\cite{TODO} for mutation testing due to several practical advantages:
(i)~class-level exclusion compatibility with our per-assertion test generation,
(ii)~structured XML output suitable for automated analysis,
(iii)~support for Java 8 projects in our evaluation dataset, and
(iv)~active maintenance with regular updates.
Table~\ref{tab:pit-mutators} shows the mutation operators from \ToolPit{}'s DEFAULTS group used in our evaluation.
These operators cover common fault patterns including arithmetic mistakes, boundary errors, and incorrect boolean logic, providing a standard benchmark for measuring test effectiveness improvements.

\begin{table}[t]
  \caption{Mutators in \ToolPit{}'s DEFAULTS group used for evaluation.}
  \label{tab:pit-mutators}
  \begin{tabular}{l l l l}
    \toprule
    &&\multicolumn{2}{l}{Example} \\
    \cmidrule{3-4}
    Mutator & Description  & Before & After\\
    \midrule
    Math                       & Replaces arithmetic operations            & \texttt{x + y}       & \texttt{x - y} \\
    Increments                 & Replaces increment/decrement              & \texttt{i++}         & \texttt{i{-}{-}} \\
    InvertNegs                 & Inverts negation of variables             & \texttt{return -x}   & \texttt{return x} \\
    \midrule
    BooleanTrueReturnVals      & Returns \texttt{true} for booleans        & \texttt{return b}    & \texttt{return true} \\
    BooleanFalseReturnVals     & Returns \texttt{false} for booleans       & \texttt{return b}    & \texttt{return false} \\
    PrimitiveReturns           & Returns \texttt{0} for numeric primitives & \texttt{return a}    & \texttt{return 0} \\
    EmptyObjectReturnVals      & Returns empty for strings                 & \texttt{return s}    & \texttt{return ""} \\
    NullReturnVals             & Returns \texttt{null} for objects         & \texttt{return o}    & \texttt{return null} \\
    \midrule
    RemoveConditionalEqualElse & Forces else for equality checks           & \texttt{if (a == b)} & \texttt{if (false)} \\
    RemoveConditionalOrderElse & Forces else for inequality checks         & \texttt{if (a < b)}  & \texttt{if (false)} \\
    ConditionalsBoundary       & Changes boundary of inequalities          & \texttt{if (a < b)}  & \texttt{if (a <= b)} \\
    \midrule
    VoidMethodCall             & Removes void method calls                 & \texttt{foo(...)}    & \texttt{/* removed */} \\
    \bottomrule
  \end{tabular}
\end{table}

\section{Approach}
\label{sec:approach}

\subsection{Overview}
\label{sec:approach-overview}

processing pipeline:
CLEANUP\_PROJECT(0),
%
DOWNLOAD\_PROJECT(1),
SETUP\_PROJECT(2),
%
ADD\_DEPENDENCIES(3),
BUILD\_PROJECT\_ORIGINAL(4),
%
GENERATE\_EVOSUITE\_TESTS(5),
POSTPROCESS\_EVOSUITE\_TESTS(6),
%
BUILD\_SPOON\_MODEL(7),
%
EXECUTE\_TESTS\_ORIGINAL(8),
COLLECT\_JUNIT\_REPORTS\_ORIGINAL(9),
COLLECT\_JACOCO\_DATA\_ORIGINAL(10),
FILTER\_TESTS\_ORIGINAL(11),
COLLECT\_PIT\_DATA\_ORIGINAL(12),
%
ANALYZE\_TESTS(13),
FILTER\_TESTS(14),
FILTER\_ASSERTIONS(15),
%
ADD\_JPF\_INSTRUMENTATION(16),
BUILD\_PROJECT\_INSTRUMENTED(17),
EXECUTE\_JPF(18),
ANALYZE\_JPF(19),
CLEANUP\_JPF\_INSTRUMENTATION(20),
%
BUILD\_PROJECT\_INITIAL(21),
EXECUTE\_TESTS\_INITIAL(22),
%
COLLECT\_JUNIT\_REPORTS\_INITIAL(23),
COLLECT\_JACOCO\_DATA\_INITIAL(24),
COLLECT\_PIT\_DATA\_INITIAL(25),
%
CLEANUP\_GENERALIZATION(26),
%
GENERALIZE\_TESTS(27),
BUILD\_PROJECT\_GENERALIZED(28),
%
EXECUTE\_TESTS\_GENERALIZED(29),
COLLECT\_JUNIT\_REPORTS\_GENERALIZED(30),
FILTER\_GENERALIZATIONS(31),
%
COLLECT\_JACOCO\_DATA\_GENERALIZED(32),
COLLECT\_PIT\_DATA\_GENERALIZED(33);

basically:
- instrument project
- detect test methods + assertions + tested methods
- for each assertion / tested method, collect input + output specification with SPF
- for each assertion / tested method, create generalization with 3 variants (utilizing extracted specifications): BASELINE, NAIVE, IMPROVED

at various points, apply filtering to avoid processing of tests / assertions / generalizations that are unlikely to successfully pass all processing steps (or even guaranteed to fail)

thoughout the pipeline, track (i) runtime, (ii) (intermediate) results (success? failure? causes? other task outputs / created files?), (iii) mutation / coverage / test data;

offers a cleanup task to revert any changes applied by generalization (we only add files, so cleanup is just removing files; no existing files are modified);

\subsection{Project Instrumentation}
\label{sec:project-instrumentation}

(shorten + merge this with the overview? we have so many subsections...)

accepts either a URL to Git repository or a path to local directory as target;
if target is a URL to a Git repository, the project is cloned automatically (with Git's default settings);
processing then continues the same for both types of projects;

detects JUnit 4 vs. JUnit 5 testing framework, others not supported;
detects Maven vs. Gradle (Groovy DSL) projects, others not supported;
adds required dependencies based on detected project type (update JUnit, add JUnit Vintage, add JaCoCo, add PIT, add jqwik);
creates separate pom.xml / build.gradle (with comments for additions by Teralizer) - original file is left untouched;
to verify successful instrumentation, project is built, tests executed, test / coverage / mutation results collected;

\subsection{Test / Assertion Detection and Analysis}
\label{sec:test-analysis}

execute test suite;
identify executed test methods via junit / surefire XML reports;
for each test method, identify all assertions in the method via Spoon (calls to methods in "org.junit.Assert" (JUnit 4) or "org.junit.jupiter.api.Assertions" (JUnit 5)));
for each (supported) assertion (assertEquals, assertTrue, assertFalse, assertThrows), identify one tested method call via Spoon (for assertThrows: the executed method, for assertEquals/True/False: the method that returned the "actual" value of the assertion);
we assume each assertion "tests" one method, and do not consider side effects => no support for, e.g., method sequences that modify object state (e.g.: list.add(...), assert(..., list.size()));
side effects could be modeled as inputs/outputs of the tested method;

tests and assertions are filtered if they cannot be (correctly) handled by the current implementation. for details, see Section "Test / Assertion / Generalization Filtering".

\subsection{Specification Extraction}
\label{sec:specification-extraction}

\subsubsection{Driver Generation}
\label{sec:driver-generation}

for each assertion / tested method call, create a driver program and a corresponding SPF configuration;
the driver program is a single class with a "public static void main(...)" method;

the main method:
(i) executes any setup methods of the test class (JUnit 4: "@Before", "@BeforeClass", JUnit 5: "@BeforeAll", "@BeforeEach"),
(ii) creates an instance of the test class, and then
(iii) calls the tested method.

the SPF configuration:
(i) sets the main method of the driver program as the entrypoint for SPF execution,
(ii) sets the tested method as SPF's "symbolicMethod" (using symbolic inputs for generalizable input parameters),
(ii) configures SPF to run in constraint-collection mode (=> no constraint solving, just following the concrete execution path),
(iii) registers + configures a custom listener that extracts input-output specifications (see next Section "SPF Execution"),
(iv) configures several execution limits (depth limit, execution time limit, etc.; see Section "Other Limits / Safeguards").

(note: we had to modify (fix?) SPF to disable constraint solving during constraint-collection mode execution.)
(note: we're actually also creating an instrumented version of the test class / method, but that's only there so we can more easily identify the tested method call during SPF execution if the same method is called multiple times.)

\subsubsection{SPF Execution}
\label{sec:spf-execution}

execute SPF once for each driver program;
start tracking symbolic state when entering the tested method;
when exiting the tested method:
(i) write the concrete input values to a file,
(ii) write the concrete output values to a file,
(iii) write the symbolic input values to a file (=> path condition / input specification),
(iv) write the symbolic output values to a file (=> output specification),
then immediately terminate the execution (no need to keep going - we have everything we need);

((show an example of extracted data here))

for tested methods that exit via thrown exception,
use the thrown exception (type) as the concrete output value.
no symbolic output value can be collected in this case
(because the (type of the) thrown exception is not a function of the symbolic input values,
but is instead constant for all sets of inputs in the partition).

\subsection{Test Transformation}
\label{sec:test-transformation}


using jqwik (1.8.5) for property-based testing (\url{https://github.com/jqwik-team/jqwik});
last official release in 2024 (1.9.2);
last commit 2 days ago (checked on: 2005-04-23);
590 GitHub stars;
built for junit 5!;
comprehensive user guide (\url{https://jqwik.net/docs/current/user-guide.html});

competition 1:
junit-quickcheck (\url{https://github.com/pholser/junit-quickcheck});
last official release in 2020 (1.0);
last commit 8 months ago;
590 GitHub stars;
built for junit 4, junit 5 support only via junit-vintage (\url{https://github.com/pholser/junit-quickcheck/issues/189#issuecomment-414706607});
documentation less comprehensive (\url{https://pholser.github.io/junit-quickcheck/site/1.0/index.html});

competition 2:
quicktheories (\url{https://github.com/quicktheories/QuickTheories});
last official release in 2018 (0.25);
last commit 6 years ago;
509 GitHub stars;
built for junit 4;

\subsubsection{"BASELINE" Generalization}
\label{sec:baseline-generalization}

transforms target test into a property-based test;
one test class per generalizable assertion;
(PIT only offers class-level selections / exclusions, so generating classes causes less "collateral damage" for failing generalizations);
uses only the original set of input values via custom arbitrary;

allows us to see how much runtime overhead jqwik introduces even without any generation of input values;

transformation steps:
clone the original test class (all further actions on the cloned class);
delete other test methods in the class (non-test methods need to be preserved because they might be used by the target test);
add a nested class "TestParameters" that can hold values for all generalizable parameters of the tested method (i.e., all parameters of type byte, short, int, long, float, double);
add a nested class "TestParametersSupplier" that can generate "arbitrary" (jqwik term) "TestParameters" instances;
for the BASELINE variant, only one instance of TestParameters is generated by the supplier;
this instance uses the same tested method input values as the original test;
delete all existingTest annotations from the test method (removing @Test is most important, but other annotations are removed as well because they are unlikely to be compatible with @Property);
add jqwik @Property annotation (seed = 0, ShrinkingMode.OFF, EdgeCasesMode.FIRST, tries = 10 / 50 / 200);
add parameter of type "TestParameters \_p\_" with annotation \@ForAll(supplier = TestParametersSupplier) to the test method;
replace tested method arguments with values from TestParameters instance \_p\_ (e.g., foo(a, b, c) -> foo(\_p\_.a, \_p\_.b, \_p\_.c); only for generalizable inputs, the others remain unchanged).
delete other assertions in the test method (unless they have return values that are used in the code, e.g., Exception e = assertThrows(...));
no need to modify assertions (because inputs stay the same, so expected outputs should also stay the same);

the original test method is always preserved in the current implementation;
this would not be necessary for cases where there is only a single assertion in the test method and generalization is successful;
this would also not be necessary for casses where all assertions in a test method are successfully generalized;
statistics on how common these cases are in R1 (or RQ2? or RQ4?).

because most of the test method is copied for each generalized assertion, this creates a lot of duplicate code;
the currently implementation does not optimize for this at all - statistics on test suite size increases in RQ2;
some of this could likely be avoided by putting in more engineering effort, e.g., automatically extracting setup functions that can be reused across all generalizations of a test method;
alternatively, we could add multiple TestParameters parameters (one for each generalized assertion) - but that might not be very maintainable either;

\subsubsection{"NAIVE" Generalization}
\label{sec:naive-generalization}

same basic processing flow as BASELINE variant (clone class, add TestParameters + Supplier, remove other tests + assertions);
uses "naive" approach for selecting sets of input values;

basic approach:
step 1: randomly generate sets of values that match the types of input parameters;
step 2: apply filter to keep only value sets that satisfy the input specification;
repeat until desired number of input sets (we use 10, 50, 200 in the evaluation) has been generated (automatically done by jqwik, we just set how many we want);
still preserves original test inputs via a custom arbitrary => no reduction of mutation score due to "bad" random values;
beware that generalization does NOT change coverage - we only test additional inputs of already covered input partitions;

problem: many TooManyFilterMissesExceptions;
reason: depending on the input specification, randomly selecting sets of input values can be (very) unlikely to produce satisfying inputs (e.g., a = b = c => 3 random ints that are equal);

example: a = b = c (all ints);
randomly generate a;
randomly generate b;
randomly generate c;
apply the a = b = c filter (likely not a match => throw away and try again; after too many non-matching attempts => TooManyFilterMissesExceptions)

\subsubsection{"IMPROVED" Generalization}
\label{sec:improved-generalization}

same basic processing flow as BASELINE variant (clone class, add TestParameters + Supplier, remove other tests + assertions);
uses "improved" approach for selecting sets of input values to reduce TooManyFilterMissesExceptions;

basic approach:
step 1: generate sets of input values that already take into account "as many constraints as possible";
step 2: apply filter to keep only value sets that satisfy the full input specification;
repeat until the desired number of input sets has been generated;
like NAIVE, IMPROVED also preserves original test inputs via a custom arbitrary;
like NAIVE, IMPROVED also does NOT cover any previously uncovered input partitions;

example: a = b \&\& b = c (all ints);
randomly generate a since we don't have any constraints to consider yet;
generate b such that b = a (i.e., take into account the a = b constraint);
generate c such that b = c (i.e., take into account the b = c constraint);
apply the a = b = c filter (trivial in this case => use a more interesting example);

currently only considers the following constraints:
var1 == (var2 | const);
var1 < (var2 | const);
var1 <= (var2 | const);
var1 > (var2 | const);
var1 >= (var2 | const);

supported types: byte, short, int, long, float, double;
mixed-type constraints are also supported (e.g., int-var < float-var);

more complex terms (e.g., "compound" terms (is this the correct terminology?), (trigonometric) function calls) are not taken into account (e.g., a < b - c, a == cos(b));
constraints that are not equality, upper- or lower-bound constraints are not taken into account either (e.g., inequality constraints);
=> show some statistics about used vs. unused constraints -> further details in RQ4 or the discussion;

actual value selection logic (code that implements this logic for all generalizable inputs is automatically generated):

if at least one equality constraint exists for a variable, all other constraints are ignored
if multiple equality constraints exist, we just take "the first one";
all equality constraints have the same value anyway because we select these at runtime based on whichever concrete values have already been assigned to involved variables;

if multiple upper / lower bounds exist, the strongest bound is used, i.e., the highest lower bound and the lowest upper bound.
as with equality constraints, this is determined at runtime, i.e., based on on whichever values have already been assigned to involved variables;

in practice, naive processing of constraints often leads to "dead ends" where no further assignments are possible;
for example: a >= b \&\& b >= a; here we would have a ">= b" constraint on "a" and a ">= a" constraint on "b";
to resolve such situations, we assign an index to each variable that occurs in the input specification, e.g., idx(a)=1, idx(b)=2;
constraints are then rewritten to apply only to the variable with the highest index, e.g. (i) "a >= b" -> "b <= a", and (ii) "b >= a" -> "b >= a";
thus, the used constraints for variable value selection become: a -> no constraints, b -> {"<= a", ">= a"};
since a is now unconstrained, we can simply select a random value for it; once a is assined, b can be assigned as well; 
similar transformations are applied for all suppored constraints (==, <, <=, >, >=);
(todo: find the correct terminology to describe this; perhaps constraint rewriting / simplification?
other related terms: constraint satisfaction problems, variable elimination, constraint propagation, domain reduction, and arc consistency algorithms)

in some cases, choices of early variables lead to unsatisfiable constraints later on;
for example: b > a \&\& b <= 0 -> no satisfying assignment for b if a a >= 0;
in this cases, we return an empty arbitrary for b, thus prompting jqwik to pick a new value for a;
similar situations occur for over-/underflows, e.g., b > a with a = Integer.MAX\_VALUE;
in this case, b = a + 1 = Integer.MIN\_VALUE, which will be rejected by filtering;

some current limitations could be resolved through more engineering effort (e.g., custom arbitraries);
one way to generate values that satisfy all / more constraints would be through constraint solving (add references);
however, this would have a significant runtime cost and would still suffer various limitations (add references);

\subsection{Test / Assertion / Generalization Filtering}
\label{sec:all-filtering}

describe filtering here or in RQ4?

we need to ensure that:
1: we do not generate any incompilable code;
2: we have a green test suite for mutation testing (PIT only works with green suites - otherwise it throws an error);
also, we would like to avoid spending processing effort on generalization attempts that are unlikely to be successful (=> early excludes);
to achieve this, we apply filtering at multiple stages + levels in the processing pipeline (for filtering data, see RQ4).

(note: large reduction of required filtering would be possible by putting in more engineering effort)

\subsubsection{Test-level Filtering}
\label{sec:test-filtering}

filterTestOriginal:
NonPassingTestFilter: filters failing tests (PIT requires green suite, and generalization of failing tests does not seem useful anyway);
TestTypeFilter: filters tests with unsupported test annotations (we currently only support @Test annotations);

filterTest:
NoAssertionsFilter: filters tests without assertions (with no assertions, choosing a target method is even trickier than it already is, and we have no useful output specification);

\subsubsection{Assertion-level Filtering}
\label{sec:assertion-filtering}

filterAssertion:
ExcludedTestFilter: filters assertions that are part of filtered or otherwise excluded tests (if the tests cannot be handled, assertion-level results cannot override this);
MissingValueFilter: filters assertions for which no tested method could be identified (without a tested method, we don't have specifications, so cannot perform generalization);
VoidReturnTypeFilter: filters assertions with a tested method that has a void return type (no return type -> no output specification -> no generalization);
UnsupportedAssertionFilter: filters unsupported assertions (we currently only support assertEquals, assertTrue, assertFalse, and (with some caveats?) asserThrows);
ParameterTypeFilter: filters assertions with a tested method that has no parameters of supported types (we can currently generalize parameters of types byte, short, int, long, float, double);

\subsubsection{Generalization-level Filtering}
\label{sec:generalization-filtering}

filterGeneralization:
NonPassingTestFilter: filters generalizations that fail during test execution (PIT requires green suite, can be either due to "incorrect" generalization or due to "bad" original tests);

\subsubsection{Other Limits / Safeguards}
\label{sec:other-filtering}

During SPF execution:
maximum PC size limit;
maximum depth limit;
maximum execution time;

During Test Transformation:
maximum Java specification size;

\subsection{Program Output / Collected Data}
\label{sec:collected-data}

primarily, we provide 1 property-based test for each original assertion (excluding ones that are filtered throughout the processing pipeline).
(do we need this section? or are outputs / data already explained well enough in the other section? even then, might still be useful to have a condensed overview here)

additionally, we provide (i) a PostgreSQL database with data about the processed projects and (ii) various intermediate results / log files.
(perhaps move this information to some "Data Availability" section.)

The database contains tables:
- project,
- test,
- assertion,
- generaliztion,
- filter\_result,
- (evosuite\_runtime),
- (evosuite\_report),
- junit\_test\_report,
- jacoco\_coverage\_report,
- pit\_coverage\_report,
- pit\_mutation\_report,
- task.

The intermediate results / log files contain:
- modified pom.xml / build.gradle files,
- command-data (all executed commands + output logs + error logs),
- jacoco-data (JaCoCo coverage CSV reports per project + variant),
- jpf-data (output logs, input-output values + specifications, driver + config + instrumented test files),
- junit-data (JUnit / Surefire XML reports per project + variant),
- pit-data (PIT linecoverage + mutations XML reports per project variant),

using PIT for mutation testing because;
most mutation testing tools
(i) do not provide results in a structured format
that's suitable for automated processing
and / or (ii) do not provide (official) support for Java 8,
and / or (iii) are not actively maintained anymore
(according to the PIT website:
\url{https://pitest.org/java_mutation_testing_systems/#summary-of-mutation-testing-systems})
using DEFAULTS set of mutators (see Table~\ref{tab:pit-mutators})
for further details about mutators, see the PIT website \footnote{\url{https://pitest.org/quickstart/mutators/}}

\begin{table}[H]
  \caption{Mutators in \ToolPit{}'s DEFAULTS group of mutators.}
  \label{tab:pit-mutators}
  \begin{tabular}{l l l l}
    \toprule
    &&\multicolumn{2}{l}{Example} \\
    \cmidrule{3-4}
    Mutator & Description  & Before & After\\
    \midrule
    Math                       & Replaces arithmetic operations            & \texttt{x + y}       & \texttt{x - y} \\
    Increments                 & Replaces increment/decrement              & \texttt{i++}         & \texttt{i{-}{-}} \\
    InvertNegs                 & Inverts negation of variables             & \texttt{return -x}   & \texttt{return x} \\
    \midrule
    BooleanTrueReturnVals      & Returns \texttt{true} for booleans        & \texttt{return b}    & \texttt{return true} \\
    BooleanFalseReturnVals     & Returns \texttt{false} for booleans       & \texttt{return b}    & \texttt{return false} \\
    PrimitiveReturns           & Returns \texttt{0} for numeric primitives & \texttt{return a}    & \texttt{return 0} \\
    EmptyObjectReturnVals      & Returns empty for strings                 & \texttt{return s}    & \texttt{return ""} \\
    NullReturnVals             & Returns \texttt{null} for objects         & \texttt{return o}    & \texttt{return null} \\
    \midrule
    RemoveConditionalEqualElse & Forces else for equality checks           & \texttt{if (a == b)} & \texttt{if (false)} \\
    RemoveConditionalOrderElse & Forces else for inequality checks         & \texttt{if (a < b)}  & \texttt{if (false)} \\
    ConditionalsBoundary       & Changes boundary of inequalities          & \texttt{if (a < b)}  & \texttt{if (a <= b)} \\
    \midrule
    VoidMethodCall             & Removes void method calls                 & \texttt{foo(...)}    & \texttt{/* removed */} \\
    \bottomrule
  \end{tabular}
\end{table}

\section{Evaluation}
\label{sec:evaluation}

Test generalization strengthens the mutation scores of existing test suites
by exploring additional inputs within already-covered execution paths.
Our approach automates the transformation from
conventional unit tests to property-based tests, reducing the manual effort
required to perform such a transformation.
We empirically evaluate the potential and limitations of semantics-based test generalization
through five research questions:

\begin{itemize}
  \item \textbf{RQ1:} How much does test generalization improve mutation detection?
  \item \textbf{RQ2:} How does constraint complexity affect constraint-aware versus random input generation?
  \item \textbf{RQ3:} To which degree does generalization affect the size and runtime of the target test suites?
  \item \textbf{RQ4:} How efficient is test generalization compared to extended test generation?
  \item \textbf{RQ5:} What are the causes of unsuccessful generalization attempts?
\end{itemize}

These questions progress from measuring direct effects to understanding practical constraints.
RQ1 establishes effectiveness through mutation score improvements.
RQ2 explores how constraint complexity affects mutation score improvements achieved by the \VariantNaive{} and \VariantImproved{} generation variants.
RQ3 quantifies effects on test suite size and execution time.
RQ4 examines the runtime costs and efficiency of test generalization compared to extended test generation via \ToolEvoSuite{}.
RQ5 analyzes failure cases to identify current limitations and guide future improvements.
Our experimental setup (Section~\ref{sec:experimental-framework}) establishes the evaluation methodology
and our evaluation results (Sections~\ref{sec:primary-effects-eval}--\ref{sec:limitations-eval})
provide the necessary data to answer RQ1-RQ5.
All experiments were run on a MacBook Air
with M2 processor and 24~GB of memory using default JVM settings.
All collected data is available in our replication package~\cite{replicationpackage}.

\subsection{Experimental Setup}
\label{sec:experimental-framework}

To identify current capabilities and limitations
of semantics-based test generalization,
we systematically evaluate our implementation of \ToolTeralizer{}
across a multitude of projects which range from
controlled benchmark cases that are well-suited for test generalization
to real-world projects that demonstrate which future advances are needed
to improve practical applicability of test generalization tools.
In this section, we describe key components of our experimental setup
and establish a shared vocabulary
that we use throughout the evaluation
to refer to different stages of the processing pipeline,
different test suite variants,
and different groups of projects
that are part of our evaluation dataset.

\subsubsection{Processing Stages and Test Suite Variants}

As shown in Figure~\ref{fig:approach-overview},
\ToolTeralizer{}'s processing pipeline consists of five stages.
The first three
(test and assertion analysis, tested method identification, and specification extraction)
are \VariantShared{} stages that are only executed once per pipeline run
because their results can be reused across generalization strategies
(\VariantBaseline{}, \VariantNaive{}, and \VariantImproved{}).
Processing starts with all \VariantOriginal{} tests.
However, each stage can exclude tests
that are unsuitable for further processing.
We refer to the subset of \VariantOriginal{} tests that remain after
the \VariantShared{} processing stages
as the \VariantInitial{} test suite.
The last two processing stages
(generalized test creation and test suite reduction)
are executed once for the \VariantBaseline{} generalization strategy
and three times each for the \VariantNaive{} and \VariantImproved{} strategies
using different values for \ToolJqwik{}'s \tries{} setting.
Thus, we distinguish the following nine test suite variants:

\begin{itemize}
  \item \VariantOriginal{}: before any processing has taken place,
  \item \VariantInitial{}: after exclusions by \VariantShared{} processing stages,
  \item \VariantBaseline{}: after \VariantBaseline{} generalization and reduction,
  \item \VariantNaiveA{}, \VariantNaiveB{}, \VariantNaiveC{}: after \VariantNaive{} generalization and reduction (10/50/200~\tries{}),
  \item \VariantImprovedA{}, \VariantImprovedB{}, \VariantImprovedC{}: after \VariantImproved{} generalization and reduction (10/50/200~\tries{}).
\end{itemize}

As described in Section~\ref{sec:three-variant-design},
using \VariantInitial{} as a shared starting point
not only reduces the runtime costs of the evaluation
but also enables a fair comparison across strategies
by avoiding non-deterministic Stage 1--3 failures such as \texttt{OutOfMemoryError}s
from affecting one strategy more strongly than another.
The used \tries{} settings of 10, 50, and 200 were selected
to demonstrate the scaling behavior of higher \tries{}
%(both in terms of mutation score improvements as well as in terms of runtime costs)
while keeping runtime costs manageable.

\subsubsection{Evaluated Projects}

Our evaluation employs three complementary datasets that progressively
reveal the gap between controlled and real-world conditions for test generalization.
The \DatasetEqBench{} benchmark~\cite{badihi_2021_eqbench}
provides numeric-focused programs
that are well-suited for symbolic analysis.
Utility methods extracted from Apache Commons projects
bridge toward real-world complexity while maintaining the focus on numeric constraints.
Projects from the RepoReapers dataset~\cite{munaiah_2017_reporeapers}
expose real-world applicability challenges.
Table~\ref{tab:dataset-statistics} provides descriptive statistics of the datasets.
For the implementation, we show the
number of files, classes, and source lines of code (SLOC).
For tests, we additionally provide the number of test methods,
i.e., methods that are annotated with \texttt{@Test}, \texttt{@RepeatedTest}, or \texttt{@ParameterizedTest}.

\textit{EqBench.}
The \DatasetEqBench{} benchmark~\cite{badihi_2021_eqbench}
(rows \DatasetsEqBenchEs{} in Table~\ref{tab:dataset-statistics})
provides controlled conditions
for automated test generalization.
Originally designed for equivalence checking research,
its 652 Java classes implement equivalent and non-equivalent program pairs
focusing on numeric computations while deliberately avoiding
features that complicate automated reasoning (e.g., recursion, reflection, and complex object graphs).
Since \DatasetEqBench{} provides only implementation code without tests,
we generated test suites using \ToolEvoSuite{}~\cite{fraser_2011_evosuite}
with three different search budgets (1s, 10s, and 60s per implementation class),
creating the dataset variants \DatasetEqBenchA{}, \DatasetEqBenchB{}, and \DatasetEqBenchC{}.
This design explores how initial test suite quality affects generalization effectiveness:
stronger initial suites offer better test diversity but less improvement potential
due to their higher initial mutation scores.

\textit{Apache Commons.}
To bridge toward real-world complexity,
we extracted numeric utility methods from Apache Commons projects
(rows \DatasetsCommons{} in Table~\ref{tab:dataset-statistics}).
Using Sourcegraph's code search~\cite{sourcegraph},
we identified public static methods with numeric or boolean parameters and return values ---
the types currently supported by \ToolTeralizer{}
(search queries are available in our replication package~\cite{replicationpackage}).
This yielded 247 classes from 17 Apache Commons projects
(commons-math, commons-numbers, commons-lang, etc.),
including all transitively called methods and dependencies
to ensure compilation (19,709 LOC total).
From this, we created four dataset variants:
\DatasetCommonsA{}, \DatasetCommonsB{}, and \DatasetCommonsC{}
use \ToolEvoSuite{}-generated tests
with 1s, 10s, and 60s per-class search budgets.
In contrast, \DatasetCommonsDev{} preserves the 725 original developer-written tests (14,389 LOC).
This enables direct comparison of generalization effectiveness between
developer-written tests and tests generated by \ToolEvoSuite{}.

\textit{RepoReapers.}
To understand current limitations in practical settings,
we selected 632 projects from RepoReapers~\cite{munaiah_2017_reporeapers},
a curated collection of 1.9 million GitHub repositories
specifically filtered for their use of sound software engineering practices
(e.g., extensive development history,
use of software testing and issue tracking,
availability of documentation).
Our selection criteria balanced technical constraints with evaluation goals.
All selected projects target Java 5--8 (for \ToolSPF{} compatibility),
use JUnit 4 or 5 through Maven (for \ToolTeralizer{} compatibility),
have standard directory structures (for automated processing),
medium-sized codebases (5,000--50,000 LOC),
and substantial test suites (20--80\% of total code).
The selected projects collectively comprise 50,474 implementation classes 
and 30,894 test classes across diverse domains and coding styles.
While \ToolTeralizer{} succeeds on \DatasetEqBench{} and partially on Apache Commons
(RQ1--RQ5, Sections~\ref{sec:primary-effects-eval}--\ref{sec:limitations-eval}),
the RepoReapers projects expose current barriers to practical applicability
(RQ6, Section~\ref{sec:limitations-eval-extended}).

\begin{table}
  \caption{Number of files, classes, source lines of code (SLOC), and test methods per project.}
  \label{tab:dataset-statistics}
  \begin{tabular}{lrrrrrrr}
    \toprule
    & \multicolumn{3}{r}{Implementation} & \multicolumn{4}{r}{Test} \\
    \cmidrule(lr){2-4} \cmidrule(lr){5-8}
    Project & Files & Classes & SLOC & Files & Classes & SLOC & Methods \\
    \midrule
    \DatasetEqBenchA{} & 544 & 652 & 27,871 & 544 & 544 & 35,666 & 4,718 \\
    \DatasetEqBenchB{} & 544 & 652 & 27,871 & 543 & 543 & 36,937 & 4,875 \\
    \DatasetEqBenchC{} & 544 & 652 & 27,871 & 544 & 544 & 37,836 & 4,974 \\
    \midrule
    \DatasetCommonsA{} & 106 & 247 & 19,709 & 103 & 103 & 17,524 & 2,481 \\
    \DatasetCommonsB{} & 106 & 247 & 19,709 & 103 & 103 & 19,082 & 2,738 \\
    \DatasetCommonsC{} & 106 & 247 & 19,709 & 102 & 102 & 18,839 & 2,735 \\
    \midrule
    \DatasetCommonsDev{} & 106 & 247 & 19,709 & 80 & 119 & 14,389 & 725 \\
    \midrule
    \DatasetRepoReapers{} (total) & 41,292 & 50,474 & 2,735,127 & 22,281 & 30,894 & 2,012,601 & 81,810 \\
    \DatasetRepoReapers{} (mean) & 65 & 79 & 4,320 & 35 & 48 & 3,179 & 162 \\
    \DatasetRepoReapers{} (median) & 49 & 56 & 3,253 & 23 & 26 & 2,107 & 86 \\
    \bottomrule
  \end{tabular}
\end{table}

\subsection{RQ1: How much does test generalization improve mutation detection?}
\label{sec:primary-effects-eval}

Mutation testing provides a rigorous measure of test effectiveness
by evaluating a test suite's ability to detect deliberately introduced faults.
For \ToolTeralizer{}, mutation scores reveal whether testing additional inputs
within existing execution paths achieves the intended improvement in fault detection capabilities.
%Section~\ref{sec:included-mutants} first establishes which tests and mutants are included in the evaluation,
%revealing systematic differences between generated and developer-written test suites that influence generalization outcomes.
Section~\ref{sec:overall-detection-rates} quantifies detection improvements
across projects and variants,
demonstrating that generalization improves mutation scores
for \DatasetsEqBenchEs{} and \DatasetsCommons{} projects, albeit to different degrees.
Section~\ref{sec:detection-rates-per-mutator} dissects these improvements by mutation operator,
uncovering how much detection of different mutants benefits from generalization.
% Finally, Section~\ref{sec:boundary-detection-effectiveness} provides a
% detailed analysis of the \VariantImproved{} generalization variant,
% demonstrating that its effectiveness correlates with constraint complexity,
% i.e., projects with more complex input constraints show greater improvements
% from constraint-aware input generation.

% \subsubsection{Included Mutants}
% \label{sec:included-mutants}

% To evaluate mutation detection effects,
% we compare \ToolPit{} results for \VariantInitial{} variants
% against \VariantNaive{} and \VariantImproved{} variants
% using \ToolPit{}'s \texttt{DEFAULTS} group of mutation operators.
% This comparison isolates generalization effects by using \VariantInitial{} as the baseline,
% ensuring that differences stem solely from the presence or absence of generalized test cases.
% Comparison to \VariantOriginal{} would not isolate generalization effects
% because the original test suite contains tests excluded during
% test analysis and specification extraction (Sections~\ref{sec:test-assertion-analysis}--\ref{sec:specification-extraction}),
% i.e., before any transformation to property-based tests occurs.

% Test inclusion rates reveal fundamental differences between generated and developer-written test suites
% that shape generalization outcomes.
% EvoSuite-generated projects achieve 83.4--85.1\% inclusion rates across variants,
% while \DatasetCommonsDev{} includes only 63.6\% of original test methods
% (Table~\ref{tab:mutants-per-project}).
% This 20-percentage-point gap reflects how different test creation approaches
% align with \ToolTeralizer{}'s current processing capabilities.

% \begin{table}[H]
  \caption{Number of total, covered, and uncovered mutants in included classes per project.}
  \label{tab:mutants-per-project}
  \begin{tabular}{lrrrrr}
    \toprule
    & Included     & Included      & \multicolumn{3}{r}{Mutants} \\
                                             \cmidrule(lr){4-6}
    Project & Test Methods & Impl. Classes & Total & Covered & Uncovered \\
    \midrule
    \DatasetEqBenchA{} & 3,937\; (83.4\%) & 607\; (93.1\%) & 23,905 & 21,492\; (89.9\%) & 2,413\; (10.1\%) \\
    \DatasetEqBenchB{} & 4,049\; (83.1\%) & 600\; (92.0\%) & 23,654 & 21,657\; (91.6\%) & 1,997\; (\phantom{0}8.4\%) \\
    \DatasetEqBenchC{} & 4,124\; (82.9\%) & 603\; (92.5\%) & 23,663 & 22,010\; (93.0\%) & 1,653\; (\phantom{0}7.0\%) \\
    \midrule
    \DatasetCommonsA{} & 2,079\; (83.8\%) & 111\; (44.9\%) & 8,581 & 7,536\; (87.8\%) & 1,045\; (12.2\%) \\
    \DatasetCommonsB{} & 2,330\; (85.1\%) & 112\; (45.3\%) & 8,391 & 7,939\; (94.6\%) & 452\; (\phantom{0}5.4\%) \\
    \DatasetCommonsC{} & 2,326\; (85.0\%) & 112\; (45.3\%) & 8,354 & 8,109\; (97.1\%) & 245\; (\phantom{0}2.9\%) \\
    \midrule
    \DatasetCommonsDev{} & 461\; (63.6\%) & 90\; (36.4\%) & 8,096 & 5,215\; (64.4\%) & 2,881\; (35.6\%) \\
    \bottomrule
  \end{tabular}
\end{table}

% EvoSuite generates tests with characteristics that facilitate automated analysis:
% simple assertion patterns that compare single values,
% explicit parameter passing without complex setup logic,
% and method-level focus that isolates individual behaviors.
% Developer-written tests, conversely, combine multiple scenarios within single methods,
% employ elaborate fixture setup spanning multiple classes,
% and validate system state changes through sequences of operations.
% While these patterns enable comprehensive integration testing,
% they create barriers for specification extraction, explaining
% why over one-third of developer tests cannot be generalized.

% The three primary exclusion causes reflect current limitations in our prototype:
% (i)~tests without assertions provide no specification to generalize,
% (ii)~tests where no tested method can be identified typically involve
% program constructs that our simple static analysis implementation cannot resolve,
% such as loops in test code or modifications of object state rather than direct return values,
% and (iii)~tests exercising methods without generalizable parameters have no numeric
% or boolean inputs to vary.
% Section~\ref{sec:filtering-eval} quantifies how frequently all exclusion causes occur,
% providing guidance for extending generalization capabilities.

% Mutation coverage patterns highlight further differences across the project environments.
% \ToolPit{} generates approximately 8,000 total mutants for Apache Commons projects
% and 23,000--24,000 for EqBench projects (Table~\ref{tab:mutants-per-project}).
% Covered mutants, i.e., those mutants that are executed by at least one included test, range
% from 64.4\% for \DatasetCommonsDev{} to 97.1\% for \DatasetCommonsC{}.
% This coverage disparity primarily reflects dataset construction rather than test quality:
% \DatasetCommonsDev{} includes only tests directly covering the extracted utility methods
% (but no tests for transitively called methods),
% while EvoSuite variants generate comprehensive suites for all included code.
% This focused scope reflects our intent to evaluate code amenable to generalization in this part of the dataset,
% as discussed in Section~\ref{sec:apache-commons-utils}.

% With the evaluation scope established, we now analyze how effectively
% the generalized tests detect mutations within the included subset.

\subsubsection{Overall Mutation Detection Rates}
\label{sec:overall-detection-rates}

Generalization improves mutation detection across all
\DatasetsEqBenchEs{} and \DatasetsCommons{} projects,
though the degree of improvement varies by project type
(Figure~\ref{fig:mutation-detection-results}).
\DatasetsEqBenchEs{} projects (rows 1--3 in the figure) show the largest improvements:
detection rates increase from 48.1--51.6\% to 49.5--55.0\%
across different \tries{} and generalization strategies,
representing absolute improvements of 1.2--3.9 percentage points
(2.4--8.2\% relative increase).
\DatasetsCommonsEs{} projects (rows 4--6) improve by 0.82--1.33 percentage points
(1.4--2.3\% relative increase),
while \DatasetCommonsDev{} (row 7) improves by only 0.05--0.07 percentage points
from its 80.4\% baseline.

\begin{figure}
  \centering
  \includegraphics[width=\linewidth]{figures/fig_mutation_detection_comparison}
  \caption{Percentage of detected mutants (left side) and improvement over INITIAL (right side) per project and generalization strategy.
  Improvements show both the absolute improvement (top value) as well as the relative improvement (bottom value).}
  \Description{Multi-panel bar chart showing mutation detection results.
  Left column displays detected percentages for seven projects:
  eqbench-es-default-1s, -10s, -60s, commons-utils-es-default-1s, -10s, -60s, and commons-utils.
  Each panel shows bars for INITIAL (blue), NAIVE variants (orange), and IMPROVED variants (green).
  Variants have subscripts (10, 50, 200) indicating tries parameter.
  Y-axis ranges from approximately 48% to 81% detection rate.
  Right column shows corresponding improvement percentages over INITIAL baseline,
  with values in parentheses and +/- indicators.
  Improvements range from minimal (0.05%) for commons-utils developer tests
  to substantial (7.82%) for eqbench-es-default-1s.
  Color legend at top: INITIAL (blue), NAIVE Variants (orange), IMPROVED Variants (green).}
  \label{fig:mutation-detection-results}
\end{figure}

In general, two factors strongly affect achievable generalization improvements:
initial test suite strength and input constraint complexity.
\DatasetsEqBenchEs{} projects offer the most suitable conditions for generalization.
This is because EvoSuite-generated tests leave more room for enhancement
(starting from 48.1--51.6\% detection rates)
than the thorough \DatasetCommonsDev{} test suite,
and the \DatasetEqBench{} benchmark's input constraints
are simpler than the input constraints of \DatasetsCommons{},
thus making valid input generation easier
(as discussed in more detail in Section~\ref{sec:constraint-complexity-eval}).
\DatasetsCommonsEs{} projects face one additional challenge:
while they also start from EvoSuite-generated tests,
the \DatasetsCommons{} methods involve more complex input specifications,
making it harder to generate inputs that satisfy input constraints.
\DatasetCommonsDev{} faces both challenges:
the mature developer-written test suite already achieves 80.4\% detection
through years of refinement of these widely-used libraries,
and it shares the same complex constraint characteristics as other \DatasetsCommons{} variants.
This leaves little opportunity for automated improvement.

Comparing the relative effectiveness of \VariantNaive{} and \VariantImproved{}
shows opposite results across \DatasetsEqBenchEs{} and \DatasetsCommonsEs{} projects.
On \DatasetsEqBenchEs{} projects (rows 1--3), \VariantNaive{} (orange bars)
outperforms \VariantImproved{} (green bars) for all \tries{} settings,
with gaps ranging from 0.17--1.21 percentage points.
This advantage is most pronounced with limited \tries{}:
\VariantNaiveA{} achieves 50.67--53.97\% detection rate
while \VariantImprovedA{} reaches only 49.46--52.86\%.
In contrast, \DatasetsCommonsEs{} projects (rows 4--6) show \VariantImproved{}
outperforming \VariantNaive{} in 7 of 9 comparisons,
with \VariantImprovedC{} detecting 57.97--59.45\% of mutants
compared to \VariantNaiveC{}'s 58.00--59.32\%.
These opposite results reflect the following patterns:
\VariantNaive{}'s better performance on \texttt{Math} mutations (59.1\% of all mutants)
versus \VariantImproved{}'s better detection of most other mutation types
as well as more effective handling of complex constraints.
In \DatasetsEqBenchEs{}, the high prevalence of \texttt{Math} mutations
and low constraint complexity favor \VariantNaive{},
while in \DatasetsCommonsEs{}, \VariantImproved{}'s
advantages on other mutations and constraint handling overcome
its \texttt{Math} mutant detection disadvantage.
We investigate the mechanisms behind these results in further detail in
Sections~\ref{sec:detection-rates-per-mutator} and \ref{sec:constraint-complexity-eval}.
\DatasetCommonsDev{} shows no meaningful differences between approaches,
with all variants achieving approximately 80.4\% detection.

Higher \tries{} settings showed improved detection rates with diminishing returns.
The increase from 10 to 50 \tries{} typically produces larger gains
than increasing from 50 to 200 \tries{}.
One notable exception to this pattern appears in \VariantImproved{} variants with only 10 \tries{}:
on \DatasetsEqBenchEs{} projects (rows 1--3),
\VariantImprovedA{} achieves only 2.4--2.8\% relative improvement
versus 5.6--6.9\% for \VariantImprovedB{} and 6.0--7.8\% for \VariantImprovedC{}.
This comparatively low increase in detection rates stems from boundary-focused generation
consuming most of the limited \tries{} of this variant,
leaving insufficient attempts for testing intermediate values.
With more \tries{}, \VariantImproved{} variants perform comparably to or better than \VariantNaive{} variants
as enough attempts remain for both boundary and non-boundary testing.
\DatasetsCommonsEs{} projects (rows 4--6) show less pronounced degradation at low \tries{},
because their more complex constraints (Section~\ref{sec:constraint-complexity-eval})
cause more boundary inputs to fail filtering,
forcing earlier exploration of non-boundary values.

Combining short test generation with subsequent generalization
can outperform longer generation alone.
On \DatasetsEqBenchEs{} projects,
1-second \ToolEvoSuite{} generation followed by \VariantNaiveC{} generalization
achieves 52.0\% detection (last orange bar in row 1),
surpassing 60-second generation alone (51.6\%, blue bar in row 3).
Similarly on \DatasetsCommonsEs{},
10-second generation plus \VariantNaiveC{} reaches 58.5\% (last orange bar in row 5)
versus 58.1\% for 60-second generation (blue bar in row 6).
These comparisons demonstrate that generalization can effectively compensate for reduced generation time.
Section~\ref{sec:execution-efficiency} analyzes the efficiency trade-offs
between different combinations of \ToolEvoSuite{} timeouts and \ToolTeralizer{} \tries{}
in further detail through Pareto front analysis.

% The reported improvements are based on single runs per project configuration.
% While property-based test generation includes randomness that could produce variation across runs,
% the consistent patterns across projects suggest the observed trends are robust.
% Multiple runs would strengthen confidence in the precise magnitude of improvements
% (a limitation we discuss in Section~\ref{sec:threats-to-validity}).

\subsubsection{Mutation Detection Rates per Mutator}
\label{sec:detection-rates-per-mutator}

Generalization effectiveness varies significantly across mutation operators,
revealing which fault types benefit most from additional test inputs
(Table~\ref{tab:detections-per-mutator}).
The \texttt{Math} mutation operator dominates the mutation landscape
in \DatasetsEqBenchEs{} and \DatasetsCommons{} projects,
making up 59.1\% of total mutants.
This strong representation of \texttt{Math} mutations
is primarly due to the focus on numeric computations in these datasets.
The next most commonly occurring mutations are 
\texttt{Conditionals\-Boundary} and \texttt{Remove\-Conditional\-Order\-Else}
mutations, each of which accounts for 11.0\% of total mutants
(both operators are applied at the same source code locations).
In contrast, the least common operators, i.e., \texttt{Increments} (0.52\%),
\texttt{Boolean\-False\-ReturnVals} (0.24\%),
and \texttt{Empty\-Object\-Return\-Vals} (0.13\%), together
account for less than 1\% of all mutants.
This large difference in mutant prevalence means that
improvements to \texttt{Math} detection rates
have proportionally larger impact on overall mutation scores
than improvements to less commonly occurring mutation types.

\begin{table}[H]
  \caption{Number of mutants and percentage of detections per mutator.}
  \label{tab:detections-per-mutator}
  \begin{tabular}{lrrrrcrrrr}
    \toprule
    & & & & & \multicolumn{5}{c}{Detected \%} \\
    \cmidrule{6-10}
    Mutator & Total & Total \% & Min \% & Max \% & INITIAL & \multicolumn{2}{c}{NAIVE$_{200}$} & \multicolumn{2}{c}{IMPROVED$_{200}$} \\
    \midrule
    Math & 61841 & 59.10 & 52.34 & 62.16 & 50.99 & 54.98 & (+3.99) & 54.36 & (+3.37) \\
    ConditionalsBoundary & 11501 & 10.99 & 8.50 & 11.94 & 27.68 & 28.89 & (+1.21) & 30.23 & (+2.55) \\
    RemoveConditionalOrderElse & 11501 & 10.99 & 8.50 & 11.94 & 61.08 & 62.29 & (+1.21) & 62.47 & (+1.39) \\
    PrimitiveReturns & 7731 & 7.39 & 6.15 & 10.09 & 89.42 & 89.63 & (+0.20) & 89.90 & (+0.47) \\
    RemoveConditionalEqualElse & 5536 & 5.29 & 3.21 & 10.40 & 58.80 & 60.87 & (+2.07) & 61.00 & (+2.20) \\
    InvertNegs & 3122 & 2.98 & 2.94 & 3.12 & 58.91 & 60.61 & (+1.70) & 60.99 & (+2.08) \\
    VoidMethodCall & 973 & 0.93 & 0.58 & 1.35 & 24.96 & 24.96 & -- & 25.49 & (+0.53) \\
    NullReturnVals & 933 & 0.89 & 2.13 & 3.38 & 98.77 & 98.77 & -- & 98.77 & -- \\
    BooleanTrueReturnVals & 569 & 0.54 & 0.17 & 1.44 & 98.55 & 98.55 & -- & 98.55 & -- \\
    Increments & 546 & 0.52 & 0.50 & 0.54 & 72.81 & 73.38 & (+0.57) & 73.50 & (+0.69) \\
    BooleanFalseReturnVals & 250 & 0.24 & 0.09 & 0.63 & 87.87 & 87.87 & -- & 87.87 & -- \\
    EmptyObjectReturnVals & 141 & 0.13 & 0.41 & 0.43 & 90.30 & 90.30 & -- & 90.30 & -- \\
    \bottomrule
  \end{tabular}
\end{table}

\VariantInitial{} detection rates reveal a clear pattern: return value mutations
(i.e., \texttt{Primitive\-Returns}, \texttt{Boolean\-True\-Return\-Vals},
\texttt{Boolean\-False\-Return\-Vals}, and \texttt{Empty\-Object\-Return\-Vals})
are caught in the large majority of cases (87.9--98.8\% detection),
while behavioral mutations prove more elusive.
For example, \texttt{VoidMethodCall} mutations achieve a detection rate of only 25.0\%
because removing void method calls typically affects only internal state
or produces side effects that are rarely verified by existing assertions.
Because these cases often require new assertions to be added to achieve detection rate improvements,
they are largely beyond the current capabilities of \ToolTeralizer{}.
In contrast, \VariantInitial{} detection rates of
27.7\% for \texttt{ConditionalsBoundary} mutations,
61.1\% for \texttt{RemoveConditionalOrderElse},
and 58.8\% for \texttt{RemoveConditionalEqualElse}
do highlight opportunities for automated improvement.
After all, all three mutations
affect partition boundaries,
which is exactly where \VariantImproved{} generalization
aims to generate additional test inputs.

\VariantNaiveC{} achieves its largest improvements on
\texttt{Math} (4.0~percentage points),
\texttt{RemoveConditionalEqualElse} (2.1~percentage points),
and \texttt{InvertNegs} (1.7~percentage points).
The success with \texttt{RemoveConditionalEqualElse} stems from
testing diverse inputs that trigger both branches of equality checks,
exposing mutations that force false branches.
\texttt{Math} mutation detection similarly benefits from a more diverse
range of tested input values because arithmetic operations often produce
different results across input ranges, exposing mutations that might
coincidentally produce correct results for any individual value.
In contrast, 4 of 5 return value mutators show zero improvement:
\texttt{Null\-Return\-Vals}, \texttt{Boolean\-True\-Return\-Vals},
\texttt{Boolean\-False\-ReturnVals}, and \texttt{EmptyObjectReturnVals}.
These already achieve high detection rates in the \VariantInitial{} variant (87.9--98.8\%)
because return value mutations often directly violate test assertions,
leaving little room for improvement through input variation.
Increasing detection rates beyond this high starting point
would likely require additional assertions to be introduced,
rather than test inputs to be varied.

\VariantImprovedC{} demonstrates different strengths than \VariantNaive{}
through constraint-aware input generation.
Comparing the two variants across all 12 mutation operators:
\VariantImprovedC{} outperforms \VariantNaiveC{} for 7 operators,
achieves the same detection rate for 4 operators (the zero-improvement return value mutations),
and underperforms for only 1 operator (\texttt{Math}).
The largest advantage is observed for \texttt{Conditionals\-Boundary} detection,
where \VariantImprovedC{} achieves 2.5 percentage points improvement
compared to \VariantNaiveC{}'s 1.2 percentage points.
These results confirm that the constraint-aware input generation strategy
followed by \VariantImproved{} variants achieves its intended purpose.
Furthermore, the slightly higher detection rates for several other mutators
suggest that testing at partition boundaries may also provide slight benefits
for some mutators that do not directly modify partition boundaries.

The \texttt{Math} mutation results highlight the inherent trade-off
in boundary versus non-boundary testing:
\VariantImprovedC{} achieves 3.4 percentage points improvement
compared to \VariantNaiveC{}'s 4.0 percentage points.
The \VariantImprovedC{} variant achieves smaller improvements here
because constraint-aware input generation produces less diverse arithmetic inputs,
concentrating on boundary values rather than exploring the full numeric range
where arithmetic mutations might create more varied outputs.
Given that \texttt{Math} mutations comprise 59.1\% of all mutants,
this difference significantly impacts overall mutation scores.
To counteract these detrimental effects,
\VariantImproved{} variants could use higher \tries{} settings
to maintain non-boundary input coverage of \VariantNaive{}
while adding additional boundary coverage.
Alternatively, more sophisticated input selection strategies
could be used to achieve a better balance between
boundary and non-boundary testing even at lower \tries{} settings,
thus avoiding the runtime cost of higher \tries{}.

Results for the 10 / 50 \tries{} variants of \VariantNaive{} and \VariantImproved{}
generally follow the same trends as those for the listed variants with 200 \tries{},
albeit with correspondingly smaller detection rate improvements over \VariantInitial{}.
For example, \VariantImprovedA{} achieves expected results for most mutation operators,
including its characteristic advantage for \texttt{ConditionalsBoundary} mutations (2.0\% vs 0.9\% for \VariantNaiveA{}).
The exception is \texttt{Math} mutation detection,
where \VariantImprovedA{} achieves only 1.1\% improvement compared to 2.8\% for \VariantNaiveA{}.
Given that \texttt{Math} mutations comprise 59.1\% of all mutants,
this explains the low overall detection rate of \VariantImprovedA{} on \DatasetsEqBenchEs{} projects
observed in rows 1--3 of Figure~\ref{fig:mutation-detection-results}.
With 50 \tries{}, the \texttt{Math} detection gap is noticeably smaller
(\VariantImprovedB{} achieves 3.1\% versus \VariantNaiveB{}'s 3.6\%),
confirming that limited \tries{} constrain arithmetic diversity
only when boundary testing consumes most attempts.
Full results for all variants are available in our replication package~\cite{replicationpackage}.

\rqanswerbox{1}{TODO}

\subsection{RQ2: How does constraint complexity affect random versus constraint-aware input generation?}
\label{sec:constraint-complexity-eval}

As discussed in Section~\ref{sec:overall-detection-rates},
\VariantNaive{} outperforms \VariantImproved{} on \DatasetsEqBenchEs{} projects.
However, \DatasetsCommonsEs{} projects show \VariantImproved{}
outperforming \VariantNaive{} in 7 of 9 cases.
To better understand these contrasting results, RQ2 examines
how constraint complexity differs across projects and
how it affects \VariantNaive{} versus \VariantImproved{}.
As described in Section~\ref{sec:constraint-encoding},
\VariantImproved{} tests
encode simple in-/equalities on numeric and boolean variables or constants
during input value generation.
More complex constraints are not encoded
during \VariantImproved{} input generation
--- and no constraints are encoded by \VariantNaive{} ---
but are still enforced during input filtering
which takes place after input generation.

Table~\ref{tab:mutation-detection-comparison} shows the model properties
of mutants that are (not) detected by \ToolTeralizer{}'s \VariantImprovedC{} generalization variant.
Models represent the constraints that inputs must satisfy
to reach each mutant along a specific execution path.
We measure model complexity through operation count (total operators)
and constraint count (individual boolean conditions),
while tracking which percentage of constraints \VariantImproved{} can encode
during input generation versus enforce through post-generation filtering.
For instance, consider \texttt{(((a < 0) \&\& (a == (b + 1))) \&\& c)}.
This model contains three constraints:
\texttt{a < 0}, \texttt{a == (b + 1)}, and~\texttt{c}.
\VariantImproved{} encodes the simple comparison \texttt{a < 0} and the boolean variable \texttt{c}
in the created input value generation code,
but encodes \texttt{a~==~(b~+~1)} only in the input value filtering code
because it contains the compound term \texttt{b + 1}.
Thus, \VariantImproved{} uses 2 of 3 total constraints for input value generation (66.7\% utilization),
and the model contains 5 operators:
\texttt{<}, \texttt{\&\&}, \texttt{==}, \texttt{+}, and another \texttt{\&\&}.

\begin{table}
  \caption{Model properties of mutants that are (not) detected by the \VariantImprovedC{} variant.}
  \label{tab:mutation-detection-comparison}
  \begin{tabular}{lcrrrrrrr}
    \toprule
    & & & \multicolumn{2}{c}{Operations} & \multicolumn{2}{c}{Constraints} & \multicolumn{2}{c}{Constraints Used} \\
    \cmidrule(lr){4-5} \cmidrule(lr){6-7} \cmidrule(lr){8-9}
    Project & Detected & Mutants & Mean & Median & Mean & Median & Mean & Median \\
    \midrule
    \DatasetEqBenchA{} & yes & 11,145 & 147 & \phantom{0}9 & \phantom{0}6 & 2 & \phantom{0}47\% & \phantom{0}80\% \\
    \DatasetEqBenchA{} & no & 10,347 & 224 & 16 & 11 & 5 & \phantom{0}23\% & \phantom{0}50\% \\
    \DatasetEqBenchB{} & yes & 11,658 & 139 & \phantom{0}9 & \phantom{0}6 & 2 & \phantom{0}62\% & 100\% \\
    \DatasetEqBenchB{} & no & 9,999 & 231 & 15 & \phantom{0}8 & 2 & \phantom{0}57\% & 100\% \\
    \DatasetEqBenchC{} & yes & 12,052 & 137 & \phantom{0}9 & \phantom{0}5 & 2 & \phantom{0}69\% & 100\% \\
    \DatasetEqBenchC{} & no & 9,958 & 218 & 11 & \phantom{0}6 & 2 & \phantom{0}67\% & 100\% \\
    \midrule
    \DatasetCommonsA{} & yes & 4,390 & 290 & 15 & \phantom{0}7 & 5 & \phantom{0}43\% & \phantom{0}84\% \\
    \DatasetCommonsA{} & no & 3,183 & 389 & 45 & 12 & 6 & \phantom{0}11\% & \phantom{0}50\% \\
    \DatasetCommonsB{} & yes & 4,660 & 467 & 23 & \phantom{0}6 & 5 & \phantom{0}46\% & \phantom{0}85\% \\
    \DatasetCommonsB{} & no & 3,309 & 507 & 46 & \phantom{0}8 & 6 & \phantom{0}10\% & \phantom{0}56\% \\
    \DatasetCommonsC{} & yes & 4,821 & 374 & 20 & \phantom{0}6 & 5 & \phantom{0}47\% & \phantom{0}85\% \\
    \DatasetCommonsC{} & no & 3,288 & 423 & 41 & 10 & 6 & \phantom{0}11\% & \phantom{0}54\% \\
    \midrule
    \DatasetCommonsDev{} & yes & 4,193 & 107 & 11 & \phantom{0}4 & 4 & \phantom{0}25\% & \phantom{0}75\% \\
    \DatasetCommonsDev{} & no & 1,022 & 173 & 10 & \phantom{0}4 & 4 & \phantom{0}19\% & \phantom{0}75\% \\
    \bottomrule
  \end{tabular}
\end{table}

Undetected mutants have more complex models than detected ones
across all evaluated projects.
Operation counts for undetected mutants are 1.2--3$\times$ higher:
\DatasetsEqBenchEs{} projects show mean counts of
218--231 operations for undetected mutants
versus 138--147 operations for detected mutants,
while \DatasetsCommonsEs{} show even larger gaps with 389--507 versus 290--468 operations.
Constraint counts follow similar patterns,
with undetected mutants having 1.0--2.5$\times$ more constraints.
Even though both \VariantNaive{} and \VariantImproved{}
achieve better generalization outcomes for simpler constraints,
more complex constraints have a stronger detrimental effect on \VariantNaive{},
which produces 2-2.5$\times$ as many \texttt{Too\-Many\-Filter\-Misses\-Exceptions} as \VariantImproved{},
as discussed in more detail in Section~\ref{sec:limitations-eval}.

Constraint utilization rates show large differences across project types.
\DatasetsEqBenchEs{} achieve 47--70\% mean constraint utilization for detected mutants,
while \DatasetsCommonsEs{} achieve only 25--47\% mean utilization.
The higher utilization in \DatasetsEqBenchEs{} reflects their simpler constraint structures:
these projects primarily use basic numeric comparisons
that match \VariantImproved{}'s encoding capabilities.
\DatasetsCommonsEs{} projects contain more compound terms
and mathematical functions that are not modeled by \ToolTeralizer{},
reducing the percentage of constraints that can guide input generation.

These utilization differences explain the contrasting detection results.
In \DatasetsEqBenchEs{}, simple constraints enable effective boundary targeting for \VariantImproved{},
yet these same simple constraints make \VariantNaive{}'s random generation viable.
The higher constraint utilization even has detrimental effects
on \VariantImproved{} detection rates
because the focus on boundary testing detracts from testing of intermediate values.
As a result, detection rates for the very common \texttt{Math} mutations decrease,
causing overall detection rates to go down despite detection rates for most other mutants increasing.

\DatasetsCommonsEs{} projects present a different scenario.
Complex constraints reduce \VariantImproved{}'s constraint utilization to 25--47\%,
causing generalized tests to generate inputs from broader ranges
that overapproximate the true partition boundaries.
As a result, fewer partition boundaries are accurately identified,
and the number of generated inputs that need to be excluded during filtering increases.
Nevertheless, constraint utilization still reduces \texttt{Too\-Many\-Filter\-Misses\-Exception} failures
relative to \VariantNaive{} (Section~\ref{sec:limitations-eval}), which
enables \VariantImproved{} variants to achieve higher mutation detection rates
than \VariantNaive{} in 7 of 9 cases
despite its \texttt{Math} mutation detection disadvantage.

Three paths emerge to further enhance \VariantImproved{}'s effectiveness.
First, the \texttt{Math} mutation trade-off can be addressed through
higher \tries{} settings or balanced generation strategies
that maintain boundary detection advantages while improving arithmetic coverage.
Second, extending \ToolTeralizer{}'s constraint encoding support
to handle more complex constraints
would further increase utilization rates,
thus enabling more effective constraint-aware input generation.
However, encoding of non-boundary constraints would require custom input generators
that are more capable than those provided by \ToolJqwik{}.
Third, adaptive strategies could select generation approaches based on
measured constraint complexity and mutation distribution,
applying constraint-aware generation where it provides the largest benefit.

\rqanswerbox{2}{
  Both input generation strategies perform better on simpler constraints,
  but \VariantNaive{}'s effectiveness degrades more strongly as constraint complexity increases.
  On \DatasetsEqBenchEs{} projects with simpler constraints,
  \VariantNaive{} outperforms \VariantImproved{}
  because random generation satisfies many constraints by chance,
  while \VariantImproved{}'s boundary focus limits arithmetic diversity within the available \tries{},
  thus reducing \texttt{Math} mutation detection rates.
  On \DatasetsCommonsEs{} projects with more complex constraints,
  \VariantNaive{} generates substantially more inputs that violate constraints,
  causing more \texttt{Too\-Many\-Filter\-Misses\-Exception} failures.
  \VariantImproved{}'s constraint-aware generation reduces these failures,
  enabling it to outperform \VariantNaive{} in 7 of 9 cases despite its \texttt{Math} mutation disadvantage.
  Balancing boundary and non-boundary testing could combine the advantages of both strategies.
}

\subsection{RQ3: Runtime Requirements}
\label{sec:runtime-eval}

We measured \ToolTeralizer{}'s runtime across seven project variants
to assess viability compared to existing automated testing tools.
Since no existing tools perform automated test generalization from unit tests to property-based tests,
we compared efficiency against \ToolEvoSuite{}, the established test generation tool.
Runtime results were collected on a MacBook Air with M2 processor and 24~GB of memory.
Shared processing stages were executed only once per project by \ToolTeralizer{}
and collected specifications were then reused across all seven generalization variants.
Processing required a total time of 3.4 hours for \DatasetCommonsDev{},
8.2--9.8 hours for \DatasetsCommonsEs{} projects,
and 24.8--30.9 hours for \DatasetsEqBenchEs{} projects.
Despite the relatively high runtime cost,
Pareto analysis (Section~\ref{sec:evosuite-efficiency}) confirms
that combining low search budget \ToolEvoSuite{} generation and \ToolTeralizer{} generalization
often achieves better detection-to-runtime ratios than simply running \ToolEvoSuite{} with higher search budgets.
Within the overall processing pipeline,
validation through mutation testing consumes 40.5--81.1\% of total processing time.
This concentration of computational cost in validation
indicates that future optimizations should target
the mutation testing phase rather than transformation algorithms.


\subsubsection{Execution Time of \ToolTeralizer{}}
\label{sec:execution-time}

Runtime distribution across processing stages reveals
that validation dominates computational cost.
Figure~\ref{fig:teralizer-runtimes} shows
generalization validation consuming 1,110--35,538 seconds (40.5--78.5\% of total processing time)
while specification extraction requires 65--2,793 seconds (0.7--8.2\%)
and test transformation takes 5--173 seconds (0.1--0.5\%).
This resembles \ToolEvoSuite{}'s distribution,
where the JUnit checking phase,
which is most comparable to \ToolTeralizer{}'s
generalization validation stage,
consumes 86\%, 72\%, and 34\% of median runtime
for 1s, 10s, and 60s search budgets respectively.

\paragraph{Non-Validation Runtimes}

Non-validation stages complete efficiently
despite containing the core generalization logic.
Specification extraction (65--2,793 seconds) uses \ToolSPF{}
to concretely execute tests in constraint collection mode,
extracting path conditions and symbolic outputs
without requiring any constraint solving calls
(Section~\ref{sec:specification-extraction}).
While this introduces some overhead
because the JVM implementation of \ToolJPF / \ToolSPF{}
is less optimized than production-ready JVM implementations
(\ToolJPF{} itself runs inside of a host JVM process \cite{TODO}),
the cost is comparatively small, representing a one-time cost
of ca.\ 100$\times$ the runtime of an original test suite execution.
Test transformation cost is almost negligible at a one-time runtime cost
of only 5--173 seconds per generalization variant across all projects.
The reason for this low cost is that test transformation
only performs syntactic replacements, e.g.,
converting JUnit annotations to \ToolJqwik{} ones,
wrapping input constraint encodings in \texttt{TestParameters} classes,
and replacing expected values in assertions 
(Section~\ref{sec:test-transformation}).
Consequently, neither specification extraction nor test transformation
offer much opportunity for further runtime improvements.

\paragraph{Validation Runtimes}

Validation costs are substantially larger than non-validation costs
due to the high runtime cost of mutation testing,
which is by far the largest contributing factor in all validation stages.
For example, even on the \VariantOriginal{} test suite
without any added generalized tests,
execution of \VariantOriginal{} validation (701--3,898 seconds)
generally requires 1.5--10$\times$ as much runtime as specification extraction,
with mutation testing representing 88\% of the total runtime cost of this stage on average.
Execution times for \VariantInitial{} validation (652--4,131 seconds)
are generally slightly lower than for \VariantOriginal{} validation
because of the filtering that takes place
between the two stages (Section~\ref{sec:all-filtering}).
Validation of generalized test suites has higher runtime requirements (1,110--35,538 seconds)
than for the \VariantOriginal{} and \VariantInitial{} ones
for largely the same reasons that individual property-based \ToolJqwik{} tests
take longer to execute than conventional JUnit tests, i.e.,
\ToolJqwik{} overhead, larger number of tested inputs,
as well as filter-and-regenerate cycles
which occur more often for \VariantNaive{} variants
and in the presence of more complex input constraints (Section~\ref{sec:test-suite-execution-time}).

While it might seem tempting to forgo any validation,
this would result in considerable increases to the execution times of generalized test suites.
After all, one of the primary purpose of validation stages is
(i)~to provide a baseline for mutation detection comparisons
(\VariantOriginal{} / \VariantInitial{} validation),
and (ii)~to identify which generalized tests do not
measurably increase mutation detection rates (generalization validation),
thereby enabling \ToolTeralizer{} to remove non-contributing tests from the final generalized test suite.
In our evaluation, this filtering reduces the number of retained generalized tests
from a total of 65,633 to 4,240 across all projects and variants
(Section~\ref{sec:test-suite-test-count}),
thus demonstrating the impact
that validation-based filtering has on generalization outcomes.
Even though there is a non-negligible one-time cost associated with this,
that cost amortizes over time compared a longer-running test suite
that incurs further cost with every repeated execution.
Future optimization efforts could focus on validation efficiency
through faster mutation testing approaches
or lightweight pre-filtering heuristics
that identify likely-beneficial candidates before full validation.

\subsubsection{Efficiency of \ToolTeralizer{} vs.\ \ToolEvoSuite{}}
\label{sec:execution-efficiency}

\begin{table}[H]
  \caption{Total runtimes of Teralizer for all evaluated projects.}
  \label{tab:teralizer-runtimes}
  \begin{tabular}{lr}
    \toprule
    Project & Runtime \\
    \midrule
    \DatasetEqBenchA{} & 24h 46min 27s \\
    \DatasetEqBenchB{} & 28h 16min 32s \\
    \DatasetEqBenchC{} & 30h 53min 22s \\
    \midrule
    \DatasetCommonsA{} & 8h 13min 55s \\
    \DatasetCommonsB{} & 9h 46min 48s \\
    \DatasetCommonsC{} & 9h 04min 32s \\
    \midrule
    \DatasetCommonsDev{} & 3h 23min 22s \\
    \bottomrule
  \end{tabular}
\end{table}

\begin{figure}[H]
  \centering
  \includegraphics[width=\linewidth]{figures/fig_teralizer_runtimes}
  \caption{Teralizer runtimes per project, processing stage, and generalization variant.}
  %\Description{@TODO}
  \label{fig:teralizer-runtimes}
\end{figure}


\begin{samepage}
  \begin{figure}[H]
    \centering
    \includegraphics[width=\textwidth]{figures/fig_teralizer_efficiency}
    \caption{Pareto fronts for EvoSuite and Teralizer variants across projects.}
    \label{fig:teralizer-efficiency}
  \end{figure}
  \begin{table}[H]
    \begin{minipage}[t]{0.48\textwidth}
      \centering
      \begin{table}[H]
  \caption{Pareto points for project: eqbench.}
  \label{tab:pareto-eqbench}
  \begin{tabular}{rrlrr}
    \toprule
    Pt. & EvoSuite & Teralizer & Det. \% & Runtime (s) \\
    \midrule
    1 & 1s & - & 48.1 & 26,479 \\
    2 & 10s & - & 50.6 & 29,861 \\
    3 & 1s & NAIVE$_{10}$ & 50.7 & 36,728 \\
    4 & 1s & IMPROVED$_{50}$ & 51.4 & 37,457 \\
    5 & 1s & NAIVE$_{50}$ & 51.7 & 37,532 \\
    6 & 10s & IMPROVED$_{10}$ & 51.9 & 41,525 \\
    7 & 10s & IMPROVED$_{50}$ & 53.6 & 42,256 \\
    8 & 10s & NAIVE$_{50}$ & 53.8 & 45,398 \\
    9 & 10s & IMPROVED$_{200}$ & 53.8 & 48,269 \\
    10 & 10s & NAIVE$_{200}$ & 54.1 & 62,938 \\
    11 & 60s & IMPROVED$_{50}$ & 54.5 & 68,093 \\
    12 & 60s & NAIVE$_{50}$ & 54.7 & 68,782 \\
    13 & 60s & IMPROVED$_{200}$ & 54.8 & 75,081 \\
    14 & 60s & NAIVE$_{200}$ & 55.0 & 93,017 \\
    \bottomrule
  \end{tabular}
\end{table}
    \end{minipage}
    \hfill
    \begin{minipage}[t]{0.48\textwidth}
      \centering
      \begin{table}[H]
  \caption{Pareto points for project: commons-utils.}
  \label{tab:pareto-commons}
  \begin{tabular}{rrlrr}
    \toprule
    Pt. & EvoSuite & Teralizer & Det. \% & Runtime (s) \\
    \midrule
    1 & 1s & - & 56.8 & 4648.7 \\
    2 & 10s & - & 57.3 & 5597.3 \\
    3 & 1s & IMPROVED$_{10}$ & 57.9 & 7294.5 \\
    4 & 60s & - & 58.1 & 10239.8 \\
    5 & 10s & NAIVE$_{10}$ & 58.1 & 10445.1 \\
    6 & 10s & IMPROVED$_{50}$ & 58.4 & 10603.4 \\
    7 & 10s & IMPROVED$_{10}$ & 58.4 & 11081.5 \\
    8 & 10s & IMPROVED$_{200}$ & 58.5 & 13270.4 \\
    9 & 60s & IMPROVED$_{10}$ & 59.3 & 13938.7 \\
    10 & 60s & IMPROVED$_{50}$ & 59.4 & 14727.5 \\
    11 & 60s & IMPROVED$_{200}$ & 59.5 & 15735.7 \\
    \bottomrule
  \end{tabular}
\end{table}
    \end{minipage}
  \end{table}
\end{samepage}




\subsection{RQ4 (Part 1 of 2): Causes of Unsuccessful Generalizations in the Evaluation Dataset}
\label{sec:filtering-eval}

In this section, we describe causes of unsuccessful generalizations in the main
evaluation dataset (commons-utils + evosuite-variants, eqbench + evosuite variants).
To get more generalizable results that enable better informed decisions about future
research directions, we also evaluated generalization success vs. failure in <number>
additional open source projects beyond the main evaluation dataset. Results of the
extended evaluation are discussed in Section~\ref{sec:filtering-eval-extended}.

overall test / assertion / generalization exclusions

\begin{table}[H]
  \caption{Included and excluded counts by variant and level.}
  \label{tab:exclusions-summary}
  \begin{tabular}{llrrr}
    \toprule
    Variant & Type & Total & \multicolumn{1}{c}{Included} & \multicolumn{1}{c}{Excluded} \\
    \midrule
    \VariantShared{} & Test & 23,246 & 19,306\; (83.1\%) & 3,940\; (16.9\%) \\
    \midrule
    \VariantShared{} & Assertion & 28,923 & 13,836\; (47.8\%) & 15,087\; (52.2\%) \\
    \midrule
    \VariantBaseline{} & Generalization & 13,836 & 13,814\; (99.8\%) & 22\; (\phantom{0}0.2\%) \\
    \VariantNaiveA{} & Generalization & 13,836 & 10,743\; (77.6\%) & 3,093\; (22.4\%) \\
    \VariantNaiveB{} & Generalization & 13,836 & 9,964\; (72.0\%) & 3,872\; (28.0\%) \\
    \VariantNaiveC{} & Generalization & 13,836 & 9,881\; (71.4\%) & 3,955\; (28.6\%) \\
    \VariantImprovedA{} & Generalization & 13,836 & 11,788\; (85.2\%) & 2,048\; (14.8\%) \\
    \VariantImprovedB{} & Generalization & 13,836 & 11,660\; (84.3\%) & 2,176\; (15.7\%) \\
    \VariantImprovedC{} & Generalization & 13,836 & 11,597\; (83.8\%) & 2,239\; (16.2\%) \\
    \bottomrule
  \end{tabular}
\end{table}

filtering-based test / assertion / generalization exclusions

\begin{table}
  \caption{Filtering results for tests and assertions in the \DatasetsCommons{} and \DatasetsEqBenchEs{} projects.}
  \label{tab:exclusions-filtering}
  \begin{tabular}{llrrrr}
    \toprule
    Level & Filter Name & Total & \multicolumn{1}{c}{Accept} & \multicolumn{1}{c}{Defer} & \multicolumn{1}{c}{Reject} \\
    \midrule
    Test & NonPassingTest & 23,246 & 21,719\; (93.4\%) & - & 1,527\; (\phantom{0}6.6\%) \\
    Test & TestType & 23,246 & 23,066\; (99.2\%) & - & 180\; (\phantom{0}0.8\%) \\
    Test & NoAssertions & 21,532 & 19,306\; (89.7\%) & - & 2,226\; (10.3\%) \\
    \midrule
    Assertion & AssertionType & 28,923 & 28,180\; (97.4\%) & - & 743\; (\phantom{0}2.6\%) \\
    Assertion & ExcludedTest & 28,923 & 27,326\; (94.5\%) & - & 1,597\; (\phantom{0}5.5\%) \\
    Assertion & MissingValue & 28,923 & 21,766\; (75.3\%) & - & 7,157\; (24.7\%) \\
    Assertion & ParameterType & 28,923 & 17,835\; (61.7\%) & 6,630\; (22.9\%) & 4,458\; (15.4\%) \\
    Assertion & VoidReturnType & 28,923 & 21,763\; (75.2\%) & 7,157\; (24.7\%) & 3\; (\phantom{0}0.0\%) \\
    \bottomrule
  \end{tabular}
\end{table}

failing tests

\begin{table}[H]
  \caption{Number of test execution failures by exception type and (generalization) variant.}
  \label{tab:exclusions-test-fails}
  \begin{tabular}{lrrrrrrrr}
    \toprule
    Variant & ORIGINAL & BASELINE & \multicolumn{3}{c}{NAIVE} & \multicolumn{3}{c}{IMPROVED} \\
    \cmidrule(lr){4-6}
    \cmidrule(lr){7-9}
    Tries & - & - & 10 & 50 & 200 & 10 & 50 & 200 \\
    \midrule
    ArithmeticException & 0 & 0 & 99 & 99 & 99 & 57 & 58 & 58 \\
    AssertionFailedError & 132 & 22 & 729 & 803 & 819 & 752 & 845 & 866 \\
    NumberFormatException & 0 & 0 & 0 & 0 & 0 & 18 & 18 & 18 \\
    TooManyFilterMissesException & 0 & 0 & 2233 & 2938 & 3005 & 1189 & 1223 & 1265 \\
    \bottomrule
  \end{tabular}
\end{table}


@TODO: Remove Table~\ref{tab:exclusions-spf}. Describe SPF execution failures in text only.

\begin{table}[H]
  \caption{Number of SPF execution failures by error type.}
  \label{tab:exclusions-spf}
  \begin{tabular}{lrr}
    \toprule
    Error Type & Total & Percent \\
    \midrule
    SPF exception & 1540 & 51.42 \\
    PC size limit exceeded & 790 & 26.38 \\
    Depth limit exceeded & 524 & 17.50 \\
    Teralizer exception & 97 & 3.24 \\
    Execution timeout & 28 & 0.93 \\
    OutOfMemoryError & 16 & 0.53 \\
    \bottomrule
  \end{tabular}
\end{table}

\subsection{RQ4 (Part 2 of 2): Causes of Unsuccessful Generalizations in Other Open Source Projects}
\label{sec:filtering-eval-extended}

This evaluation only uses the IMPROVED$_{200}$
generalization variant, but results also apply to all other variants.

\begin{table}[H]
  \caption{Number of processing failures and remaining projects per processing stage.}
  \label{tab:processing-failures-per-stage}
  \begin{tabular}{l r r}
    \toprule
    Processing Stage & Failures & Remaining Projects \\
    \midrule
    Total projects & - & 1160\; (\phantom{.}100 \%) \\
    SETUP\_PROJECT & 355 & 805\; (69.4 \%) \\
    BUILD\_PROJECT\_ORIGINAL & 189 & 616\; (53.1 \%) \\
    BUILD\_SPOON\_MODEL & 8 & 608\; (52.4 \%) \\
    EXECUTE\_TESTS\_ORIGINAL & 61 & 547\; (47.2 \%) \\
    COLLECT\_JUNIT\_REPORTS\_ORIGINAL & 31 & 516\; (44.5 \%) \\
    BUILD\_PROJECT\_INSTRUMENTED & 1 & 515\; (44.4 \%) \\
    EXECUTE\_TESTS\_INITIAL & 130 & 385\; (33.2 \%) \\
    COLLECT\_JACOCO\_DATA\_INITIAL & 41 & 344\; (29.7 \%) \\
    COLLECT\_PIT\_DATA\_INITIAL & 64 & 280\; (24.1 \%) \\
    COLLECT\_PIT\_DATA\_GENERALIZED & 270 & 10\; (\phantom{0}0.9 \%) \\
    Successfully processed & - & 10\; (\phantom{0}0.9 \%) \\
    \bottomrule
  \end{tabular}
\end{table}

\begin{table}[H]
  \caption{Causes of processing failures per processing stage.}
  \label{tab:processing-failure-causes}
  \begin{tabularx}{\textwidth}{l X}
    \toprule
    Processing Stage & Causes of Processing Failures \\
    \midrule
    SETUP\_PROJECT & dependency resolution error (329), sources / tests not found (26) \\
    BUILD\_PROJECT\_ORIGINAL & compilation error (171), compilation outputs not found (18) \\
    BUILD\_SPOON\_MODEL & Spoon execution error (8) \\
    EXECUTE\_TESTS\_ORIGINAL & JUnit execution error (13), timeout exceeded (48) \\
    COLLECT\_JUNIT\_REPORTS\_ORIGINAL & JUnit outputs not found (31) \\
    BUILD\_PROJECT\_INSTRUMENTED & compilation error (1) \\
    EXECUTE\_TESTS\_INITIAL & all tests excluded (129), timeout exceeded (1) \\
    COLLECT\_JACOCO\_DATA\_INITIAL & JaCoCo execution error (1), JaCoCo outputs not found (40) \\
    COLLECT\_PIT\_DATA\_INITIAL & PIT execution error (16), PIT outputs not found (4), all classes excluded (3), failed to map PIT data to a test (1), timeout exceeded (40) \\
    COLLECT\_PIT\_DATA\_GENERALIZED & all generalizations excluded (269), failed to map PIT data to a generalization (1) \\
    \bottomrule
  \end{tabularx}
\end{table}


Broadly speaking, there are three main ways to increase the number of projects
that can be successfully processed: (i)~reducing exclusions caused by filtering
(129 + 3 + 269 = 401 projects), (ii)~adding support for more varied project structures
(26 + 18 + 31 + 40 + 4 = 119 projects), and (iii)~increasing timeouts (48 + 1 + 40 = 89 projects).
Depedency resolution and compilation errors in the original project code (329 + 171 = 500 projects)
as well as external tool errors (8 + 13 + 1 + 16 = 38 projects) are less actionable.
Furthermore, Teralizer errors only occur in a small number of cases (1 + 1 + 1 = 3 projects),
so also offer comparatively little opportunity for improvements.

\begin{table}[H]
  \caption{Filtering results of the extended dataset for tests, assertions, and generalizations.}
  \label{tab:exclusions-filtering-extended}
  \begin{tabular}{lllrrrr}
    \toprule
    Variant & Type & Filter Name & Total & \multicolumn{1}{c}{Accept} & \multicolumn{1}{c}{Defer} & \multicolumn{1}{c}{Reject} \\
    \midrule
    \VariantShared{} & Test & NonPassingTest & 74,332 & 65,579\; (88.2\%) & - & 8,753\; (11.8\%) \\
    \VariantShared{} & Test & TestType & 74,332 & 65,052\; (87.5\%) & - & 9,280\; (12.5\%) \\
    \VariantShared{} & Test & NoAssertions & 56,853 & 33,390\; (58.7\%) & - & 23,463\; (41.3\%) \\
    \midrule
    \VariantShared{} & Assertion & AssertionType & 122,166 & 92,996\; (76.1\%) & - & 29,170\; (23.9\%) \\
    \VariantShared{} & Assertion & ExcludedTest & 122,166 & 101,519\; (83.1\%) & - & 20,647\; (16.9\%) \\
    \VariantShared{} & Assertion & MissingValue & 122,166 & 51,430\; (42.1\%) & - & 70,736\; (57.9\%) \\
    \VariantShared{} & Assertion & ParameterType & 122,166 & 5,393\; (\phantom{0}4.4\%) & 56,484\; (46.2\%) & 60,289\; (49.4\%) \\
    \VariantShared{} & Assertion & ReturnType & 122,166 & 11,650\; (\phantom{0}9.5\%) & 70,736\; (57.9\%) & 39,780\; (32.6\%) \\
    \midrule
    \VariantImprovedC{} & Generalization & NonPassingTest & 229 & 206\; (90.0\%) & - & 23\; (10.0\%) \\
    \bottomrule
  \end{tabular}
\end{table}

The most common filtering reasons are unsupported parameter and return types.

@TODO: Better explain these filtering results by prooviding information about:
(i) which types are identified in the tested method signatures, and
(ii) which assertions are used in the tests.

Successful generalizations did not improve mutation scores in any of the successfully
processed projects. However, some generalizations indicate that there weak
preconditions are used in the source / test code of the original implementation.

@TODO: Check which generalizations fail the "NonPassingTest" filter due to (i) bugs in
our generalization implemenation vs. (ii) weak preconditions.

% Performance improvements of Teralizer itself can NOT be used as a substitute for
% (i) because the timeouts occur for tasks that are performed by external tools.

\section{Discussion}
\label{sec:discussion}

Our evaluation demonstrates that
semantics-based test generalization via symbolic analysis
is viable but currently constrained to
specific application environments and test architectures.
Under controlled conditions that match current tool capabilities,
\ToolTeralizer{} successfully generalizes 40.1\% of assertions (RQ5)
and improves mutation detection by 1--4 percentage points (RQ1).
As a post-processing step for generated tests,
generalization offers competitive efficiency.
For example, combining 1-second \ToolEvoSuite{} generation with \ToolTeralizer{}'s generalization
achieves comparable mutation detection to 60-second generation alone
while reducing total processing time by 31.9\% (RQ4).

However, fully automated generalization of real-world projects
faces substantial barriers (RQ6):
only 0.6\% of assertions pass early analysis and filtering stages
(versus 47.8\% under controlled conditions),
only 0.2\% of assertions successfully generalize,
and only 1.7\% of real-world projects complete the processing pipeline.
This section discusses when and why generalization succeeds (Section~\ref{sec:when-it-works}),
when and why it fails (Section~\ref{sec:when-it-fails}),
and how the effectiveness, efficiency, and applicability
of semantics-based test generalization can be improved
through combined efforts of the scientific research community
as well as practitioners who are interested in applying
test generalization in their own projects.

\subsection{When and Why Generalization Succeeds}
\label{sec:when-it-works}

Generalization succeeds when implementation and test properties
of the target projects align
with both \ToolTeralizer{}'s current static analysis capabilities
as well as the capabilities of \ToolSPF{}'s symbolic analysis.
Implementation code that is amenable to generalization
is primarily focused on numeric computations
through pure deterministic functions without any side effects
(thus enabling symbolic analysis)
and is organized in standard project structures that facilitate detection
of required output artifacts such as compilation outputs
as well as (mutation) testing and coverage reports.
Tests amenable to generalization are single-assertion unit tests
without complex setup logic, loops, or interprocedural control flow.
Together, these structural properties enable
reliable static analysis for identifying tested methods
and extracting path-exact specifications through \ToolSPF{}'s single-path symbolic analysis.

When the described conditions are satisfied,
\ToolTeralizer{} achieves moderate mutation score improvements at reasonable runtime cost.
Generalization increases mutation scores by 1.2--3.9 percentage points in \DatasetsEqBenchEs{} projects
and by 0.82--1.33 percentage points in \DatasetsCommonsEs{} projects
compared to \ToolEvoSuite{}-generated baselines (Figure~\ref{fig:mutation-detection-results}).
Beyond effectiveness, generalization offers competitive efficiency when combined with test generation,
optimizing complementary dimensions where \ToolEvoSuite{} achieves breadth through coverage-guided exploration
and \ToolTeralizer{} achieves depth through partition-based input generation within already-covered test execution paths.
For example, 1-second \ToolEvoSuite{} generation combined with \VariantNaiveB{} generalization
achieves 51.7\% detection in 37,532 seconds, thus outperforming 60-second generation alone
which achieves 51.6\% detection in 55,075 seconds (Figure~\ref{fig:teralizer-efficiency}).

Outcomes vary based on constraint complexity and
original test suite effectiveness of the target projects.
More complex constraints hinder mutation score improvements
because they increase the number of \texttt{Too\-Many\-Filter\-Misses\-Exceptions}.
Similarly, stronger original test suites leave less room for mutation score improvements.
For example, \VariantNaive{} and \VariantImproved{}
both show larger mutation score improvements
for \DatasetsEqBenchEs{} than for \DatasetsCommonsEs{}
because of the simpler constraints in \DatasetsEqBenchEs{}
(137--231 vs. 290--507 average operation counts, Table~\ref{tab:mutation-detection-comparison}),
and larger mutation score improvements for \DatasetsCommonsEs{} than for \DatasetCommonsDev{}
because of \DatasetCommonsDev{}'s stronger original tests
(56.77--58.12\% vs. 80.35\% \VariantInitial{} mutation detection rate, Figure~\ref{fig:mutation-detection-results}).

\VariantNaive{} is more effective than \VariantImproved{} for simpler constraints
(Figure~\ref{fig:mutation-detection-results}, rows 1--3),
whereas \VariantImproved{} is more effective than \VariantNaive{} for more complex constraints
(Figure~\ref{fig:mutation-detection-results}, rows 4--6).
There are two primary factors that explain these results (Section~\ref{sec:constraint-complexity-eval}).
First, simpler constraints are easier to satisfy by chance.
Consequently, the difference in observed \texttt{TooManyFilterMissesExceptions}
between \VariantNaive{} and \VariantImproved{} is smaller in such cases
than for more complex constraints.
Second, simpler constraints enable \VariantImproved{}
to more reliably encode input partition boundaries.
As a result, it spends more \tries{} on boundary testing
but neglects non-boundary testing,
thus limiting overall mutation detection improvements.
This effect is particularly pronounced at lower \tries{} settings
where \VariantImprovedA{} noticeably underperforms
all other generalization strategies
(Figure~\ref{fig:mutation-detection-results}, rows 1--3).

\subsection{When and Why Generalization Fails}
\label{sec:when-it-fails}

Fully automated generalization fails
for the vast majority of evaluated real-world projects.
Overall, only 1.7\% of these projects (11 of 632)
successfully pass the full processing pipeline
(Table~\ref{tab:processing-failures})
and only 0.2\% of assertions (206 of 122,153)
are successfully generalized
(Table~\ref{tab:exclusions-breakdown-extended}).
Exclusions occur primarily during processing of
Stage~1~+~2 (project analysis, 79.4\% exclusion rate)
as well as Stage 5 (test suite reduction, 90.4\% exclusion rate).
In contrast, processing of Stage~3 (specification extraction) and 4 (generalized test creation)
succeeds in most cases, showing exclusion rates of only 10.0\% and 2.6\%, respectively.
These results indicate that the core generalization mechanism operates reliably 
when suitable generalization candidates are encountered.
However, early filtering for such candidates
and late-stage identification of generalizations that
improve overall test suite effectiveness
reflect significant barriers to practical adoption.

Broadly speaking, there are three high-level causes
that explain the high exclusion rates we observe
when evaluating fully automated test generalization via \ToolTeralizer{} in real-world projects.
First, \ToolTeralizer{} is a research prototype,
which correspondingly limits the scope of its current capabilities.
Second, extracting accurate specifications for generalization
of test oracles is a fundamentally non-trivial problem,
even more so when moving beyond the domain of pure functions and numeric-focused programs.
Third, factors such as execution errors in \ToolTeralizer{}'s dependencies,
timeouts enforced for evaluation purposes,
and test failures in original test suites
are beyond the direct control of the generalization mechanism itself,
but still increase the number of unsuccessful generalization attempts. 
Throughout the rest of this section,
we describe how these three high-level causes affect generalization outcomes.
This summary of failure causes then serves as the basis for discussion of future improvements
in Section~\ref{sec:how-to-improve}.

\paragraph{Implementation Limitations}

There are four limiting factors in our current prototype
that are implementation specific rather than fundamental:
(i)~it only supports JUnit 4 and JUnit 5 tests and assertions,
(ii)~it only supports generalization of tests that contain at least one assertion,
(iii)~it only performs intraprocedural static analysis within test methods to detect assertions,
and (iv)~it only supports projects that use default output directories and formats
for compilation outputs, JUnit reports, JaCoCo reports, and PIT reports.
Limitation (iv) directly causes
95 project-level exclusions (15.0\% of projects)
due to output detection and processing failures
(Table~\ref{tab:processing-failures}, rows \#4, \#5, \#15, \#16, and \#17).
Limitations (i)--(iii) contribute to the exclusion of
129 projects (20.4\%) for which all tests are excluded
(Table~\ref{tab:processing-failures}, row~\#2)
and 266 projects (42.1\%) for which all assertions are excluded
(Table~\ref{tab:processing-failures}, rows \#1 and \#8).

While the exact degree of influence
on the 129 test- and 266 assertion-related exclusions
is difficult to quantify precisely
(because both are also affected by the other two high-level factors),
filter rejections provide at least an approximate measure.
Limitation (i) causes all 12.5\% of \texttt{TestType} rejections
(Table~\ref{tab:exclusions-filtering-extended})
because these tests use JUnit~3.
Furthermore, limitation (ii) causes all true positive \texttt{NoAssertions} rejections
and (i)+(iii) cause all false positive \texttt{NoAssertions} rejections,
together accounting for the exclusion of 41.3\% of tests
(Table~\ref{tab:exclusions-filtering-extended}).
Thus, limitations (i)--(iii) have comparatively high impact
on the 129 test-related exclusions
--- the only other test-excluding factor is 11.8\% \texttt{NonPassingTest} rejections.
In contrast, their influence on the 266 assertion-related exclusions is comparatively low,
contributing only to 16.9\% \texttt{ExcludedTest} rejections
--- the lowest rate among all assertion-level filters
(Table~\ref{tab:exclusions-filtering-extended}).

\section{Related Work}
\label{sec:related-work}

Our work draws on ideas from test amplification and symbolic analysis
to automate the transformation from conventional unit tests to property-based tests.
This section reviews prior approaches to test generalization
(Section~\ref{sec:rw-generalization}),
discusses research approaches and directions that could
improve specification inference capabilities of our current prototype
(Section~\ref{sec:rw-inference})
and explores synergies with related techniques as well as developer perspectives
(Section~\ref{sec:rw-synergies}).

\subsection{Test Generalization}
\label{sec:rw-generalization}

Property-based testing~\cite{claessen_2000_quickcheck} and
parameterized unit testing~\cite{tillmann_2005_parameterized}
enable multi-input validation through general properties,
differing primarily in input generation strategy:
property-based tests (PBTs) traditionally use random generation to produce inputs,
whereas \citeauthor{tillmann_2008_pex}
suggest to execute parameterized unit tests (PUTs) symbolically,
utilizing constraint solving to select inputs
for test parameters~\cite{tillmann_2008_pex}.
Both approaches require developers
to manually specify general assertions
that hold across ranges of inputs
rather than specific input-output examples
used in conventional unit tests (CUTs).
\citeauthor{thummalapenta_2011_retrofitting}~\cite{thummalapenta_2011_retrofitting}
demonstrated manual strategies for retrofitting CUTs to PUTs.

\citeauthor{fraser_2011_generating_put}~\cite{fraser_2011_generating_put}
automated generation of PUTs from CUTs, but use tests without existing assertions
as a starting point.
This sidesteps the problem of automated oracle generalization.
However, it often causes generated PUTs to overfit the implementation~\cite{fraser_2011_generating_put}
because the lack of validated oracles 
makes it difficult to distinguish
intentional behavior from incidental state changes or outputs.
PROZE~\cite{tiwari_2024_proze} uses runtime inputs and outputs to transform CUTs to PUTs
but does not generalize beyond observed values.
JARVIS~\cite{peleg_2018_jarvis} introduced automated CUT-to-PBT transformation
using black-box analysis with predefined abstraction templates,
which produces overapproximations that require multiple related tests to constrain.
We instead use white-box symbolic analysis along concrete execution paths,
extracting path-exact specifications that generalize oracles
from individual input-output examples.

\subsection{Specification Inference}
\label{sec:rw-inference}

% Paragraph 1: Type support advances for symbolic analysis
Type support limitations fundamentally constrain
specification extraction through single-path symbolic analysis,
as discussed in Sections~\ref{sec:specification-extraction},
\ref{sec:limitations-eval-extended}, and~\ref{sec:when-it-fails}.
These limitations stem from our reliance on \ToolSPF{}~\cite{pasareanu_2013_symbolic},
a symbolic execution tool designed for path exploration.
Because full symbolic execution requires constraint solving
to determine path feasibility,
SPF only encodes constraints for types with adequate solver support.
Extending solver capabilities remains an active research area,
with recent work showing improvements for
string constraints~\cite{chen_2024_z3noodler,lotz_2025_s2s,chen_2025_ostrich2},
heap-allocated structures~\cite{copia_2022_lissa,copia_2023_pli,braione_2016_jbse},
arrays~\cite{niemetz_2023_bitwuzla},
and floating-point arithmetic~\cite{yang_2025_floating_point}.
As solver support improves and symbolic execution tools
correspondingly extend their constraint encoding,
semantics-based test generalization would also benefit.

% Paragraph 2: Alternative specification inference
Alternative approaches to specification inference
largely avoid type support limitations inherent to symbolic analysis,
but infer general specifications that describe overall method behavior
rather than path-exact constraints, which complicates oracle generalization.
Houdini~\cite{flanagan_2001_houdini} pioneered template-based inference,
generating candidate annotations and using verification to filter them.
Daikon~\cite{ernst_2007_daikon} introduced dynamic invariant detection from execution traces.
More recent tools target specific specification types:
EvoSpex~\cite{molina_2023_evospex} uses evolutionary search to infer postconditions,
SpecFuzzer~\cite{molina_2022_specfuzzer} combines grammar-based fuzzing with mutation analysis
for class specifications,
and PreCA~\cite{menguy_2022_preca} employs constraint acquisition~\cite{bessiere_2017_constraint_acquisition}
to infer preconditions from input-output observations.
LLM-based techniques offer yet another path:
SpecGen~\cite{ma_2025_specgen} uses conversational prompting
with mutation-based refinement to generate specifications from source code,
whereas ClassInvGen~\cite{sun_2025_classinvgen} co-evolves class invariants with test inputs.

\subsection{Synergies and Developer Perspective}
\label{sec:rw-synergies}

% Paragraph 1: Synergies with test generation and oracle inference
Test generalization builds on existing tests and their assertions,
creating natural synergies with techniques that produce or enrich them.
Test generation tools such as
\ToolEvoSuite{}~\cite{fraser_2011_evosuite}, Randoop~\cite{pacheco_2007_randoop},
and UTBot~\cite{utbot_2024_sbft}
produce complete unit tests through search-based, random, and hybrid approaches,
while DSpot~\cite{danglot_2019_dspot} amplifies existing tests to cover additional branches.
Oracle inference techniques such as
TOGA~\cite{dinella_2022_toga}, TOGLL~\cite{hossain_2024_togll}, and AsserT5~\cite{primbs_2025_assert5}
add assertions to tests that lack them.
All of these expand the pool of available generalization candidates.
RQ4 demonstrates this combination:
pairing \ToolEvoSuite{}'s generation with \ToolTeralizer{}'s generalization
achieves higher mutation scores at lower runtimes than test generation alone.
However, generated tests and inferred oracles
risk overfitting the implementation
rather than capturing intended specifications~\cite{barr_2015_oracle},
and this risk carries through to any subsequent generalization.

% Paragraph 2: Developer perspective and PBT barriers
For use cases beyond fully automated pipelines,
developer interaction with generalized tests becomes relevant.
Studies of test amplification show that developers filter and edit amplified tests extensively
before adding them to their test suites~\cite{wessel_2024_shaken,brandt_2022_developer}.
By making minimal structural changes
--- parameterizing inputs and expected values
while preserving the original test logic ---
test generalization may reduce friction
compared to approaches that generate entirely new test code.
However, property-based testing introduces its own complexity:
moving from example-based to property-based thinking
requires a conceptual shift that can be difficult for developers~\cite{goldstein_2024_pbt_practice,hughes_2016_experiences}.
Thus, generator constraints and generalized oracles that replace concrete values
must be presented appropriately for developers to understand and trust them.
Improving understandability of generalized tests therefore represents a research direction
that must be tackled to better support use cases
outside of fully automated testing scenarios.

\section{Conclusions}
\label{sec:conclusions}

This paper introduced a semantics-based approach
for automated test generalization,
using specifications extracted through single-path symbolic analysis
to transform conventional unit tests into property-based tests.
We implemented this approach in a prototype tool called \ToolTeralizer{}.
Under controlled conditions matching current symbolic analysis capabilities,
\ToolTeralizer{} achieves mutation score improvements of 1--4 percentage points
compared to \ToolEvoSuite{}-generated baselines.
However, our systematic evaluation across 630+ real-world Java projects
from the RepoReapers dataset
reveals substantial barriers to fully automated generalization under real-world conditions:
only 1.7\% of projects complete the processing pipeline,
and 98.3\% of assertions are excluded before reaching generalized test creation.
By analyzing these exclusions in detail,
we distinguish engineering limitations of our prototype
from fundamental research challenges in specification extraction,
providing concrete guidance for advancing the field.

The primary barrier to fully automated generalization is limited type support in existing symbolic analysis tools and approaches:
current tools cannot precisely encode constraints for strings, collections, and objects,
causing the majority of assertion-level exclusions.
As symbolic analysis improves to support additional types,
semantics-based test generalization would directly benefit.
In contrast, many other limitations of \ToolTeralizer{}
are addressable through engineering improvements
without requiring research advances:
interprocedural analysis would recover assertions in helper methods,
broader framework support would reduce test-level exclusions,
and extended constraint encoding in generated tests would improve effectiveness and efficiency
by reducing filter-and-regenerate cycles.
Our complete implementation and replication package
is publicly available to support reproduction and extension of this work~\cite{replicationpackage}.


%%
%% The acknowledgments section is defined using the "acks" environment
%% (and NOT an unnumbered section). This ensures the proper
%% identification of the section in the article metadata, and the
%% consistent spelling of the heading.
\begin{acks}
This research was funded in whole or in part by the Austrian Science Fund (FWF) 10.55776/P36698. For open access purposes, the author has applied a CC BY public copyright license to any author accepted manuscript version arising from this submission.
\end{acks}

%%
%% The next two lines define the bibliography style to be used, and
%% the bibliography file.
\bibliographystyle{ACM-Reference-Format}
\bibliography{main}

\end{document}
\endinput
%%
%% End of file `sample-manuscript.tex'.
